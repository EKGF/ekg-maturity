\pagebreak
\chapter{Maturity Model in Essence}\label{ch:maturity-model-in-essence}

\section{Essence language, kernel, practices and methods}\label{subsec:essence-language-kernel-practices-and-methods}

\iindex{Essence} is a radical approach to software engineering methods originated by \gls{semat},
and now an \gls{omg}\autocite{omgwebsite} standard.
Essence identifies the essentials of software engineering common to all software development endeavours.
These essentials do not prescribe a method or practices for software engineering.
The essentials are expressed in a visual and textual language with which a common kernel of software engineering
is described\,---\,the essence of software engineering.
The Essence kernel is actionable\,---\,it can be used as is to run a software engineering endeavour albeit
with an assumption of significant tacit knowledge of software development.
However, the kernel is extensible using the language and this allows practices to be described which give more detail
and explicit descriptions.
A collection of practices can be used together to describe a method.

\section{\glsfmtshort{ekgmethod}}\label{subsec:ekg-method}

Recognizing the advantage of the Essence approach, the \agnos \gls{ekgmethod} is described as
a set of practices to extend the Essence kernel.
Now that Essence exists, it would be almost irresponsible to introduce a new method in any other way.

\section{Capability Maturity Models}\label{subsec:capability-maturity-models}

The Essence language is independent of the Essence Kernel for software engineering;
it is possible to describe kernels for other kinds of endeavours.
Closely related to software engineering are the various capability maturity models, which are used to measure and
guide improvements in organisations' software engineering capability.
To date, maturity models have been described using their own terminology, with many similar issues around a lack of
standardisation that led to the creation of the \iindex{SEMAT} initiative.
Where a method is described using Essence, the application of a maturity model described using its own
terminology would undo many of the benefits offered by Essence.

\section{\glsfmtshort{ekgmm} Essence kernel and practices}\label{subsec:ekg-mm-essence-kernel-and-practices}

To address the incongruity, describe above, an experiment has been made to express the \glsfmtshort{ekgmm}
using Essence.
To do this, using the Essence language a new kernel and practices have been described.
This initial mapping has not changed any of the content, but rather has mapped the components of
the \glsfmtshort{ekgmm} to Essence language elements to determine the "fit".
The diagram below shows a top level view of the mapping, and the following paragraphs describe the initial mappings.

\centerimg[width=\dimexpr\textwidth+3pt\relax,height=300pt]{../ekg-mm/images/KGMM-Levels-Essence.png}

This diagram is a graphical representation of \glsfmtshort{ekgmm} expressed in Essence\index{essence},
showing the areas of concern, alphas\index{alpha} and levels in the style used for example multi-phase methods
in the \iindex{OMG} standard.
The states shown for the alphas are examples for the diagram only\,---\,the complete state models for the alphas
is work in progress.

The shaded background shows the four areas of concern, identified by colors:

\begin{enumerate}
    \item Business (green)
    \item Data (blue)
    \item Technology (yellow)
    \item Organisation (grey)
\end{enumerate}

Across the top, separated into the appropriate Area of Concern, are the alphas.
These are the intangible concepts that we need to progress through maturity.
The states of the alphas are progressed by performing Activities (not shown in this diagram).
The levels of maturity are shown as the horizontal bands across the Areas of Concern and alphas.
The red triangles indicate checkpoints that need to be achieved to reach each level.
The checkpoints are specified as required states for each alpha.
These required states are shown on the lifelines for alpha.

\begin{description}[nosep,font=\bfseries]

  \item [areas of concern]
  The \gls{ekgmm} pillars are mapped to Essence areas of concern.
  The rationale for this mapping is that these \gls{ekgmm} pillars serve the same purpose as
  Essence areas of concern, groupings of elements that share the same dimensional focus.

  \item [alphas]
  The \gls{ekgmm} components are mapped to Essence alphas.
  The rationale for this mapping is that these \gls{ekgmm} components serve the same purpose as
  Essence alphas (abstract level progress health attribute)\,---\,the abstract representation of the subjects
  we wish to progress through our endeavour through the use of states.

  \item [practices]
  The \gls{ekgmm} maturity levels are mapped to Essence practices.
  The rationale for this mapping is that depending on the maturity level, the \gls{ekgmm} capabilities
  require changed or additional processes to be performed with different work product states resulting.
  A practice for each maturity level can define activities that realise the activity space at the relevant level.

  \item [activity spaces, activities and work products]
  The Capabilities in the \gls{ekgmm} are mapped to activity spaces and activities.
  The rationale for this mapping is that each \gls{ekgmm} capability can be represented as an Essence activity space,
  which is an abstract placeholder for activities to be performed.
  The activities to be performed (which includes the work products to progressed through their levels of detail)
  will be specified in the practices for each level.

  \item [checkpoint pattern]
  The maturity levels in the \gls{ekgmm} are mapped to sets of alpha states, described using the
  Essence checkpoint pattern.
  The rationale for this mapping is the alpha states will indicate the individual component maturities and
  the set of maturities required for a maturity level can then be easily expressed using the Essence pattern.

\end{description}
