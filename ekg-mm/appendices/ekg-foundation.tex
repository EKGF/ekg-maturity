\chapter{\glsfmtshort{ekg} Foundation}
\label{ch:ekg-foundation}
\index{\glsfmtshort{ekg}!foundation!see {EKGF}}
\index{EKGF}

This document\,---\,the \glsfirst{ekgmm}\,---\,is one of the key artifacts produced by the \glsxtrfull{ekgf}.

\paragraph{Vision and Expectation Management}

It captures our thinking around \myuline{the vision of \gls{ekg}},
how it affects your business and what can be expected of it at certain levels of capability maturity.

\paragraph{\Glsfmtlongpl{nfr}}

The \gls{ekgmm} is therefore seen as the overall framework for all \enquote{\glspl{nfr}}\,---\,requirements\,---\,%
clustered per capability\,---\,that are not \myuline{directly} related to any given "use case".

\paragraph{\glsfmtlong{uct}}

For instance, if it is your company's strategy\footnote{See \Nameref{ch:ekg-mm-what-is-strategy}} to be able to have real-time risk insights at all levels
of your organization then we can separate the specific functional requirements of that very large use case from
all the \glspl{nfr} that would come into place around your data management and governance practices, the state of
your technology landscape, the culture of your organization and so forth.
In this example, the use case \textit{Real-time risk management}\,---\,as the top-level strategic use case\,---\,would
be broken down into various \textit{sub-use cases} resulting in a so-called \textit{\enquote{\glsfirst{uct}}}.
Each node in that tree represents some functionality\,---\,a capability so you will.
How does this relate to the \gls{ekgmm}? Well, per use case in that tree structure you can determine how realistic
it is to actually be able to implement that use case in relation to the level of maturity of the organization as a
whole and more specifically in relation to any of the capabilities that are being covered in this document.

\paragraph{Use Cases Replace Silos}

Another major point to make about the \glsxtrfull{uct} is that it is the backbone component-structure
of your \gls{ekg}.
In a way you could see them as the \gls{ekg}'s equivalent of your current siloed application- and data-landscape.
Rather than building a new silo everytime for every new use case, you would "build" new use cases in the \gls{ekg}.
Use cases are \textit{logical silos} in that sense.
In a data-centric \gls{ekg} world, there are no more independent applications that are run as separate systems.
Just this particular change alone has major implications on how things are organized.
The \gls{ekgmm} covers these\,---\,and many other\,---\,implications.

\paragraph{Capabilities as Use Cases}

The \gls{ekgmm} is about the generic capabilities and how they may evolve over time and how some of them are
required to be available at a certain level of maturity.
That being said, almost every capability that we describe in this document is in itself also a use case.
Many if not most of these capabilities will more and more depend on high quality detailed and "holistic" data.

\paragraph{Other Artifacts}

\begin{description}[font=\bfseries,leftmargin=0cm]
    \item[\gls{ekgmanifesto}] The \gls{ekgmanifesto} is a combination of long term vision statements\,---\,the
                              "Why"\,---\, combined with the underlying \iindex{principles}
                              for \gls{ekg}\,---\,the "How".
    \item[\gls{ekgmethod}]    The assumption is that each organization in a data-intensive industry will end up
                              owning their own \gls{ekg}\,---\,possibly shared across organizations\,---\,and therefore
                              requires oversight by a group of people\,---\,however organized\,---\,that we call
                              the \gls{ekg:coe}.

                              \gls{ekgmethod} describes all the practices of the \gls{ekg:coe},
                              from inception to production.
    \item[EKG/Catalog]        The EKG/Catalog describes all the common use cases (and some of their "sub-use cases")
                              for \gls{ekg}.
                              The goal is to also publish these use cases on the ekgf.org website under the so-called
                              "use case portal" and make them available for download as modules, basically the same
                              kind of ecosystem of reusable components as any other successful technology-stack has.
                              This is comparable to what \href{https://central.sonatype.org/}{"Maven Central"} is for
                              the Java world or what \href{https://docs.npmjs.com/about-npm}{"npmjs.com"} is for the
                              JavaScript world.
\end{description}

\paragraph{Use Case Central}

One of the primary goals that the Foundation is working towards is\'---\'as briefly mentioned above\'---\'to publish
use cases on the \href{https://ekgf.org}{ekgf.org} website, initially just as a searchable catalog of use cases,
just text basically, but soon after as a graph, as part of an \gls{ekg}.
Some members of the Foundation\,---\,in particular \agnos and \eccenca\,---\,already have \gls{ekg} infrastructure
running in production that allows them to run "no-code" or "\gls{low-code}" use cases as full transactional applications.
The idea is to standardize this, to publish use cases as "reusable components", with all their dependencies, ready
for immediate deployment, just like one would do in the Java/JVM world or in the JavaScript world or with any
other successful technology stack.

\paragraph{Portals}

The Foundation is still in its start-up phase.
Plans are to invest in the development of a more advanced website that consists of various "portals" focusing
on serving a specific need for a specific audience:

\begin{description}[font=\bfseries,nosep,leftmargin=!,labelwidth=\widthof{\bfseries Member Directory \& Services}]
    \item[Use Cases]
        \begin{itemize}
            \item Short term:
            \begin{itemize}
                \item Audience: Business Executive, Vendor
                    \begin{itemize}
                        \item Understand the business problems that can, and have, been addressed
                        \item Use a framework for structuring projects
                        \item Get directed to relevant reuse points in other portals (e.g. for ontologies)
                    \end{itemize}
            \end{itemize}
            \item Long term:
            \begin{itemize}
                \item Audience: Executive, Vendor, Knowledge Graph Engineer
                    \begin{itemize}
                        \item Identify, select, download \& provision production-ready use cases
                    \end{itemize}
            \end{itemize}
        \end{itemize}
    \item[Best Practices]
        \begin{itemize}
            \item Audience: Project Lead, Consultant, any role in the \gls{ekg:coe}
            \begin{itemize}
                \item Reduce the risk and cost associated with new \gls{ekg} projects
            \end{itemize}
            \item Audience: Vendor
            \begin{itemize}
                \item understand how to develop, sell and deploy their products (and services)
                  to maximize applicability and success
            \end{itemize}
        \end{itemize}
    \item[Software]
        \begin{itemize}
            \item Audience: Architect, Vendor, Academic
                \begin{itemize}
                    \item Access reusable software curated for EKG purposes (may be hosted externally)
                \end{itemize}
            \item Audience: Vendor, Academic
                \begin{itemize}
                    \item Position existing products or components to increase uptake
                \end{itemize}
        \end{itemize}
    \item[Ontologies]
        \begin{itemize}
            \item Audience: Modeler
                \begin{itemize}
                    \item Assess ontologies curated as reusable for \glspl{ekg:use-case}
                    \item Apply tooling to automate ontology development and measurement
                \end{itemize}
            \item Audience: Academic
                \begin{itemize}
                    \item Access a set of ontologies for research, analysis, or extension
                \end{itemize}
        \end{itemize}
    \item[Datasets]
        \begin{itemize}
            \item Audience: Analyst
                \begin{itemize}
                    \item Access reusable RDF resources curated for EKG purposes (may be hosted externally)
                \end{itemize}
            \item Audience: Analyst, Vendor
                \begin{itemize}
                    \item Make use of data resources for demonstration and experimentation
                \end{itemize}
        \end{itemize}
    \item[Member Directory \& Services]
        \begin{itemize}
            \item Connect with people and parties with the knowledge to help
                  (see also appendix \nameref{appendix:ekgf-corporate-members})
            \item Promote your knowledge and skills
        \end{itemize}
\end{description}

\paragraph{Persona Stories}

The Foundation recognizes different types of "personas" that it needs to serve:

\begin{description}[font=\bfseries,leftmargin=!,labelwidth=\widthof{\bfseries Technical Architect}]
    \item[\texttt{<as a>}]      \texttt{<I want to>}
    \item[Business Executive]   Realize the benefits and minimize the risks of \gls{ekg} by having access to
                                proven methods, best practices and a community of experts.
    \item[Vendor]               Make it easy for organizations to understand and successfully adopt my product across
                                multiple industry sectors.
    \item[Modeler]              Have access to a proven and consistent set of deployable use cases,
                                models \& ontologies that can be used with my organization’s \gls{ekg}.
    \item[Technical Architect]  Have access to components and interfaces with supporting technology architectures
                                that I can assemble and deploy within my environment.
    \item[Consultant]           Have access to\,---\,and contribute to\,---\,\gls{ekg}-related best practices
                                (e.g. \glsfmtshort{ekgmethod} \& \glsfmtshort{ekgmm}) as well as a community of
                                potential customers and skilled associates.
    \item[Academic]             The chance to make a meaningful and recognized contribution that builds upon
                                frameworks to address pressing business needs.
    \item[\glsxtrshort{ekg} Engineer] be informed about best practices, role descriptions, education and certifications.
\end{description}
