%
% C.4.1 Data-management Operating Model -- Summary
%
The people, processes, capabilities, and tools that define the role of \iindex{data management} in delivering
value to an organization’s customers.\footnote{%
    There is also a capability called ``\nameref{sec:ekg-mm-business-operating-model}'' in the
    Business Pillar that has a different meaning, see \secref{sec:ekg-mm-business-operating-model}.
}

The \gls{operating-model} can help stakeholders understand the complexity of the data manufacturing process and how
components relate to each other.
It specifies the roles and responsibilities of the stakeholders involved in the data management program.
It provides a framework and policies to help align governance concepts with operating requirements and
organizational culture.
An effective \gls{operating-model} can both describe the way the organization does business today (“as is”) and
communicate a vision of how an operation will work in the future ("to be").
\improve{JAG>Make the link to the corresponding capabilities in the business pillar here}

