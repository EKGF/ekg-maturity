%
% D.2.13 Knowledge Graph Virtualization -- Summary
%
(Work in progress)

Virtualization provides knowledge graphs the capability to source data not designed for linked data concepts
like relational data sources.

It is one of the many techniques that are available in the arsenal of \gls{ekg} engineers to get data from a given
source to be "connected" to all other data in the \gls{ekg}.

Generally these are the options at a high level:

\begin{itemize}
    \item The \gls{ekg} itself is the authoritative source of a given dataset.
    \begin{itemize}
        \item The dataset is complex
        \begin{itemize}
            \item Use a triple store (no virtualization needed)
        \end{itemize}
        \item The dataset has a very large volume (> hundreds of billions of facts)
        \begin{itemize}
            \item Use a specialized database type and apply virtualization.
        \end{itemize}
    \end{itemize}
    \item The \gls{ekg} is not the authoritative source of a given dataset.
    \begin{itemize}
        \item The data is relatively clean and there is either a real-time requirement or a massive volume of data.
        \begin{itemize}
            \item Virtualization\,---\,as provided by various vendors like \stardog and \graphdb\,---\,may be a
            good option because it gives an easy path from \iindex{SPARQL} to whichever backend database,
            translating SPARQL "on-the-fly" to \iindex{SQL} or other query languages.
        \end{itemize}
        \item None of the above:
        \begin{itemize}
            \item Process all data from the given source in one batch \gls{etl} pipeline,
            store it in a triple store controlled by the \gls{ekg:platform} and serve it from there.
            No virtualization needed, therefore more opportunities to enhance quality in the pipeline (since
            virtualization often comes at the price of being less flexible in terms of available options to enhance
            data quality).
        \end{itemize}
    \end{itemize}
\end{itemize}

By the use of RDF mapping languages, relational data is mapped to semantic knowledge graphs leveraging tools that
implement RML\index{tool!RML}, R2RML\index{tool!R2RML}, OBDA\index{tool!R2RML}, YARRRML\index{tool!YARRRML},
D2RQ\index{tool!D2RQ} and many vendor specifications.

Lossless conversion to an RDF data model can be achieved by defining terms to relational schema
definitions of columns, type, and tables in mapping definitions.
Depending on the tool used to perform the virtualization,
data can be materialized into RDF files and/or ad-hoc on demand.
A knowledge graph virtualization tool that operates on relational data will execute a SQL query and
convert the results to RDF.
If the results of the virtualization is materialized, the query executed and the results stored as RDF files.
If the virtualization is performed ad-hoc and not stored, the query will be excused to the source system each
time the query is performed.
If there is a large resultset expected from virtualization,
it usually more performant to materialize the data and load the RDF into a knowledge graph.
If a small detail based result set that  is susceptible to change is expected from the virtualization,
then ad-hoc is desirable.
Virtualization also can be done on heterogeneous data outside of the knowledge graph,
before an RDF mapping language is applied.
Data virtualization can be used to allow multiple data sources to be combined and queried as a single data source.
External data virtualization can be used to get past limitations for example with R2RML which defines a
single data source.
Many relational data and knowledge graph virtualization tools also support Spark.
Mapping can be very prescriptive to an ontology from a source system, or less.
The less prescriptive approach is to allow data simply to be converted to an RDF model,
later to be modified by SPARQL queries.
The more prescriptive approach requires more diligence in model mapping.

