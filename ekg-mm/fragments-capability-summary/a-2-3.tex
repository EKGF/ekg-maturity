%
% A.2.3 Change Management -- Summary
%
\somequote{%
    Change management\,---\,The business process that coordinates and monitors all changes to the business processes
    and applications operated by the business, as well as to its internal equipment, resources, operating systems,
    and procedures.
    The change management discipline is carried out in a way that minimizes the risk of problems that will affect
    the operating environment and service delivery to the users.%
}{ASCM}{https://ASCM.org}

Change management guides the business for change required to react to problems or to proactively plan changes to
mitigate risk, obtain a good result for divesting and/or acquiring new areas of business, product changes for
shifts in customer needs, improve processes for efficiency and cost savings, stay ahead of competitors,
take advantage of new technologies, etc.

Change management is the discipline to guide realization and sustainment of business strategy intent.
Impact of changes in strategy should be modeled and serve as a basis for executing the change.
Change can be managed for any aspect of the business to include: geographic and facility footprint, market expansion,
product development, business organization and staffing, new sales channels, process improvement, technology change,
compliance to changes in regulations, and more.