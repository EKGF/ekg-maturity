%
% D.2.12 Knowledge Graph Federation -- Summary
%
\ifoptionfinal{
\TODO[inline]{Create summary for \thesection}
}{
Knowledge Graphs can be built from combined data sources from a queryable service layer and API.
Enterprises inherently have this need as their data footprints are vast,
and usually do not store their data in a single source instance.
Federation can be applied to the knowledge graph by leveraging linked data facilities.
Various query protocols such as the SPARQL query language has a SERVICE facility to combine remote endpoints,
and GraphQL can combine APIs.
Regardless of the query protocol used, federation provides a link that abstracts the underlying system in a
way that seamlessly ties sources together.
A knowledge graph provides capabilities to federate queries on the backend.
If the knowledge graph is discoverable via a service or endpoint,
a client can also federate remote knowledge graphs not only in server backends,
but even in browser based faceted implementations.

Combining sources with federation is usually done in a select/read goal, but not create/update/delete.
No assumptions should be made when queries are performed via federation that they are done in a transacted manner.
An extension of federation is virtualization, which will be discussed in further section.
Virtualization provides linked data transformation from a source not designed for knowledge graph
in a materialized or ad-hoc manner using a mapping facility.
}