%
% D.2.12 Knowledge Graph Federation -- Summary
%
An \glslocalreset{ekg}\gls{ekg} can be seen by end-users as "one thing",
a "holistic" collection of all connected data, similar to the web.

However, as also specified in principle \Nameref{sec:ekg-principle-distributed}, the \gls{ekg} is distributed by
nature, assuming that it is not realistic in very large organizations or even eco-systems to have only one physical
implementation of a fully centralized \gls{ekg}.
That means that different parts of the \gls{ekg} are served by different installations or
deployments owned and controlled by different parts of the organization or even other organizations.

Each deployment can be configured to connect to any number of "backend" data sources
(or destinations/sinks), some of which can be real triple stores (aka RDF Databases, quad stores or semantic graph databases)
and some of which can be relational databases, key/value stores or any other database technologies.
Or even just services with \glspl{api} that are used to get or store data.

At higher levels of \gls{ekg} platform maturity, all access to the \gls{ekg} is provided via this service layer\,---\,generally
called the \gls{ekg:platform}\,---\,that takes care of federation of any request to any backend data source using any technology.

All that technology is however hidden for the user.
In that sense an \gls{ekg:platform} is just a \gls{soa} layer\footnote{See \url{https://en.wikipedia.org/wiki/Service-oriented_architecture}}.
However, it is a fully model-driven \gls{soa} layer and one that works with all other known deployments of the \gls{ekg:platform}.

The federation facilities provided by the \gls{ekg:platform} are leveraging the principles of the linked data standard\cite{linked-data}
as originally defined by Sir Tim Berners-Lee\index{tim berners-lee} in 2005.
However, the original linked data standard does not provide many of the facilities that are required for mission-critical
enterprise use cases such as model-driven entitlement enforcement, automatic selection of the right version of the truth
for the given context and so forth.

Various query protocols have built-in federation facilties, for instance the SPARQL\index{sparql} query language has a
facility\,---\,via the \lstinline|SERVICE| keyword\,---\,to federate a query across multiple remote endpoints and
the GraphQL\index{graphql} query language can combine \glspl{api} of multiple remote systems.
Regardless of the query protocol used, federation provides a link that abstracts the underlying system in a
way that seamlessly ties sources together.

An \gls{ekg:platform} provides capabilities to federate queries on the backend.

The \gls{ekg:platform} is discoverable by other services, \glspl{ekg} or browsers
a client can also federate remote knowledge graphs not only in server backends,
but even in browser based faceted implementations.

An extension of federation is \iindex{virtualization}, which will be discussed in further section.
Virtualization provides linked data transformation from a source not designed for knowledge graph
in a materialized or ad-hoc manner using a mapping facility.