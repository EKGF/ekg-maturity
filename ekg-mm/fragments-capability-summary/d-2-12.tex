%
% D.2.12 Knowledge Graph Federation -- Summary
%
\glslocalreset{ekg}\Glspl{ekg} can be built from combined data sources from a queryable service layer and \gls{api}.
Enterprises inherently have this need as their data footprints are vast,
and usually do not store their data in a single source instance.

At higher levels of \gls{ekg} platform maturity, all access to the \gls{ekg} is provided by a service layer\,---\,generally
called the \gls{ekg:platform}\,---\,that takes care of federation of any request to any backend data source using any technology.

The federation facilities provided by the \gls{ekg:platform} are leveraging the principles of the linked data standard\cite{linked-data}
as originally defined by Sir Tim Berners-Lee\index{tim berners-lee} in 2005.
However, the original linked data standard does not provide many of the facilities that are required for mission-critical
enterprise use cases such as model-driven entitlement enforcement, automatic selection of the right version of the truth
for the given context and so forth.

Various query protocols have built-in federation facilties, for instance the SPARQL\index{sparql} query language has a
facility\,---\,via the \lstinline|SERVICE| keyword\,---\,to federate a query across multiple remote endpoints and
the GraphQL\index{graphql} query language can combine \glspl{api} of multiple remote systems.
Regardless of the query protocol used, federation provides a link that abstracts the underlying system in a
way that seamlessly ties sources together.

An \gls{ekg:platform} provides capabilities to federate queries on the backend.

The \gls{ekg:platform} is discoverable by other services, \glspl{ekg} or browsers
a client can also federate remote knowledge graphs not only in server backends,
but even in browser based faceted implementations.

Combining sources with federation is usually done in a select/read goal, but not create/update/delete.
No assumptions should be made when queries are performed via federation that they are done in a transacted manner.
An extension of federation is \iindex{virtualization}, which will be discussed in further section.
Virtualization provides linked data transformation from a source not designed for knowledge graph
in a materialized or ad-hoc manner using a mapping facility.