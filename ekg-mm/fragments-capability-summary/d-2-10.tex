%
% D.2.10 Identifier Resolution -- Summary
%
Identifier resolution is a process of combining multiple identifiers across devices, spreadsheets, repositories, and
platforms into a cohesive profile.
The process includes searching across disparate datasets and analyzing content to find (and resolve) matches based
on available data records and attributes.
Identifier resolution is complicated by distinctions in both structure and meaning because various information systems
can vary in quality, completeness, format, and nomenclature.

Scenario / Use Case -
Use a URI for identification without considering impacts of impacts to unique identity by using an named identifier.

If you will only have a single gremlin vehicle, ever for-sale then you could go with a URI for which there isn’t a
concern for the uniqueness as it’s not possible another will come along.
Or, it could be managed manually by checking for existence of the name of the vehicle prior and incrementing.\\
First gremlin - \\
{\footnotesize\url{<https://ekgf.org/cars/for-sale#gremlin>}}\\
Second gremlin - \\
{\footnotesize\url{<https://ekgf.org/cars/for-sale#gremlin2>}} \\
Pro(s): Easy to implement \\
Con(s): Chance of duplicates and collisions high \\

Scenario / Use Case -
Any and all attributes can be used to make the URI somewhat unique for identification and can be used for the URI

URL Encoded name concatenated to the URI .

    {\footnotesize\url{<https://ekgf.org/cars/for-sale#gremlinUsed%20Vintage%20Gremlin%20%237>} \\ \url{<http://purl.org/goodrelations/v1#name>} "Used Vintage Gremlin \#7”}

Pro(s): Easy to implement \\
Con(s): Chance of duplicates and collisions moderate, uses more details from the attributes.
Discouraged, considered an anti-pattern for designing ontologies.

Scenario / Use Case -
Use a unique identifier from an external service in the URI .

    {\footnotesize\url{<https://ekgf.org/cars/for-sale#gremlin>}}
Could have a Vehicle Identification Number added to the URI, the organization responsible for VINs guarantees uniqueness
    {\footnotesize\url{<https://ekgf.org/cars/for-sale#gremlinA3A465H399999>}}


Pro(s): Easy to implement \\
Con(s): Chance of duplicates and collisions moderate, uses more details from the attributes


Scenario / Use Case -
We can promote blank nodes to more explicit local resources in the form urn: and/or \_id:

Perhaps we adopt urn:org:method:id, method could be base64, sha256, or others which would indicate the type.
could also look like:

create a blank node for VIN of the vehicle
    {\footnotesize \_:A3A465H399999}

    {\footnotesize\url{<urn:ekgf:vin:_:A3A465H399999>}}

Pro(s): Gives a more normalized approach for blank nodes to be used with other systems.
Or not-well defined canonical scheme. \\
Con(s): Debatable utility if you have a namespace already, if possible use a prefix instead.

Scenario / Use Case -
We can use an iso8601 date timestamp for a unique identifier for the URI .

iso8601 generated date/time \\
2021-11-22T03:28:40+0000 \\
URL Encoded \\
{\url {2021-11-22T03%3A28%3A40%2B0000}} \\
    {\footnotesize\url{<https://ekgf.org/cars/for-sale#gremlin2021-11-22T03%3A28%3A40%2B0000>}}

Pro(s): Simple to implement with a timestamp, quite reasonable \\
Con(s): There is still a very small chance of duplicates.

Scenario / Use Case -
We can use a UUID for a unique identifier for the URI .

Generated the following UUID :
    {\footnotesize e37b55e6-7447-4427-bcd8-9dae64750a1d }, concat to the URI .

    {\footnotesize\url{<https://ekgf.org/cars/for-sale#gremline37b55e6-7447-4427-bcd8-9dae64750a1d>}}

Pro(s): Simple to implement with a uuid, quite reasonable \\
Con(s): Not repeatable, will need to store as each generated identifier is unique

Scenario / Use Case -
We can compute SHA 256 for which we can use as a basis of identity from attributes of an object.
SHA hash values have the advantage of being reproducible vs UUIDs/iso8601 dates.

In this example we compute SHA 256 from the name “Used Vintage Gremlin \#7”

    {\footnotesize\url{<https://ekgf.org/cars/for-sale#gremlin>} \\ \url{<http://purl.org/goodrelations/v1#name>} "Used Vintage Gremlin \#7”}

URIs with a SHA 256 will need an ALPHA character leading it as an URI cannot lead with a NUMERIC

Attributes of an object can be used to make the object URI unique  by using a SHA 256. \\
    {\footnotesize\url{<https://ekgf.org/cars/for-sale#gremline1752b9574895a962d4bfa554b98296d3acc06fdd5871f75f1c902ed3d5c13c3>}  \\ \url{<http://purl.org/goodrelations/v1#name>} "Used Vintage Gremlin \#7”}

SHA256 can be regenerated in a one way manner vs UUID which cannot.
UUIDs may need to  be managed once generated.

Pro(s): Simple to implement like a uuid as SPARQL and various languages have support, quite reasonable, also
repeatable one way \\
Con(s): Some may find the length of the identifier an eyesore, that said the length is completely reasonable and safe
for URIs if a bit challenging to type


Scenario / Use Case -
Use an AES (or other methods such as ChaCha ) encrypted querystring for additional security.

It may be desirable to encrypt a query string / part of a URI to prevent tampering and obfuscating parameters.
Note that this should not be relied on for access control, so username and password are perhaps not good candidates
for encryption- instead http headers should be used per OWASP .

If we take the VIN A3A465H399999 from the gremlin and create a SHA256 from the VIN .

SHA 256 of A3A465H399999 \\

{\footnotesize 27d68368497b314c6bae0fa8ae61fd2c78406c01f3202c0bd73a63f1065e668b }

We then could encrypt the SHA 256 using a key !!!!!!!ekgfknowledgegraph!!!!!! mode ECB .
Base64 encoded results are \\
{\footnotesize agMNw71rsJcJOrMdTIVKEQqfwqmU56HtzZozR2DYjrvsyr9+bJXMfVZk2x0hqwz55njtWLMmzcwPqYDz6ovJW1tODpnWD60srd9046E/FWE=}
Which could then be incorporated as part of the URI \\
    {\scriptsize\url{<https://ekgf.org/cars/for-sale#gremlinagMNw71rsJcJOrMdTIVKEQqfwqmU56HtzZozR2DYjrvsyr9+bJXMfVZk2x0hqwz55njtWLMmzcwPqYDz6ovJW1tODpnWD60srd9046E/FWE=>}} \\
Decrypted Base64 would yield \\
{\scriptsize MjdkNjgzNjg0OTdiMzE0YzZiYWUwZmE4YWU2MWZkMmM3ODQwNmMwMWYzMjAyYzBiZDczYTYzZjEwNjVlNjY4Yg== } \\
Converted to plain text from Base 64 would yield the SHA 256 of the VIN \\
27d68368497b314c6bae0fa8ae61fd2c78406c01f3202c0bd73a63f1065e668b \\
Querystring parameters besides an identifier, can also be handled by AES encryption

Pro(s): Provides extra encryption for IRI/URI \\
Con(s): Some may find the length of the identifier an eyesore, and the AES encryption excessive and perhaps if such \\
encryption is required the information shouldn’t be in an URI

Scenario / Use Case -
Use a Decentralized Identifier / DID

This could also include distributed identifier services from the w3c, the below is non registered but taken
from a sha256 of a VIN,then encrypted using AES256 with key !!!!!!!ekgfknowledgegraph!!!!!!! using ECB mode from the
previous encryption example.
The format of a DID is :

did:method:key \\
{\footnotesize did:ekgf:agMNw71rsJcJOrMdTIVKEQqfwqmU56HtzZozR2DYjrvsyr9+bJXMfVZk2x0hqwz55njtWLMmzcwPqYDz6ovJW1tODpnWD60srd9046E/FWE= }

Pro(s): Leverages an external organization to create the identifier, emerging as a standard. \\
Con(s): Leverages an external organization to create the identifier, specs and software may need to be written for
internal usage.
