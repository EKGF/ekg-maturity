\bookmarksetup{bold=true, color=blue}
\setcounter{part}{1}
\part{Data Pillar}\label{pt:ekgmm-b} % B Data

The Data Pillar has the following components:

\begin{itemize}[leftmargin=.5in]
  \item [\ref{ch:ekg-mm-b-1}] \nameref{ch:ekg-mm-b-1} % B.1 Data Strategy
  \item [\ref{ch:ekg-mm-b-2}] \nameref{ch:ekg-mm-b-2} % B.2 Data Architecture
  \item [\ref{ch:ekg-mm-b-3}] \nameref{ch:ekg-mm-b-3} % B.3 Data Quality
  \item [\ref{ch:ekg-mm-b-4}] \nameref{ch:ekg-mm-b-4} % B.4 Data Governance
\end{itemize}

%\paragraph{Levels}
%
%\begin{description}[nosep,font=\bfseries]
%
%    \item [1. \ekgmmLevelOneLabel]
%    Core data management capabilities (\gls{operating-model}, inventory, data architecture,
%    \hyperref[sec:ekg-mm-business-vocabularies]{vocabularies, business glossary or terminology},
%    pipeline management, etc.) are being performed.
%    Specific use cases are being implemented with specialist teams for the pilot initiative.
%
%    \item [2. \ekgmmLevelTwoLabel]
%    Critical data elements are prioritized in the ontology.
%    Approach to identity and meaning resolution is established.
%    Use case trees are defined and modeled to capture shared data relationships.
%    The knowledge graph is becoming the central point for integration.
%
%    \item [3. \ekgmmLevelThreeLabel]
%    Inventory is embedded into the \gls{ekg} and linked to governance.
%    Data is expressed as formal ontologies, onboarded into the \gls{ekg} and searchable.
%    Data flows are defined and modeled.
%    The \gls{ekg} is the authoritative source for data.
%
%\end{description}

\chapter{Data Strategy}\label{ch:ekg-mm-b-1}\label{ch:ekg-mm-data-strategy} % B.1 Data Strategy

The \nameref{ch:ekg-mm-b-1} component has the following capabilities:

\begin{itemize}[leftmargin=.5in]
    \item [\ref{sec:ekg-mm-b-1-1}] \nameref{sec:ekg-mm-b-1-1}\,---\,\gls{ekg} as the semantic \index{data fabric} for the organization
    \item [\ref{sec:ekg-mm-b-1-2}] \nameref{sec:ekg-mm-b-1-2}\,---\,Business rationale, justification, and \glsxtrshort{roi} of the \gls{ekg}
\end{itemize}

\section{Goals \& Objectives}\label{sec:ekgmm-b-1-1} % B.1.1 Goals & Objectives

\gls{ekg} as the semantic \iindex{data fabric} for the organization.

We welcome your input here

%
% B.1.2 Business Case
%
\ekgmmCapability{b-1-2}{ekg-business-case}{Business Case}

We welcome your input here.


\chapter{Data Architecture}\label{ch:ekg-mm-b-2}\label{ch:ekg-mm-data-architecture} % B.2 Data Architecture

The \nameref{ch:ekg-mm-b-2} component has the following capabilities:

\begin{itemize}[leftmargin=.5in]
  \ekgmmCapabilityItem{b-2-1}
  \ekgmmCapabilityItem{b-2-2}
  \ekgmmCapabilityItem{b-2-3}
  \ekgmmCapabilityItem{b-2-4}
\end{itemize}


%
% B.2.1 Ontologies
%
\ekgmmCapability{b-2-1}{ontologies-and-data-models}{Ontologies}

\ekgmmCapabilitySectionContributionToEnterprise

An ontology is about:

\begin{itemize}
    \item explicit meanings and relationships; the terms used are less important
    \item a combination of definitions in both text and logic
\end{itemize}

An ontology can be the basis of, but is broader than:
\begin{itemize}
    \item a taxonomy
    \item a vocabulary
    \item a data or object model
    \item a conceptual model
    \item a specific serialization format
\end{itemize}

Ontologies can be expressed at different levels of sophistication, with different scopes,
and in a combination of languages.
The basic structures include:

\begin{itemize}[leftmargin=.8in,font=\itshape]
    \item [individual] a representation of a business object or item which is the subject of information to be managed.
                       An individual usually has a unique identity.
                       For example \lstinline|Person X| or \lstinline|Shipment Y|.
                       Many such individuals might represent the same real world object.
    \item [data value] strings, numbers, dates which represent the data.
    \item [property]   a type of data point that may be associated with individuals.
                       An individual, a property and a value\,---\,which may be a data value or another
                       individual\,---\, form a triple.\newline
                       For example person \lstinline|X hasBirthDate D| or person \lstinline|X hasMother Y|.
                       Triples whose value is another individual form relationships.
                       Properties may have generalizations, for example hasMother is a subProperty of hasParent.
    \item [class]      a category applied to individuals, that determines what you can do with them,
                       the properties you can expect to see, and the rules that might apply;
                       an individual may be a member of many classes associated;
                       classes may have generalizations.
                       Note that, unlike more traditional approaches, properties are independent of classes though
                       they may be used to infer class membership.
                       For example, given the triple \lstinline|X hasMother Y| you may be able to infer that both
                       \lstinline|X| and \lstinline|Y| are members of the class \lstinline|Person|,
                       or at least \lstinline|Animal|.
    \item [ontology]   grouping of the above for management and identification purposes.
\end{itemize}

\ekgmmCapabilityContributionToEKG{b-2-1}

\ekgmmCapabilityContributionToEnterprise{b-2-1}

\ekgmmCapabilitySectionDimensions

\begin{itemize}
    \item How much of the enterprise data is covered by ontologies?
    \item How well is ontology coverage mapped to business need?
    \item To what extent are concepts independent of but mapped to terminology/vocabulary?
    \item Level of sophistication of textual and logic definitions
    \item Level of tooling is available and used
    \item Level of training and trained people
    \item Level of process (including change management), guidelines and standards
    \item Level of modularity and reuse - internal and external
    \item Extent of examples and tests
    \item Extent of traceability with different logical and physical data models
\end{itemize}

\ekgmmCapabilitySectionLevels

The following criteria for each level are abbreviated: each item is shorthand for:

\begin{itemize}
    \item documented process
    \item trained participants
    \item implemented process and/or technology
    \item monitoring and improvement
\end{itemize}

\ekgmmCapabilitySectionLevelsOneFive

\ekgmmscoringlevelOne

\begin{itemize}
    \item Minimal ontologies which could be as simple as a list of classes and properties used in graphs
    \item Basic metadata (definition, label) for each class and property
    \item Each individual (in data) has at least one explicit class
    \item Ontology coverage for each use case in scope of the project;
          project includes minimal number of ontologies and classes not justified by a use case
    \item Definitions catalogued and under change management

\end{itemize}

\ekgmmscoringlevelTwo

\begin{itemize}
    \item Ontologies expressed in a standard ontology language (could be as simple as \gls{ekg:rdf-schema})
    \item Common (shared or mapped) concepts across \gls{ekg} projects
    \item Ability to see ontology usage by use cases, vocabularies and datasets
    \item Namespace scheme established and used for new ontologies in the \gls{ekg}
    \item Ontology guidelines in place and implemented, including common metadata
    \item Documented approach for external ontologies, including selection and adaptation
    \item Annotated example files for documentation and training
    \item Test files based on use cases covering all used ontology elements
    \item Ontology change management includes impact analysis and stakeholder approval
    \item Tooling for ontology diagrams and documentation
    \item Automated basic checking of ontology syntax
    \item Access to at least one trained \gls{ekg:ontologist}
\end{itemize}

\ekgmmscoringlevelThree

\begin{itemize}
    \item Modeling of required data and constraints by use case, including for stored and communicated data
    \item Automated validation of ontologies (for guideline compliance, and for logical consistency),
          with results as triples
    \item Automated testing and validation of test data with ontologies (per use case)
    \item Separation of concerns to support enterprise management such as bi-temporality, transactions and events
    \item Automated transformation of ontologies to use common serialization and metadata
    \item Automated checking of ontologies against different profiles (e.g. OWL-RL) to check for technology support
    \item Automated checking of ontologies against different best practices
    \item Ontology source changes linked to automated KGOps for testing and deployment
    \item Impact analysis identifies ontology breaking changes which require fixes to existing EKG data
    \item EKG-wide ontology browsing and searching
    \item Follow-your-nose UI starting from any ontology element URI
    \item Follow-your-nose API starting from any ontology element URI
    \item Trained ontologist available to each project (possibly via \gls{ekg:coe})
\end{itemize}

\ekgmmscoringlevelFour

\begin{itemize}
    \item Separation of ontologies from vocabularies, with multiple vocabularies for different communities
          mapped to the same concepts
    \item Ontology architecture management process, including use of patterns and modularity
    \item Generation of logic into business language
    \item Automated fixes to existing EKG data in response to ontology breaking changes
    \item Basic ontology metrics and reporting, including usage in data
    \item Generation of ontologies/shapes for external interchange
\end{itemize}

\ekgmmscoringlevelFour

\begin{itemize}
    \item Sophisticated ontology metrics and reporting, including trends
    \item Matching and differencing of ontologies from different sources
    \item Automated matching of ontologies with vocabularies
    \item Generation of validation code for external interchange
    \item Wizard for developing ontologies from business questions
    \item Inducing of ontologies from instance data
\end{itemize}

%
% B.2.2 Inventory Management
%
\ekgmmCapability{b-2-2}{inventory-management}{Inventory Management}

\ekgmmCapabilitySectionContributionToEnterprise

Linking data inventory to the knowledge graph (and business concepts) ensures the precision of meaning at the most
granular level.
Data in the knowledge graph is traceable to all application usage allowing users to find data of interest through
assisted and contextual search and navigation.
The unique ability of the knowledge graph to connect data with metadata enables users to perform flexible queries
in ways that were not previously possible.

\ekgmmCapabilitySectionDimensions

\begin{core-questions}

  \item [\thesection.1] Do one or more data inventories exist
  \item [\thesection.2] Is the inventory based on defined standards (for both meaning and format)
  \item [\thesection.3] Is defined and in-scope data (both breadth and depth) covered in the inventories
  \item [\thesection.4] Is the data inventory linked to \glspl{sor} and authorized data distribution points
  \item [\thesection.5] Is the inventory linked to the business meaning of the data and expressed using standards
  \item [\thesection.6] Is the creation and maintenance of the data inventory mandated by policy and incorporated
                        into the data strategy
  \item [\thesection.7] Is the quality of the content in the inventory measured, reported to involved stakeholders,
                        and used for process enhancement

\end{core-questions}

\ekgmmCapabilitySectionLevelsOneFive

(concepts are inherited as levels progress)

\ekgmmscoringlevelOne

\begin{scoring}

  \item Sources, data sets, and metadata are onboarded and expressed as formal ontologies
  \item Authoritative (upstream) data sources and (downstream) consumers are documented and verified by users,
        data, and technology
  \item The inventory of applications is defined and selected for graph applications
  \item Requirements and dependencies for each outbound data flow are documented and verified (implementation in the
        graph is not a requirement)
  \item Business glossaries for in-scope use cases are defined and verified in the graph (including a list of
        data sources and datasets)
  \item Policy implemented mandating inventory maintenance and only authorizing the use of data that has been logged
        into the inventory

\end{scoring}

\ekgmmscoringlevelTwo

\begin{scoring}

  \item \glspl{uct} are defined, standardized, and implemented
  \item All upstream data sources are linked to the authorized systems of record and distribution points
  \item Policy mandating the use of \glspl{sor} and documentation of data flow is implemented
  \item Entitlements have been defined in the graph (governing access to sources of data in the inventory)
  \item Classifications (i.e. criticality, security, privacy) are aligned with the use case tree and captured in the
        knowledge graph
  \item Governance requirements (i.e. use cases, \iindex{accountability}, data sources, data flows, \glspl{sla})
        are modeled and registered into the knowledge graph

\end{scoring}

\ekgmmscoringlevelThree

\begin{scoring}

  \item Data inventory is centralized in the graph and linked to governance for defined use cases
  \item Ontologies and data models (including change history and transformations) are registered in the knowledge graph
  \item Entitlements are calculated within the inventory and enforced at the datapoint level
  \item Data Quality\index{data!quality} is automatically calculated (fine-grained with dynamic value resolution) within
        the inventory for each use case
  \item Data retention rules are registered in the graph and automatically enforced
  \item Full audit trail for all upstream and downstream data usage is registered in the graph
  \item Data elements, calculation methods, and \glspl{cde} are linked to individual regulatory requirements

\end{scoring}

\ekgmmscoringlevelFour

\begin{scoring}

  \item Connected inventory has been extended to include real-time (transactional) data
  \item Inventory is extended to external suppliers and third parties along the supply chain
  \item The inventory is fully integrated with machine learning to optimize data flow
  \item The “value of data” is calculated and classified within the organizational inventory

\end{scoring}


%
% B.2.3 Business Terminology (formerly called Business Glossary)
%
\ekgmmCapability{b-2-3}{business-terminology}{Business Terminology}
\index{business!glossary}
\index{business!terminology}

\ekgmmCapabilitySectionContributionToEnterprise

Parts of large organizations, and the others they communicate with, often have their own local business language.
Likewise, operational systems\,---\,including their data warehouses and content repositories\,---\,use
data models and predefined physical schemas that are designed to serve specific purposes.
The ability of the knowledge graph to link this terminology with its actual meaning defined via ontologies at
its most granular (atomic) level\index{granular data} eliminates confusion over meaning when data is validated
or transformed.
Standalone business glossaries provide useful input but the terms need to be reconciled to their
enterprise-level meaning via the \gls{ekg}.

\ekgmmCapabilitySectionDimensions

%\begin{core-questions}
%
%  \item [\thesection.1] Are there multiple (official) business glossaries for lines of business or functions
%  \item [\thesection.2] Are there internal standards for content, format, and use of the business glossaries
%  \item [\thesection.3] Is the business meaning of the data verified by accountable subject matter experts
%  \item [\thesection.4] If the organization maintains multiple business glossaries, how are they reconciled,
%                        verified, maintained and governed
%  \item [\thesection.5] Are the glossaries mapped to all expressions of the data in various systems
%                        (how is this maintained)
%  \item [\thesection.4] Are the business glossaries accessible and searchable from the corporate intranet
%
%\end{core-questions}

\ekgmmCapabilitySectionLevelsOneFive

Core Data Management Requirements: The organization has one (or many) officially designated glossaries,
front-to-back \gls{sme} review and verification, official glossaries are captured in an inventory,
there is a defined policy mandating that all glossaries have verified definitions,
all glossaries are maintained with a defined mechanism for \iindex{change management},
all in-scope domains are covered (and all content is included), mapped to processes and expressions
(lineage and transformation), \iindex{governance} (\iindex{ownership}, \iindex{transformation}, logic,
allowable values, \iindex{audit trail})

\ekgmmscoringlevelOne

\begin{scoring}

  \item Business terms for \glspl{ekg:use-case} are captured and mapped to an \iindex{ontology}
        (possibly as simple labels)

%  \item The organization has one (or many) officially designated glossaries
%  \item Front-to-back SME review and verification
%  \item Official glossaries are captured in an inventory
%  \item There is a defined policy mandating that all glossaries have verified definitions
%  \item All glossaries are maintained with a defined mechanism for change management
%  \item All in-scope domains are covered (and all content is included)
%  \item Mapped to processes and expressions (lineage and transformation)
%  \item Governance (ownership, transformation, logic, allowable values, audit trail)

\end{scoring}

\ekgmmscoringlevelTwo

\begin{scoring}

  \item Relevant terms (for \glspl{ekg:use-case}) are associated with the use case independently of
  their corresponding ontologies

%  \item All concepts for any KG use case are covered by the glossary
%  \item Process exists to add new terms (and updating) into the official glossaries

\end{scoring}

\ekgmmscoringlevelThree

\begin{scoring}

  \item Existing glossaries within the organization, within the scope of supported use cases,
        are mapped to ontologies and imported into the \gls{ekg}
  \item Terms are grouped into vocabularies for reuse in different communities
  \item Natural language used for definitions links to other terms used
  \item Ontology logic is presented to business as natural language
  \item Terms are searchable via the knowledge graph interface

%  \item Relevant terms (for KG use cases) are expressed as RDF and imported into the knowledge graph
%  \item Relevant terms are linked to their corresponding ontologies

\end{scoring}

\ekgmmscoringlevelFour

\begin{scoring}

  \item \gls{ekg} is the authoritative source for all terms, scoped by community, context and use case
  \item The governance processes for all new terms (and changes) are managed directly within the \gls{ekg}
  \item The results of ontology inferencing are presented in business natural language
  \item Data usage (per term) is accessible

%  \item All glossaries within the organization are expressed as RDF and imported into the knowledge graph
%  \item All glossaries are searchable via the knowledge graph interface
%  \item The governance process (including all metadata) for glossaries are on-boarded into the knowledge graph
%        [passive, linked to existing governance tool, read-only]

\end{scoring}

\ekgmmscoringlevelFive

\begin{scoring}

  \item Terminology is used to support \gls{nlp} of unstructured data for the \gls{ekg}

%  \item The \glsxtrlong{ekg:platform} is the authoritative source for all glossaries
%  \item All glossary terms are derived directly from the ontologies
%  \item The governance processes for all new terms (and changes) are managed directly within the \glsxtrshort{ekg}
%  \item Data usage (per glossary term) is monitored and reported

\end{scoring}


%
% B.2.4 Data Mapping (formerly known as Data Integration & Interoperability)
%
\ekgmmCapability{b-2-4}{data-mapping}{Data Mapping}
\index{data!mapping}

\ekgmmCapabilitySectionContributionToEnterprise

Data integration is challenging because data from multiple sources can have different file formats, data structures,
definitions, and contextual meanings.
Using the \gls{ekg} during data integration standardizes the meaning of data and makes the content understandable
to both humans and machines.
Embedding \iindex{referenceable meaning} into the data using machine-readable standards facilitates
automatic validation and assurance of data quality\index{data!quality}.
Registering data integration activities into the \gls{ekg} generates full data transparency across
linked processes.\index{lineage}
Finally, ontology-based metadata representations make it possible to embed business rules and accommodate
different values, identities, and definitions that existed at various times in the entity lifecycle.

\ekgmmCapabilitySectionDimensions

\begin{core-questions}

  \item [\thesection.1] Are the data integration activities, their systems, repositories, and connections
                        known and tracked
  \item [\thesection.2] Are data integration activities linked to data inventory, business glossaries, and data models
  \item [\thesection.3] Are all data integration input and output datasets documented, tracked, and governed
  \item [\thesection.4] Are there reusable standards and defined business rules for performing data integration
  \item [\thesection.5] Are data integration patterns, tools, and technologies defined, governed, and used
  \item [\thesection.6] Has the firm established a central data integration function
                        (i.e. integration Center of Excellence) to manage \glsxtrshort{etl} across both
                        internal and external data pipelines
\end{core-questions}

\ekgmmCapabilitySectionDimensions

The goal is not always a single source of data - but rather the ability to choose the right authoritative source
for the appropriate context.

\ekgmmCapabilitySectionLevelsOneFive

\ekgmmscoringlevelOne

\begin{scoring}

    \item All data sources are identified and documented for in-scope use cases
    \item Do we know the authoritative source for each data set (should not be able to do integration without
          using approved authoritative source)
    \item Does everyone agree that we are using the right sources (the right source for every context) --
          link to governance
    \item Do we have an approved list of what each source feeds (precise description at the entity level that
          we can get from an approved source\,---\,must know if this is the primary source of the data per the
          use case context).
          For any given entity do I have all the potential sources and for a specific context do I know which
          is authorized.
    \item There is a defined governance process for change management and testing (clear picture of all the
          dependencies for data integration).
          If there are changes to authoritative sources\,---\,do we know the downstream implications (tracked and tested)
    \item Are all \glspl{technology-stack} known and supported by current teams (are all key systems under the
          management and governance of the organization\,---\,should not have ghost systems that are not controlled
          as part of the integration process)
    \item Entitlement policies\index{entitlement!policies} and classification rules (i.e. security, PII,
          business sensitive) are defined and verified
    \item Data Quality\index{data!quality} requirements are defined, documented, and verified

\end{scoring}

\ekgmmscoringlevelTwo

\begin{scoring}

    \item All information (above) are identified, precisely defined, and on-boarded into the knowledge graph
    \item Able to do datapoint lineage (detailed and complete view of the data integration landscape)
    \item Start making the \gls{ekg} the central point for data integration (the \gls{ekg} becomes the Rosetta stone of
          integration)\,---\,onboard systems, convert to RDF, integrate into \gls{ekg} (defined as the
          data integration strategy\,---\,not necessarily complete)
    \item All data sets that are on-boarded into the \gls{ekg} are coming from the authoritative sources.
          There are no man-in-the-middle systems.
          The goal is direct from the authoritative source to the target system for in-scope use cases.
          Must get the most granular data directly from the authoritative sources.
    \item All datasets are "\glspl{sdd}".
    \item [policy] all data is obtained from the \gls{ekg} as the authoritative source.
          Do not go directly to the originating source of the data.
    \item Entitlement policies and classification requirements are on-boarded into the \gls{ekg}
    \item \hyperref[sec:ekg-mm-data-quality-business-rules]{Data quality business rules}
          are on-boarded into the \gls{ekg}

\end{scoring}

\ekgmmscoringlevelThree

\begin{scoring}

    \item Data is precisely defined (granular level) - expressed as formal ontologies - and on-boarded into the \gls{ekg}
    \item All data flows are modeled, defined, and registered in the \gls{ekg} (full lineage in the \gls{ekg} for all
          in-scope applications)
    \item Start to make the \gls{ekg} the authoritative source (set-up to facilitate decommissioning of systems).
          The \gls{ekg} is structured to become the “new” system for in-scope applications (as soon as all
          connections emanate from the \gls{ekg}).
    \item Entitlements are automatically managed enforced

\end{scoring}

\ekgmmscoringlevelFour

\begin{scoring}

    \item [policy] All downstream client systems are using authoritative sources as the only source of information
          for in-scope datasets (\gls{ekg} is in the middle of all data flows)
    \item All “\iindex{cottage industry} systems” are replaced by the \gls{ekg}
          (and \gls{ekg} is able to perform all the requirements of any system it replaced --
          reporting, entitlement, quality control)

\end{scoring}


\chapter{Data Quality}\label{ch:ekg-mm-b-3} % B.3 Data Quality
\index{data quality}

The \nameref{ch:ekg-mm-b-3} component has the following capabilities:

\begin{itemize}[leftmargin=.5in]
  \item [\ref{sec:ekgmm-b-3-1}] \nameref{sec:ekgmm-b-3-1}\,---\,Strategy and approach for managing data quality
  \item [\ref{sec:ekgmm-b-3-2}] \nameref{sec:ekgmm-b-3-2}\,---\,Approach and rules to validate fit-for-purpose data quality
  \item [\ref{sec:ekgmm-b-3-3}] \nameref{sec:ekgmm-b-3-3}\,---\,Checkpoints and control processes for managing data quality
\end{itemize}

%
% B.3.1 Data Quality Framework
%
\ekgmmCapability{b-3-1}{data-quality-framework}{Data Quality Framework}

\ekgmmContextSection

Data in the knowledge graph is structured around interconnectivity and precision.
The goal is assurance of granular meaning so that users have confidence they are getting all the information they
need to understand context and solve ad hoc business questions.
In a semantic environment, every datapoint is resolved to a universally unique identifier, ensuring discoverability
across repositories and domains.
Data and metadata are connected so that logical errors and data inconsistencies are detected before they enter
the system.
From a compliance perspective, data in the graph is immutable because \iindex{lineage} can be traced and
nothing can be deleted except by policy.

\ekgmmcorequestionssection

\begin{core-questions}

  \item [\thesection.1] Has the data management strategy for the organization been defined, verified and aligned
                        with the \gls{operating-model}
  \item [\thesection.2] Have business requirements been captured and linked to granular data concepts
  \item [\thesection.3] Have relevant data sources been identified and linked to \glspl{sor}
  \item [\thesection.4] Is data precisely defined and mapped to enterprise models\todo{or any semantic
                        machine-readable model i.e. ontology?}
  \item [\thesection.5] Is the governance infrastructure (roles, responsibilities, funding requirements,
                        measurement criteria) imple\-mented and operational
  \item [\thesection.6] Are communications mechanisms in place to ensure that quality issues are verified,
                        addressed at source and linked to consuming applications
  \item [\thesection.7] Are end-users getting the data they need (and able to use it without the need for
                        reconciliation or manual transformation)

\end{core-questions}

\subsection*{Concepts to Discuss}

\begin{enumerate}

  \item have business requirements been captured and verified in the form of use case trees and business user stories
  \item are all data sources identified; defined in the knowledge graph; SOR and authorized distribution points;
  \item all data in machine-readable form and linked to ontologies

\end{enumerate}

\section[Business Rules]{Data Quality Business Rules}% B.3.2 Business Rules
\label{sec:ekgmm-b-3-2}
\label{sec:ekg-mm-data-quality-business-rules}
\index{data quality}

Data Quality business rules ensure that the data is fit for its intended purpose.
Subject matter experts specify the criteria used to validate and enforce data integrity.
The criteria are translated into agreed specifications (i.e. business rules) which are later codified for
\iindex{data profiling} or measuring conformity.
Data quality business rules can also be embedded into data capture systems to ensure validity at source.

\ekgmmContextSection

Data quality is an integrated feature of the knowledge graph.
Data quality rules are linked to structured business vocabularies\todo{or better: ontologies} to ensure that meaning
is shared and not obscured by vague terms or cryptic codes.
The logic of business rules and policies are\change{is} captured and expressed as executable models and consistently
enforced across all systems and processes.
These quality constraints (models) allow firms to measure the quality of the data and perform verification across
disparate systems.
In the mature \gls{ekg}, violations of logic or integrity are identified and prevented before data
enters the system.

\kgmmcorequestionssection

\begin{core-questions}

  \item [\thesection.1] Are data quality business rules specified, formalized, and expressed in a standardized manner
  \item [\thesection.2] Is there a clear line of business \iindex{accountability} (owned, funded, and governed) for the
                        \iindex{data quality} rules
  \item [\thesection.3] Is there a centrally managed repository of business rules\improve{explain that it does not
                        necessarily have to be central as long as it is agreed and enforced at the right scope}
  \item [\thesection.4] Is there a clearly defined mechanism for logging additions and performing updates
  \item [\thesection.5] Are the business rules aligned with business applications and traceable to source systems
  \item [\thesection.6] Are the data quality business rules automated and expressed in a machine-executable format

\end{core-questions}

\kgmmscoringsection

\kgmmscoringlevelOne

\begin{scoring}

  \item \hyperref[sec:ekg-mm-data-quality-business-rules]{Data quality business rules} (conditions) have been defined,
        documented and, verified by \glspl{sme} (process for evaluation and acceptance defined)
  \item Business rules are aligned with in-scope use cases and specific user stories
  \item Business rules are standardized and registered into a repository with a defined mechanism for logging
        additions and performing updates

\end{scoring}

\kgmmscoringlevelTwo

\begin{scoring}

  \item A defined architecture exists to translate business rules into machine-executable code (some rules will be
        OWL\index{OWL} expressions, some will be SHACL shapes\index{SHACL} and constraints, some will be translated into
        workflow logic)
  \item Business \iindex{provenance} and \iindex{lineage} are traceable across the \iindex{data supply chain} and
        evaluated against defined business rules (all business rules must be traceable and understandable in
        context\,---\,must understand the purpose and importance of the rule)
  \item Business rules for in-scope use cases are implemented in the \gls{ekg}

\end{scoring}

\kgmmscoringlevelThree

\begin{scoring}

    \item [Metrics] The measurement criteria are defined for data quality business rules (which rules are executed,
          how often, improvement)
    \item [Performance] The value of business rules are related to business concepts (products, financial performance,
          organizational objectives)\,---\,able to trace the core relationship between the business objectives and
          the data quality business rules (correlation between rules and outcomes are known, able to be queried and
          traceable within the \gls{ekg})

\end{scoring}

\kgmmscoringlevelFour

\begin{scoring}

    \item Business rules are combined with \gls{ai} capability for compliance (dynamic optimization of
          business rules)
    \item Model-driven (senior management can begin to optimize business objectives using business rules in
          the \gls{ekg}\,---\,i.e. alignment of business rules with “what if” scenarios)

\end{scoring}

\kgmmscoringlevelFive

\begin{scoring}

    \item All business rules are driven by business objectives (objectives are in the \gls{ekg} with appropriate scorecards)

\end{scoring}

%
% B.3.3 Data Quality Execution
%
\ekgmmCapability{b-3-3}{data-quality-execution}{Data Quality Execution}
\index{data!quality!execution}

\ekgmmCapabilitySectionContributionToEnterprise

Data in the knowledge graph is aligned to precise meaning so that errors and definitional conflicts are
verified at source before they are introduced into operational systems.
In the knowledge graph data, rules and metadata are co-resident and reusable without impedance mismatch.
Quality execution is rules-based and unhooked from both schemas and data models that are tailored to
specific applications.
Metadata is embedded into the content so that users always understand what the data represents as it moves across
organizational boundaries.

\ekgmmCapabilitySectionDimensions

\begin{core-questions}

  \item [\thesection.1] Is there a defined process for executing data quality across linked applications
  \item [\thesection.1] Is there a link between business rules and execution
  \item [\thesection.1] Is the governance structure (and funding mechanisms) for managing data quality issues
                        operational
  \item [\thesection.1] Have service level and data sharing agreements been defined and verified by consumers
  \item [\thesection.1] Is there a defined (and repeatable) process for tracing data quality issues to
                        systems of record and downstream applications
  \item [\thesection.1] Is data quality remediation coordinated across the data production and consumption process
  \item [\thesection.1] Are data quality metrics and scorecards captured and reported to involved stakeholders

\end{core-questions}



\chapter{Data Governance}\label{ch:ekg-mm-b-4} % B.4 Data Governance

The \nameref{ch:ekg-mm-b-4} component has the following capabilities:

\begin{itemize}[leftmargin=.5in]
  \item [\ref{sec:ekg-mm-b-4-1}] \nameref{sec:ekg-mm-b-4-1}\,---\,Structure and approach for the \gls{ekg} \gls{coe}
  \item [\ref{sec:ekg-mm-b-4-2}] \nameref{sec:ekg-mm-b-4-2}\,---\,Policies, procedures, and standards for managing the data lifecycle
  \item [\ref{sec:ekg-mm-b-4-3}] \nameref{sec:ekg-mm-b-4-3}\,---\,Management of the data production and manufacturing process
  \item [\ref{sec:ekg-mm-b-4-4}] \nameref{sec:ekg-mm-b-4-4}\,---\,\gls{ekg} access control mechanisms
  \item [\ref{sec:ekg-mm-b-4-5}] \nameref{sec:ekg-mm-b-4-5}\,---\,Identification and tracking of \glsfmtlongpl{cde} and uses
  \item [\ref{sec:ekg-mm-b-4-6}] \nameref{sec:ekg-mm-b-4-6}\,---\,Creation and implementation of the data control environment
\end{itemize}

%
% B.4.1 Data-management Operating Model
%
\ekgmmCapability{b-4-1}{data-management-operating-model}{Data-management \Glsfmttext{operating-model}}

\ekgmmContextSection

The \gls{operating-model} for the \glsfirst{ekg:coe} establishes a new way of running the organization
to enable companies to accelerate development and operate more efficiently.
The \glsxtrshort{ekg} \iindex{governance framework}\index{\glsfmtshort{ekg}!governance framework} focuses on
combining new data capabilities (i.e. resolvable identity, standard ontologies, open standards)
as part of an integrated process.
A well defined \gls{operating-model} specifies architecture components that support flexible and reusable data to help
organizations achieve improvements in revenue, customer experience and cost.

\ekgmmcorequestionssection

\begin{core-questions}

  \item [\thesection.1] Have the underlying principles (and challenges) of data management been established and
                        accepted by involved stakeholders
  \item [\thesection.2] Does the \gls{odm} have authority to enforce adherence to policy
  \item [\thesection.3] Has the data management funding model been established and implemented
  \item [\thesection.4] Has the organization identified and recruited stakeholders with sufficient skill sets to
                        implement the data management program
  \item [\thesection.5] Are involved stakeholders held accountable to data management program deliverables
  \item [\thesection.6] Has the organization defined and validated the \glspl{operating-model},
                        and workflows necessary to implement the data management program
  \item [\thesection.7] Are metrics and KPIs captured and actively used to improve the data management operating process
  \item [\thesection.8] Is the data management \gls{operating-model} audited for compliance and effectiveness

\end{core-questions}


%
% B.4.2 Data-management Policy
%
\ekgmmCapability{b-4-2}{data-management-policy}{Data-management Policy}

\ekgmmCapabilityContributionToEnterprise{b-4-2}

\ekgmmCapabilityContributionToEKG{b-4-2}

\begin{maturity-dimensions}

  \item Are policies and standards in alignment with data management strategy
  \item Are data management policies documented, complete and operational
  \item Are data management policies linked to operational control functions as well as the
        SDLC process of the organization
  \item Have data policies been created in collaboration with (and approved by) business, technology
        and operational stakeholders
  \item Are data management policies aligned with the requirements of the \glsxtrshort{ekg}
  \item Have data management policies been approved by executive management and audit

\end{maturity-dimensions}


%
% B.4.3 Data Production \& Consumption
%
\ekgmmCapability{b-4-3}{data-production-and-consumption}{Data Production \& Consumption}

\ekgmmContextSection

Management of the production/consumption process is one of the essential components of data management.
It is based on an understanding of what data is being exchanged and the ability to map dependencies and
transformations (\iindex{lineage}).
The knowledge graph is the logical distribution point for data manufacturing\,---\,leveraging the ontology to ensure
consistency of meaning across systems, people and processes.
The \glsxtrshort{ekg} can trace data flow and validate that defined use cases are aligned with agreed usage and
obtained from authoritative sources.
Entitlements and redistribution rights are automatically tracked and fully auditable.
Data quality rules\index{data!quality!rules} rules are linked to business vocabulary and structurally enforced
across systems and processes to ensure consistency.

\ekgmmcorequestionssection

\begin{core-questions}

  \item [\thesection.1] Is the data production/consumption process documented and verified from authoritative sources
                        to consuming applications
  \item [\thesection.2] Are the requirements for data quality (and criticality) defined, verified and expressed as SLAs
                        or other forms of data sharing agreement
  \item [\thesection.3] Is there an inventory of SLAs and is data production monitored against these agreements
  \item [\thesection.4] Is the data production/consumption process connected to ontologies and
                        official business glossaries
  \item [\thesection.5] Are entitlements, permissions and authorizations tracked as data flows across systems
  \item [\thesection.6] Can data quality errors and deficiencies be traced back to where the data originated
  \item [\thesection.7] Are the data production/consumption governance processes in place and operational
                        (\iindex{ownership} and \iindex{accountability})

\end{core-questions}


\section{Entitlement Management}\label{sec:ekg-mm-b-4-4} % B.4.4 Entitlement Management

Entitlement management is technology that grants and enforces access rights to data, devices, systems and services.
Entitlement systems are linked to organizational policies (rules) governing access.
These systems track access to applications to ensure that actions are in line with policies and to provide data
about access for audit purposes.
Entitlements need to be managed in sync with the goals for security, business continuity and compliance.
There can be multiple proprietary systems that all have unique entitlement structures and
individuals can move across departments and perform a variety of roles.

\ekgmmContextSection

Classification and access rights within the \gls{ekg} are linked to dataflow and managed at the granular level.
\iindex{Lineage} is automatically tracked and fully auditable by source, purpose and responsible party.
Security is embedded into the design of the data and not constrained by either systems or administrative complexity.
Rules can be modeled for all circumstances and controlled at both the datapoint and applications level.
Using a knowledge graph for fine-grained access control reduces complexity and enhances enforcement capability.

\kgmmcorequestionssection

\begin{core-questions}

  \item [\thesection.1] Are the rules for entitlement and access control documented and verified for defined circumstances
  \item [\thesection.2] Are entitlement rules and application logic expressed in machine-executable format
  \item [\thesection.3] How are entitlement permissions (and changes) administered, tracked, and audited
  \item [\thesection.4] At what level are entitlements managed (i.e. datapoint, platform, applications, role)
  \item [\thesection.5] Are entitlement classifications linked to lineage, dataflow and transformation processes
  \item [\thesection.6] Are entitlements and access control linked to \glspl{sor}

\end{core-questions}

\kgmmscoringsection

\kgmmscoringlevelOne

\begin{scoring}

  \item Manual alignment of law, policy and owner requirements to rule (source of the data that determines entitlement
        imported from systems like Active Directory or LDAP and the like “shadow IT”) –-
        KG is not yet the authoritative source
  \item Administration requires manual evaluation
  \item \glsfirst{rbac} (conventional state of the art) implemented for the limited KG use cases

\end{scoring}

\kgmmscoringlevelTwo

\begin{scoring}

  \item Strategy (exist) to move from Active Directory to KG as the authoritative source
        (artifacts = documents and plans)
  \item Planning underway to convert the KG as the source of entitlement control
        (goal is to eliminate manual process for update)
  \item More advanced levels of granularity (data source owner’s policies) are captured in the KG
  \item Data source prioritization (based on data owners policies) incorporated into the graph –-
        must protect data that owners “manage data control in the same strict way that the data owner
        already has in place” (technical assurance to data owner that the sensitive and private data will be controlled)
  \item Entitlements are linked to governance processes and internal data process flow (auditable)

\end{scoring}

\kgmmscoringlevelThree

\begin{scoring}

    \item \glsfmtshort{ekgbac} –- artifacts = demonstration of \gls{ekgbac}
          (onboarded the organizational management use case; organizational units, employees, contractors are in
          the KG and linked to \gls{hr}) –- KG must know who you are, roles and access rights before level 3
    \item All people, processes and data (organizational management landscape use case) are linked
    \item Law, regulation, internal policies and requirements of original data owner are modeled as ontology
          plus business logic (and corresponding data is onboarded)
    \item Entitlements are automatically enforced based (on executable policy)
    \item Must be able to push entitlement policies into the KG
    \item The \gls{ekg} covers all domains and all jurisdictions for the scope of its use cases –-
          complete overview of all related regulations, laws and policies;
          all factors that have an impact on enterprise decisions are known and part of the EKG
    \item \gls{ekg} uses data from many sources; systems owners will participate as long as they are confident that
          the requirements are equivalent (EKG and existing system must be kept in alignment for each data source
    \item Entitlement requirements are linked to the actual transactions and “things” that need to be protected

\end{scoring}

\kgmmscoringlevelFour

\begin{scoring}

  \item Entitlements are linked to AI and reasoning capability to ensure that all entitlement decisions are
        in line with business objectives and organizational goals
  \item Entitlements become more “intelligent” and allow for flexible analysis on the value of
        activities and decisions (i.e. is this a smart thing for us to do)
  \item Artifacts = KG must use advanced AI capabilities; must see evidence of AI algorithms
        (show us that you have \gls{ai} capabilities (must have onboarded the organizational objectives,
        business objectives and roles –- all connected to the organizational management landscape;
        linked to \nameref{sec:ekg-mm-risk-management} (risk factors are linked to entitlement);
        linked to cost/profitability

\end{scoring}

\kgmmscoringlevelFive

\begin{scoring}

  \item All mundane and repetitive tasks are implemented automatically
        (all workflows and approval steps are done by the EKG) –- artifacts = workflows and
        approval steps are fully automated
  \item Entitlements and pervasive and comprehensive for all activities
  \item Little need for human interaction; completely controlled environment about access rights and data usage

\end{scoring}

%
% B.4.5 Classification Management
%
\ekgmmCapability{b-4-5}{classification-management}{Classification Management}

\ekgmmContextSection

In the \glsxtrshort{ekg}, \glsfmtlongpl{cde} are defined by data quality business rules\index{data!quality!rules}
and expressed as machine-executable models\index{executable models}.\improve{i.e. ontologies}
\improve[inline]{%
    JAG>We should be more explicit about this by saying that \glspl{cde}
    basically do not exist as such anymore, a data element is deemed critical from the point of view of the
    enterprise perhaps but that's just one viewpoint, one context.
    In the \gls{ekg} we have to serve any viewpoint.
    For every use case there are data elements (concepts) that are critical which may be a whole
    different set of data elements than what enterprise level deems to be critical.
    All data elements are critical somewhere in some context.
    This has to do with the "single version of the truth delusion".
    It's of course fine to label certain data elements as "critical" but it should be tied to the use case.
    Data element X is critical for use case Y.},
These rules are automatically executed across systems, processes and applications to ensure consistency.
Data is linked to the ontology and resolved to a single identifier to mitigate confusion about meaning when the
data is onboarded or transformed.
The \glsxtrshort{ekg} is able to trace data flow and verify that criticality is aligned with \glspl{sor}
and agreed usage.

\ekgmmcorequestionssection

\begin{core-questions}

  \item [\thesection.1] Are critical data elements identified, verified and prioritized for specific
                        use cases and applications
  \item [\thesection.2] How is the organization managing the distinctions between granular data attributes and
                        derived/calculated business concepts
  \item [\thesection.3] Is the inventory of critical data elements implemented and linked to how the
                        data is being consumed
  \item [\thesection.4] Are critical business elements connected to business glossaries, ontologies,
                        \glspl{sor} and authorized distribution points
  \item [\thesection.5] Is the front-to-back flow of data defined, validated and linked to the designations of
                        critical data
  \item [\thesection.6] Are the governance mechanisms for managing critical data defined and operational

\end{core-questions}


%
% B.4.6 Risk and Control Environment
%
\ekgmmCapability{b-4-6}{risk-and-control-environment}{Risk \& Control Environment}

\ekgmmCapabilitySectionContributionToEnterprise

Control processes in the \glsxtrshort{ekg} environment are managed as a structured set of executable business rules
or models\,---\,in the EKG\,---,\ and automatically enforced across all use cases.
The meaning of the data is structurally validated at the point of data capture to prevent bad data from
entering the system\improve{avoid "the system", use "EKG"} at ingestion.
In the knowledge graph, data and metadata are fully integrated to ensure reusability across systems
and operational processes.
The \glsxtrshort{ekg} is able to map data flow including all dependencies and transformations to verify that
data is obtained from authorized \glspl{sor}, identified according to access rights and
aligned with intended usage.

\ekgmmCapabilitySectionDimensions

\begin{core-questions}

  \item [\thesection.1] Has the organization developed a structured framework outlining the principles of how
                        operational risk is identified, assessed, monitored and controlled
  \item [\thesection.2] Has the operational risk framework been adopted by executive management and verified by
                        internal audit
  \item [\thesection.3] Have oversight mechanisms been adopted to ensure compliance with operational risk
                        control policies and governance procedures
  \item [\thesection.4] Are technology resiliency and continuity plans in place to ensue systems integrity,
                        security and availability during mergers, acquisitions and consolidations

\end{core-questions}



