\bookmarksetup{bold=true, color=blue}
\setcounter{part}{1}
\part{Data Pillar}\label{pt:ekgmm-b} % B Data

The Data Pillar has the following components:

\begin{itemize}[leftmargin=.5in]
  \item [\ref{ch:ekg-mm-b-1}] \nameref{ch:ekg-mm-b-1} % B.1 Data Strategy
  \item [\ref{ch:ekg-mm-b-2}] \nameref{ch:ekg-mm-b-2} % B.2 Data Architecture
  \item [\ref{ch:ekg-mm-b-3}] \nameref{ch:ekg-mm-b-3} % B.3 Data Quality
  \item [\ref{ch:ekg-mm-b-4}] \nameref{ch:ekg-mm-b-4} % B.4 Data Governance
\end{itemize}

%\paragraph{Levels}
%
%\begin{description}[nosep,font=\bfseries]
%
%    \item [1. \ekgmmLevelOneLabel]
%    Core data management capabilities (\gls{operating-model}, inventory, data architecture,
%    \hyperref[sec:ekg-mm-business-vocabularies]{vocabularies, business glossary or terminology},
%    pipeline management, etc.) are being performed.
%    Specific use cases are being implemented with specialist teams for the pilot initiative.
%
%    \item [2. \ekgmmLevelTwoLabel]
%    Critical data elements are prioritized in the ontology.
%    Approach to identity and meaning resolution is established.
%    Use case trees are defined and modeled to capture shared data relationships.
%    The knowledge graph is becoming the central point for integration.
%
%    \item [3. \ekgmmLevelThreeLabel]
%    Inventory is embedded into the \gls{ekg} and linked to governance.
%    Data is expressed as formal ontologies, onboarded into the \gls{ekg} and searchable.
%    Data flows are defined and modeled.
%    The \gls{ekg} is the authoritative source for data.
%
%\end{description}

\chapter{Data Strategy}\label{ch:ekg-mm-b-1}\label{ch:ekg-mm-data-strategy} % B.1 Data Strategy

The \nameref{ch:ekg-mm-b-1} component has the following capabilities:

\begin{itemize}[leftmargin=.5in]
    \ekgmmCapabilityItem{b-1-1}
    \ekgmmCapabilityItem{b-1-2}
\end{itemize}

%
% B.1.1 Goals & Objectives
%
\ekgmmCapability{b-1-1}{data-strategy-goals-and-objectives}{Goals \& Objectives}

We welcome your input here

\section{Business Case}\label{sec:ekg-mm-b-1-2} % B.1.2 Business Case

Business rationale, justification and \gls{roi} of the \gls{ekg}.

We welcome your input here.


\chapter{Data Architecture}\label{ch:ekg-mm-b-2}\label{ch:ekg-mm-data-architecture} % B.2 Data Architecture

The \nameref{ch:ekg-mm-b-2} component has the following capabilities:

\begin{itemize}[leftmargin=.5in]
  \ekgmmCapabilityItem{b-2-1}
  \ekgmmCapabilityItem{b-2-2}
  \ekgmmCapabilityItem{b-2-3}
  \ekgmmCapabilityItem{b-2-4}
\end{itemize}


%
% B.2.1 Ontologies
%
\ekgmmCapability{b-2-1}{ontologies-and-data-models}{Ontologies}

\ekgmmCapabilitySectionContributionToEnterprise

An ontology is about:

\begin{itemize}
    \item explicit meanings and relationships; the terms used are less important
    \item a combination of definitions in both text and logic
\end{itemize}

An ontology can be the basis of, but is broader than:
\begin{itemize}
    \item a taxonomy
    \item a vocabulary
    \item a data or object model
    \item a conceptual model
    \item a specific serialization format
\end{itemize}

Ontologies can be expressed at different levels of sophistication, with different scopes,
and in a combination of languages.
The basic structures include:

\begin{itemize}[leftmargin=.8in,font=\itshape]
    \item [individual] a representation of a business object or item which is the subject of information to be managed.
                       An individual usually has a unique identity.
                       For example \lstinline|Person X| or \lstinline|Shipment Y|.
                       Many such individuals might represent the same real world object.
    \item [data value] strings, numbers, dates which represent the data.
    \item [property]   a type of data point that may be associated with individuals.
                       An individual, a property and a value\,---\,which may be a data value or another
                       individual\,---\, form a triple.\newline
                       For example person \lstinline|X hasBirthDate D| or person \lstinline|X hasMother Y|.
                       Triples whose value is another individual form relationships.
                       Properties may have generalizations, for example hasMother is a subProperty of hasParent.
    \item [class]      a category applied to individuals, that determines what you can do with them,
                       the properties you can expect to see, and the rules that might apply;
                       an individual may be a member of many classes associated;
                       classes may have generalizations.
                       Note that, unlike more traditional approaches, properties are independent of classes though
                       they may be used to infer class membership.
                       For example, given the triple \lstinline|X hasMother Y| you may be able to infer that both
                       \lstinline|X| and \lstinline|Y| are members of the class \lstinline|Person|,
                       or at least \lstinline|Animal|.
    \item [ontology]   grouping of the above for management and identification purposes.
\end{itemize}

\ekgmmCapabilityContributionToEKG{b-2-1}

\ekgmmCapabilityContributionToEnterprise{b-2-1}

\ekgmmCapabilitySectionDimensions

\begin{itemize}
    \item How much of the enterprise data is covered by ontologies?
    \item How well is ontology coverage mapped to business need?
    \item To what extent are concepts independent of but mapped to terminology/vocabulary?
    \item Level of sophistication of textual and logic definitions
    \item Level of tooling is available and used
    \item Level of training and trained people
    \item Level of process (including change management), guidelines and standards
    \item Level of modularity and reuse - internal and external
    \item Extent of examples and tests
    \item Extent of traceability with different logical and physical data models
\end{itemize}

\ekgmmCapabilitySectionLevels

The following criteria for each level are abbreviated: each item is shorthand for:

\begin{itemize}
    \item documented process
    \item trained participants
    \item implemented process and/or technology
    \item monitoring and improvement
\end{itemize}

\ekgmmCapabilitySectionLevelsOneFive

\ekgmmscoringlevelOne

\begin{itemize}
    \item Minimal ontologies which could be as simple as a list of classes and properties used in graphs
    \item Basic metadata (definition, label) for each class and property
    \item Each individual (in data) has at least one explicit class
    \item Ontology coverage for each use case in scope of the project;
          project includes minimal number of ontologies and classes not justified by a use case
    \item Definitions catalogued and under change management

\end{itemize}

\ekgmmscoringlevelTwo

\begin{itemize}
    \item Ontologies expressed in a standard ontology language (could be as simple as \gls{ekg:rdf-schema})
    \item Common (shared or mapped) concepts across \gls{ekg} projects
    \item Ability to see ontology usage by use cases, vocabularies and datasets
    \item Namespace scheme established and used for new ontologies in the \gls{ekg}
    \item Ontology guidelines in place and implemented, including common metadata
    \item Documented approach for external ontologies, including selection and adaptation
    \item Annotated example files for documentation and training
    \item Test files based on use cases covering all used ontology elements
    \item Ontology change management includes impact analysis and stakeholder approval
    \item Tooling for ontology diagrams and documentation
    \item Automated basic checking of ontology syntax
    \item Access to at least one trained \gls{ekg:ontologist}
\end{itemize}

\ekgmmscoringlevelThree

\begin{itemize}
    \item Modeling of required data and constraints by use case, including for stored and communicated data
    \item Automated validation of ontologies (for guideline compliance, and for logical consistency),
          with results as triples
    \item Automated testing and validation of test data with ontologies (per use case)
    \item Separation of concerns to support enterprise management such as bi-temporality, transactions and events
    \item Automated transformation of ontologies to use common serialization and metadata
    \item Automated checking of ontologies against different profiles (e.g. OWL-RL) to check for technology support
    \item Automated checking of ontologies against different best practices
    \item Ontology source changes linked to automated KGOps for testing and deployment
    \item Impact analysis identifies ontology breaking changes which require fixes to existing EKG data
    \item EKG-wide ontology browsing and searching
    \item Follow-your-nose UI starting from any ontology element URI
    \item Follow-your-nose API starting from any ontology element URI
    \item Trained ontologist available to each project (possibly via \gls{ekg:coe})
\end{itemize}

\ekgmmscoringlevelFour

\begin{itemize}
    \item Separation of ontologies from vocabularies, with multiple vocabularies for different communities
          mapped to the same concepts
    \item Ontology architecture management process, including use of patterns and modularity
    \item Generation of logic into business language
    \item Automated fixes to existing EKG data in response to ontology breaking changes
    \item Basic ontology metrics and reporting, including usage in data
    \item Generation of ontologies/shapes for external interchange
\end{itemize}

\ekgmmscoringlevelFour

\begin{itemize}
    \item Sophisticated ontology metrics and reporting, including trends
    \item Matching and differencing of ontologies from different sources
    \item Automated matching of ontologies with vocabularies
    \item Generation of validation code for external interchange
    \item Wizard for developing ontologies from business questions
    \item Inducing of ontologies from instance data
\end{itemize}

\input{../../ekg-mm/sections/b-2-data-architecture-2-inventory-management.tex}
\input{../../ekg-mm/sections/b-2-data-architecture-3-business-terminology.tex}
\input{../../ekg-mm/sections/b-2-data-architecture-4-data-integration.tex}

\chapter{Data Quality}\label{ch:ekg-mm-b-3} % B.3 Data Quality
\index{data!quality}

The \nameref{ch:ekg-mm-b-3} component has the following capabilities:

\begin{itemize}[leftmargin=.5in]
  \ekgmmCapabilityItem{b-3-1}
  \ekgmmCapabilityItem{b-3-2}
  \ekgmmCapabilityItem{b-3-3}
\end{itemize}

\section[Framework]{Data Quality Framework}% B.3.1 Data Quality Framework
\label{sec:ekg-mm-b-3-1}
\label{sec:ekg-mm-data-quality-framework}

Data quality is a measurement of the degree to which any dataset is fit for its intended purpose.
It is based on an understanding of application requirements and derived by reverse engineering of the
\iindex{data production process}.
A data quality framework is an agreed methodology including operational controls, governance processes and
measurement mechanisms.
The framework is designed to support organizational priorities for data quality based on criticality and business value.

\ekgmmContextSection

Data in the knowledge graph is structured around interconnectivity and precision.
The goal is assurance of granular meaning so that users have confidence they are getting all the information they
need to understand context and solve ad hoc business questions.
In a semantic environment, every datapoint is resolved to a universally unique identifier, ensuring discoverability
across repositories and domains.
Data and metadata are connected so that logical errors and data inconsistencies are detected before they enter
the system.
From a compliance perspective, data in the graph is immutable because \iindex{lineage} can be traced and
nothing can be deleted except by policy.

\kgmmcorequestionssection

\begin{core-questions}

  \item [\thesection.1] Has the data management strategy for the organization been defined, verified and aligned
                        with the \gls{operating-model}
  \item [\thesection.2] Have business requirements been captured and linked to granular data concepts
  \item [\thesection.3] Have relevant data sources been identified and linked to \glspl{sor}
  \item [\thesection.4] Is data precisely defined and mapped to enterprise models\todo{or any semantic
                        machine-readable model i.e. ontology?}
  \item [\thesection.5] Is the governance infrastructure (roles, responsibilities, funding requirements,
                        measurement criteria) imple\-mented and operational
  \item [\thesection.6] Are communications mechanisms in place to ensure that quality issues are verified,
                        addressed at source and linked to consuming applications
  \item [\thesection.7] Are end-users getting the data they need (and able to use it without the need for
                        reconciliation or manual transformation)

\end{core-questions}

\subsection*{Concepts to Discuss}

\begin{enumerate}

  \item have business requirements been captured and verified in the form of use case trees and business user stories
  \item are all data sources identified; defined in the knowledge graph; SOR and authorized distribution points;
  \item all data in machine-readable form and linked to ontologies

\end{enumerate}

%
% B.3.2 Business Rules
%
\ekgmmCapability{b-3-2}{data-quality-business-rules}{Data Quality Business Rules}
\index{data!quality!rules}

\ekgmmCapabilitySectionContributionToEnterprise

Data quality is an integrated feature of the \gls{ekg}.
Data quality rules are linked to structured business vocabularies\todo{or better: ontologies} to ensure that meaning
is shared and not obscured by vague terms or cryptic codes.
The logic of business rules and policies are\change{is} captured and expressed as executable models and consistently
enforced across all systems and processes.
These quality constraints (models) allow firms to measure the quality of the data and perform verification across
disparate systems.
In the mature \gls{ekg}, violations of logic or integrity are identified and prevented before data
enters the system.

\ekgmmCapabilitySectionDimensions

\begin{core-questions}

  \item [\thesection.1] Are data quality business rules specified, formalized, and expressed in a standardized manner
  \item [\thesection.2] Is there a clear line of business \iindex{accountability} (owned, funded, and governed) for the
                        data quality rules\index{data!quality!rules}
  \item [\thesection.3] Is there a centrally managed repository of business rules\improve{explain that it does not
                        necessarily have to be central as long as it is agreed and enforced at the right scope}
  \item [\thesection.4] Is there a clearly defined mechanism for logging additions and performing updates
  \item [\thesection.5] Are the business rules aligned with business applications and traceable to source systems
  \item [\thesection.6] Are the data quality business rules automated and expressed in a machine-executable format

\end{core-questions}

\ekgmmCapabilitySectionLevelsOneFive

\ekgmmscoringlevelOne

\begin{scoring}

  \item \hyperref[sec:ekg-mm-data-quality-business-rules]{Data quality business rules} (conditions) have been defined,
        documented and, verified by \glspl{sme} (process for evaluation and acceptance defined)
  \item Business rules are aligned with in-scope use cases and specific user stories
  \item Business rules are standardized and registered into a repository with a defined mechanism for logging
        additions and performing updates

\end{scoring}

\ekgmmscoringlevelTwo

\begin{scoring}

  \item A defined architecture exists to translate business rules into machine-executable code (some rules will be
        OWL\index{OWL} expressions, some will be SHACL shapes\index{SHACL} and constraints, some will be translated into
        workflow logic)
  \item Business \iindex{provenance} and \iindex{lineage} are traceable across the \iindex{data!supply chain} and
        evaluated against defined business rules (all business rules must be traceable and understandable in
        context\,---\,must understand the purpose and importance of the rule)
  \item Business rules for in-scope use cases are implemented in the \gls{ekg}

\end{scoring}

\ekgmmscoringlevelThree

\begin{scoring}

    \item [Metrics] The measurement criteria are defined for data quality business rules (which rules are executed,
          how often, improvement)
    \item [Performance] The value of business rules are related to business concepts (products, financial performance,
          organizational objectives)\,---\,able to trace the core relationship between the business objectives and
          the data quality business rules (correlation between rules and outcomes are known, able to be queried and
          traceable within the \gls{ekg})

\end{scoring}

\ekgmmscoringlevelFour

\begin{scoring}

    \item Business rules are combined with \gls{ai} capability for compliance (dynamic optimization of
          business rules)
    \item Model-driven (senior management can begin to optimize business objectives using business rules in
          the \gls{ekg}\,---\,i.e. alignment of business rules with “what if” scenarios)

\end{scoring}

\ekgmmscoringlevelFive

\begin{scoring}

    \item All business rules are driven by business objectives (objectives are in the \gls{ekg} with appropriate scorecards)

\end{scoring}

\section[Execution]{Data Quality Execution}% \thesection.3 Data Quality Execution
\label{sec:ekg-mm-b-3-3}
\label{sec:ekg-mm-data-quality-execution}
\index{data quality}

Implementation of the data quality program is a control process mechanism for organizations.
It is driven by the execution of business rules at the point of data capture (validation).
key components include data profiling, root cause analysis, data remediation and issue management.
The primary objective of the data quality program is to accurately and consistently represent the concepts and
values needed to meet the needs of consumers.
A well-orchestrated data quality program requires communication and operational coordination across the
organizational landscape.

\ekgmmContextSection

Data in the knowledge graph is aligned to precise meaning so that errors and definitional conflicts are
verified at source before they are introduced into operational systems.
In the knowledge graph data, rules and metadata are co-resident and reusable without impedance mismatch.
Quality execution is rules-based and unhooked from both schemas and data models that are tailored to
specific applications.
Metadata is embedded into the content so that users always understand what the data represents as it moves across
organizational boundaries.

\kgmmcorequestionssection

\begin{core-questions}

  \item [\thesection.1] Is there a defined process for executing data quality across linked applications
  \item [\thesection.1] Is there a link between business rules and execution
  \item [\thesection.1] Is the governance structure (and funding mechanisms) for managing data quality issues
                        operational
  \item [\thesection.1] Have service level and data sharing agreements been defined and verified by consumers
  \item [\thesection.1] Is there a defined (and repeatable) process for tracing data quality issues to
                        systems of record and downstream applications
  \item [\thesection.1] Is data quality remediation coordinated across the data production and consumption process
  \item [\thesection.1] Are data quality metrics and scorecards captured and reported to involved stakeholders

\end{core-questions}



\chapter{Data Governance}\label{ch:ekg-mm-b-4} % B.4 Data Governance

The \nameref{ch:ekg-mm-b-4} component has the following capabilities:

\begin{itemize}[leftmargin=.5in]
  \ekgmmCapabilityItem{b-4-1}
  \ekgmmCapabilityItem{b-4-2}
  \ekgmmCapabilityItem{b-4-3}
  \ekgmmCapabilityItem{b-4-4}
\end{itemize}

\section[Operating Model]{Data-management \Glsfmttext{operating-model}}\label{sec:ekg-mm-b-4-1} % B.4.1 Operating Model

The people, processes, capabilities, and tools that define the role of data management in delivering value to an
organization’s customers.\footnote{%
    There is also a capability called ``\nameref{sec:ekg-mm-a-1-4}'' in the
    Business Pillar that has a different meaning, see \secref{sec:ekg-mm-a-1-4}.
}

The \gls{operating-model} can help stakeholders understand the complexity of the data manufacturing process and how
components relate to each other.
It specifies the roles and responsibilities of the stakeholders involved in the data management program.
It provides a framework and policies to help align governance concepts with operating requirements and
organizational culture.
An effective \gls{operating-model} can both describe the way the organization does business today (“as is”) and
communicate a vision of how an operation will work in the future (“to be”).

\ekgmmContextSection

The \gls{operating-model} for the \glsfirst{ekg:coe} establishes a new way of running the organization
to enable companies to accelerate development and operate more efficiently.
The \glsxtrshort{ekg} \iindex{governance framework}\index{\glsxtrshort{ekg}!governance framework} focuses on
combining new data capabilities (i.e. resolvable identity, standard ontologies, open standards)
as part of an integrated process.
A well defined \gls{operating-model} specifies architecture components that support flexible and reusable data to help
organizations achieve improvements in revenue, customer experience and cost.

\kgmmcorequestionssection

\begin{core-questions}

  \item [\thesection.1] Have the underlying principles (and challenges) of data management been established and
                        accepted by involved stakeholders
  \item [\thesection.2] Does the \gls{odm} have authority to enforce adherence to policy
  \item [\thesection.3] Has the data management funding model been established and implemented
  \item [\thesection.4] Has the organization identified and recruited stakeholders with sufficient skill sets to
                        implement the data management program
  \item [\thesection.5] Are involved stakeholders held accountable to data management program deliverables
  \item [\thesection.6] Has the organization defined and validated the \glspl{operating-model},
                        and workflows necessary to implement the data management program
  \item [\thesection.7] Are metrics and KPIs captured and actively used to improve the data management operating process
  \item [\thesection.8] Is the data management \gls{operating-model} audited for compliance and effectiveness

\end{core-questions}


%
% B.4.2 Data-management Policy
%
\ekgmmCapability{b-4-2}{data-management-policy}{Data-management Policy}

\ekgmmCapabilityContributionToEnterprise{b-4-2}

\ekgmmCapabilityContributionToEKG{b-4-2}

\begin{maturity-dimensions}

  \item Are policies and standards in alignment with data management strategy
  \item Are data management policies documented, complete and operational
  \item Are data management policies linked to operational control functions as well as the
        SDLC process of the organization
  \item Have data policies been created in collaboration with (and approved by) business, technology
        and operational stakeholders
  \item Are data management policies aligned with the requirements of the \glsxtrshort{ekg}
  \item Have data management policies been approved by executive management and audit

\end{maturity-dimensions}


%
% B.4.3 Classification Management
%
\ekgmmCapability{b-4-3}{classification-management}{Classification Management}

\ekgmmCapabilityContributionToEnterprise{b-4-3}

\ekgmmCapabilityContributionToEKG{b-4-3}

\begin{maturity-dimensions}

  \item Are critical data elements identified, verified and prioritized for specific
        use cases and applications
  \item How is the organization managing the distinctions between granular data attributes and
        derived/calculated business concepts
  \item Is the inventory of critical data elements implemented and linked to how the
        data is being consumed
  \item Are critical business elements connected to business glossaries, ontologies,
        \glspl{sor} and authorized distribution points
  \item Is the front-to-back flow of data defined, validated and linked to the designations of
        critical data
  \item Are the governance mechanisms for managing critical data defined and operational

\end{maturity-dimensions}


\input{../../ekg-mm/sections/b-4-data-governance-4-risk-and-control-framework.tex}

