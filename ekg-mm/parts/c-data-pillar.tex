\bookmarksetup{bold=true, color=blue}
\part{Data Pillar}\label{pt:ekg-mm-c} % C Data Pillar

The \currentname has the following components:

\begin{itemize}[leftmargin=.5in]
  \item [\ref{ch:ekg-mm-c-1}] \nameref{ch:ekg-mm-c-1} % C.1 Data Strategy
  \item [\ref{ch:ekg-mm-c-2}] \nameref{ch:ekg-mm-c-2} % C.2 Data Architecture
  \item [\ref{ch:ekg-mm-c-3}] \nameref{ch:ekg-mm-c-3} % C.3 Data Quality
  \item [\ref{ch:ekg-mm-c-4}] \nameref{ch:ekg-mm-c-4} % C.4 Data Governance
\end{itemize}

%\paragraph{Levels}
%
%\begin{description}[nosep,font=\bfseries]
%
%    \item [1. \ekgmmLevelOneLabel]
%    Core data management capabilities (\gls{operating-model}, inventory, data architecture,
%    \hyperref[sec:ekg-mm-business-vocabularies]{vocabularies, business glossary or terminology},
%    pipeline management, etc.) are being performed.
%    Specific use cases are being implemented with specialist teams for the pilot initiative.
%
%    \item [2. \ekgmmLevelTwoLabel]
%    Critical data elements are prioritized in the ontology.
%    Approach to identity and meaning resolution is established.
%    Use case trees are defined and modeled to capture shared data relationships.
%    The knowledge graph is becoming the central point for integration.
%
%    \item [3. \ekgmmLevelThreeLabel]
%    Inventory is embedded into the \gls{ekg} and linked to governance.
%    Data is expressed as formal ontologies, onboarded into the \gls{ekg} and searchable.
%    Data flows are defined and modeled.
%    The \gls{ekg} is the authoritative source for data.
%
%\end{description}

\chapter{Data Strategy}\label{ch:ekg-mm-c-1}\label{ch:ekg-mm-data-strategy} % C.1 Data Strategy

The \nameref{ch:ekg-mm-c-1} component has the following capabilities:

\begin{itemize}[leftmargin=.5in]
    \ekgmmCapabilityItem{c-1-1}
    \ekgmmCapabilityItem{c-1-2}
\end{itemize}

%
% C.1.1 Goals & Objectives
%
\ekgmmCapability{c-1-1}{data-strategy-goals-and-objectives}{Goals \& Objectives}

%
% C.1.2 Business Case
%
\ekgmmCapability{c-1-2}{ekg-business-case}{Business Case}


\chapter{Data Architecture}\label{ch:ekg-mm-c-2}\label{ch:ekg-mm-data-architecture} % C.2 Data Architecture
\index{data!architecture}

The \nameref{ch:ekg-mm-c-2} component has the following capabilities:

\begin{itemize}[leftmargin=.5in]
  \ekgmmCapabilityItem{c-2-1} % C.2.1 Ontologies
  \ekgmmCapabilityItem{c-2-2} % C.2.2 Business Vocabularies
  \ekgmmCapabilityItem{c-2-3} % C.2.3 Datasets
  \ekgmmCapabilityItem{c-2-4} % C.2.4 Data Mapping
\end{itemize}


%
% C.2.1 Ontologies
%
\ekgmmCapability{c-2-1}{ontologies}{Ontologies}

An ontology is about:

\begin{itemize}
    \item explicit meanings and relationships; the terms used are less important
    \item a combination of definitions in both text and logic
\end{itemize}

An ontology can be the basis of, but is broader than:
\begin{itemize}
    \item a taxonomy\index{taxonomy}
    \item a vocabulary\index{vocabulary}
    \item a data or object model\index{data!model}\index{object model}
    \item a conceptual model\index{conceptual model}
    \item a specific serialization format\index{serialization format}
\end{itemize}

Ontologies can be expressed at different levels of sophistication, with different scopes,
and in a combination of languages.
The basic structures include:

\begin{itemize}[leftmargin=.8in,font=\itshape]
    \item [individual] a representation of a business object or item which is the subject of information to be managed.
                       An individual\index{individual} usually has a unique identity.\index{identity}
                       For example \lstinline|Person X| or \lstinline|Shipment Y|.
                       Many such individuals might represent the same real world object.
    \item [data value] strings, numbers, dates which represent the data.
    \item [property]   a type of data point that may be associated with individuals.
                       An individual, a property\index{property} and a value\,---\,which may be a data value or another
                       individual\,---\, form a triple.\newline
                       For example person \lstinline|X hasBirthDate D| or person \lstinline|X hasMother Y|.
                       Triples whose value is another individual form relationships.
                       Properties may have generalizations, for example \lstinline|hasMother| is a
                       \lstinline|subProperty| of \lstinline|hasParent|.
    \item [class]      a category applied to individuals, that determines what you can do with them,
                       the properties you can expect to see, and the rules that might apply;
                       an individual may be a member of many classes associated;
                       classes may have generalizations.
                       Note that, unlike more traditional approaches, properties are independent of
                       classes\index{class} though they may be used to infer\index{inference} class membership.
                       For example, given the triple \lstinline|X hasMother Y| you may be able to infer that both
                       \lstinline|X| and \lstinline|Y| are members of the class \lstinline|Person|,
                       or at least \lstinline|Animal|.
    \item [ontology]   grouping of the above for management and identification purposes.
\end{itemize}

\ekgmmCapabilityContributionToEnterprise{c-2-1}

\ekgmmCapabilityContributionToEKG{c-2-1}

\begin{maturity-dimensions}

    \item How much of the enterprise data is covered by ontologies?
    \item How well is ontology coverage mapped to business need?\index{business!needs}
    \item To what extent are concepts independent of but mapped to terminology/vocabulary?
    \item Level of sophistication of textual and logic definitions
    \item Level of tooling is available and used
    \item Level of training and trained people
    \item Level of process (including change management), guidelines and standards
    \item Level of modularity and reuse\,---\,internal and external
    \item Extent of examples and tests
    \item Extent of \iindex{traceability} with different logical and physical data models\index{data!physical models}
    
\end{maturity-dimensions}

\ekgmmCapabilitySectionLevels

The following criteria for each level are abbreviated: each item is shorthand for:

\begin{itemize}
    \item documented process
    \item trained participants
    \item implemented process and/or technology
    \item monitoring and improvement
\end{itemize}

\ekgmmCapabilitySectionLevelsOneFive

\begin{level-assessment}{c-2-1}{1}

    \item Minimal ontologies which could be as simple as a list of classes and properties used in graphs
    \item Basic metadata (definition, label) for each class and property
    \item Each individual (in data) has at least one explicit class
    \item Ontology coverage for each use case in scope of the project;
          project includes minimal number of ontologies and classes not justified by a use case
    \item Definitions catalogued and under change management

\end{level-assessment}

\begin{level-assessment}{c-2-1}{2}

    \item Ontologies expressed in a standard ontology language (could be as simple as \gls{ekg:rdf-schema})
    \item Common (shared or mapped) concepts across \gls{ekg} projects
    \item Ability to see ontology usage by use cases, vocabularies and datasets
    \item Namespace scheme established and used for new ontologies in the \gls{ekg}
    \item Ontology guidelines in place and implemented, including common metadata
    \item Documented approach for external ontologies, including selection and adaptation
    \item Annotated example files for documentation and training
    \item Test files based on use cases covering all used ontology elements
    \item Ontology change management includes impact analysis and stakeholder approval
    \item Tooling for ontology diagrams and documentation
    \item Automated basic checking of ontology syntax
    \item Access to at least one trained \gls{ekg:ontologist}

\end{level-assessment}

\begin{level-assessment}{c-2-1}{3}

    \item Modeling of required data and constraints by use case, including for stored and communicated data
    \item Automated validation of ontologies (for guideline compliance, and for logical consistency),
          with results as triples\index{triple}
    \item Automated testing and validation of test data with ontologies (per use case)\index{use case}
    \item Separation of concerns to support enterprise management such as
          bi-temporality\index{temporality}\index{temporality!bi-temporality}, transactions and events
    \item Automated transformation of ontologies to use common serialization and metadata
    \item Automated checking of ontologies against different profiles (e.g. OWL-RL)\index{OWL}
          to check for technology support
    \item Automated checking of ontologies against different best practices
    \item Ontology source changes linked to automated~\nameref{sec:ekg-mm-kg-operations}
          (see~\ref{sec:ekg-mm-kg-operations}) for testing and deployment
    \item Impact analysis identifies ontology breaking changes which require fixes to existing \gls{ekg} data
    \item \Gls{ekg}-wide ontology browsing and searching
    \item \textit{Follow-your-nose}\index{follow-your-nose} UI starting from any
          ontology element \glsfmtshort{uri}\footnote{\label{foot:predicate-iri}see also \gls{ekg:iri:predicate}}
    \item \textit{Follow-your-nose} \gls{api} starting from any
          ontology element \glsfmtshort{uri}\footnoteref{foot:predicate-iri}
    \item Trained ontologist\index{ontology!expert} available to each project (ideally via the \gls{ekg:coe})

\end{level-assessment}

\begin{level-assessment}{c-2-1}{4}

    \item Separation of ontologies from vocabularies, with multiple vocabularies for different communities
          mapped to the same concepts
    \item Ontology architecture management process, including use of patterns and modularity
    \item Generation of logic into business language
    \item Automated fixes to existing \gls{ekg} data in response to ontology breaking changes
    \item Basic ontology metrics and reporting, including usage in data
    \item Generation of ontologies/shapes for external interchange

\end{level-assessment}

\begin{level-assessment}{c-2-1}{5}

    \item Sophisticated ontology metrics\index{ontology!metrics} and reporting, including trends
    \item Matching and differencing of ontologies from different sources
    \item Automated matching of ontologies with vocabularies
    \item Generation of validation code for external interchange
    \item Wizard for developing ontologies from business questions
    \item Inducing of ontologies from instance data

\end{level-assessment}

\input{../../ekg-mm/sections/c-2-data-architecture-2-business-vocabularies.tex}
%
% C.2.3 Datasets
%
\ekgmmCapability{c-2-3}{datasets}{Datasets}

\ekgmmCapabilityContributionToEnterprise{c-2-3}

\ekgmmCapabilityContributionToEKG{c-2-3}

\begin{maturity-dimensions}

  \item Do one or more data inventories exist
  \item Is the inventory based on defined standards (for both meaning and format)
  \item Is defined and in-scope data (both breadth and depth) covered in the inventories
  \item Is the data inventory linked to \glspl{sor} and authorized data distribution points
  \item Is the inventory linked to the business meaning of the data and expressed using standards
  \item Is the creation and maintenance of the data inventory mandated by policy and incorporated
        into the data strategy
  \item Is the quality of the content in the inventory measured, reported to involved stakeholders,
        and used for process enhancement

\end{maturity-dimensions}

\ekgmmCapabilitySectionLevelsOneFive

(concepts are inherited as levels progress)

\begin{level-assessment}{c-2-3}{1}

  \item Sources, data sets, and metadata are onboarded and expressed as formal ontologies
  \item Authoritative (upstream) data sources and (downstream) consumers are documented and verified by users,
        data, and technology
  \item The inventory of applications is defined and selected for graph applications
  \item Requirements and dependencies for each outbound data flow are documented and verified (implementation in the
        graph is not a requirement)
  \item Business glossaries for in-scope use cases are defined and verified in the graph (including a list of
        data sources and datasets)
  \item Policy implemented mandating inventory maintenance and only authorizing the use of data that has been logged
        into the inventory

\end{level-assessment}

\begin{level-assessment}{c-2-3}{2}

  \item \glspl{uct} are defined, standardized, and implemented
  \item All upstream data sources are linked to the authorized systems of record and distribution points
  \item Policy mandating the use of \glspl{sor} and documentation of data flow is implemented
  \item Entitlements have been defined in the graph (governing access to sources of data in the inventory)
  \item Classifications (i.e. criticality, security, privacy) are aligned with the use case tree and captured in the
        knowledge graph
  \item Governance requirements (i.e. use cases, \iindex{accountability}, data sources, data flows, \glspl{sla})
        are modeled and registered into the knowledge graph

\end{level-assessment}

\begin{level-assessment}{c-2-3}{3}

  \item Data inventory is centralized in the graph and linked to governance for defined use cases
  \item Ontologies and data models (including change history and transformations) are registered in the knowledge graph
  \item Entitlements are calculated within the inventory and enforced at the datapoint level
  \item Data Quality\index{data!quality} is automatically calculated (fine-grained with dynamic value resolution) within
        the inventory for each use case
  \item Data retention rules are registered in the graph and automatically enforced
  \item Full audit trail for all upstream and downstream data usage is registered in the graph
  \item Data elements, calculation methods, and \glspl{cde} are linked to individual regulatory requirements

\end{level-assessment}

\begin{level-assessment}{c-2-3}{4}

  \item Connected inventory has been extended to include real-time (transactional) data
  \item Inventory is extended to external suppliers and third parties along the supply chain
  \item The inventory is fully integrated with machine learning to optimize data flow
  \item The “value of data” is calculated and classified within the organizational inventory

\end{level-assessment}


%
% C.2.4 Data Mapping (formerly known as Data Integration & Interoperability)
%
\ekgmmCapability{c-2-4}{data-mapping}{Data Mapping}
\index{data!mapping}

\ekgmmCapabilityContributionToEnterprise{c-2-4}

\ekgmmCapabilityContributionToEKG{c-2-4}

\begin{maturity-dimensions}

  \item Are the data integration activities, their systems, repositories, and connections
        known and tracked
  \item Are data integration activities linked to data inventory, business glossaries, and data models
  \item Are all data integration input and output datasets documented, tracked, and governed
  \item Are there reusable standards and defined business rules for performing data integration
  \item Are data integration patterns, tools, and technologies defined, governed, and used
  \item Has the firm established a central data integration function
        (i.e. integration Center of Excellence) to manage \glsxtrshort{etl} across both
        internal and external data pipelines
\end{maturity-dimensions}

The goal is not always a single source of data - but rather the ability to choose the right authoritative source
for the appropriate context.

\ekgmmCapabilitySectionLevelsOneFive

\begin{level-assessment}{c-2-4}{1}

    \item All data sources are identified and documented for in-scope use cases
    \item Do we know the authoritative source for each data set (should not be able to do integration without
          using approved authoritative source)
    \item Does everyone agree that we are using the right sources (the right source for every context) --
          link to governance
    \item Do we have an approved list of what each source feeds (precise description at the entity level that
          we can get from an approved source\,---\,must know if this is the primary source of the data per the
          use case context).
          For any given entity do I have all the potential sources and for a specific context do I know which
          is authorized.
    \item There is a defined governance process for change management and testing (clear picture of all the
          dependencies for data integration).
          If there are changes to authoritative sources\,---\,do we know the downstream implications (tracked and tested)
    \item Are all \glspl{technology-stack} known and supported by current teams (are all key systems under the
          management and governance of the organization\,---\,should not have ghost systems that are not controlled
          as part of the integration process)
    \item Entitlement policies\index{entitlement!policies} and classification rules (i.e. security, PII,
          business sensitive) are defined and verified
    \item Data Quality\index{data!quality} requirements are defined, documented, and verified

\end{level-assessment}

\begin{level-assessment}{c-2-4}{2}

    \item All information (above) are identified, precisely defined, and on-boarded into the knowledge graph
    \item Able to do datapoint lineage (detailed and complete view of the data integration landscape)
    \item Start making the \gls{ekg} the central point for data integration (the \gls{ekg} becomes the Rosetta stone of
          integration)\,---\,onboard systems, convert to RDF, integrate into \gls{ekg} (defined as the
          data integration strategy\,---\,not necessarily complete)
    \item All data sets that are on-boarded into the \gls{ekg} are coming from the authoritative sources.
          There are no man-in-the-middle systems.
          The goal is direct from the authoritative source to the target system for in-scope use cases.
          Must get the most granular data directly from the authoritative sources.
    \item All datasets are "\glspl{sdd}".
    \item \textbf{Policy}\,---\,All data is obtained from the \gls{ekg} as the authoritative source.
          Do not go directly to the originating source of the data.
    \item Entitlement policies and classification requirements are on-boarded into the \gls{ekg}
    \item \hyperref[sec:ekg-mm-data-quality-business-rules]{Data quality business rules}
          are on-boarded into the \gls{ekg}

\end{level-assessment}

\begin{level-assessment}{c-2-4}{3}

    \item Data is precisely defined (granular level) - expressed as formal ontologies - and on-boarded into the \gls{ekg}
    \item All data flows are modeled, defined, and registered in the \gls{ekg} (full lineage in the \gls{ekg} for all
          in-scope applications)
    \item Start to make the \gls{ekg} the authoritative source (set-up to facilitate decommissioning of systems).
          The \gls{ekg} is structured to become the “new” system for in-scope applications (as soon as all
          connections emanate from the \gls{ekg}).
    \item Entitlements are automatically managed enforced

\end{level-assessment}

\begin{level-assessment}{c-2-4}{4}

    \item \textbf{policy}\,---\,All downstream client systems are using authoritative sources as the only source of information
          for in-scope datasets (\gls{ekg} is in the middle of all data flows)
    \item All “\iindex{cottage industry} systems” are replaced by the \gls{ekg}
          (and \gls{ekg} is able to perform all the requirements of any system it replaced --
          reporting, entitlement, quality control)

\end{level-assessment}


\chapter{Data Quality}\label{ch:ekg-mm-c-3}\label{ch:ekg-mm-data-quality} % C.3 Data Quality
\index{data!quality}

The \nameref{ch:ekg-mm-c-3} component has the following capabilities:

\begin{itemize}[leftmargin=.5in]
  \ekgmmCapabilityItem{c-3-1}
\end{itemize}

%
% C.3.1 Business Rules
%
\ekgmmCapability{c-3-1}{data-quality-business-rules}{Data Quality Business Rules}
\index{data!quality!rules}

\ekgmmCapabilityContributionToEnterprise{c-3-1}

\ekgmmCapabilityContributionToEKG{c-3-1}

\begin{maturity-dimensions}

  \item Are data quality business rules specified, formalized, and expressed in a standardized manner
  \item Is there a clear line of business \iindex{accountability} (owned, funded, and governed) for the
        data quality rules\index{data!quality!rules}
  \item Is there a centrally managed repository of business rules\improve{explain that it does not
        necessarily have to be central as long as it is agreed and enforced at the right scope}
  \item Is there a clearly defined mechanism for logging additions and performing updates
  \item Are the business rules aligned with business applications and traceable to source systems
  \item Are the data quality business rules automated and expressed in a machine-executable format

\end{maturity-dimensions}

\ekgmmCapabilitySectionLevelsOneFive

\begin{level-assessment}{c-3-1}{1}

  \item \hyperref[sec:ekg-mm-data-quality-business-rules]{Data quality business rules} (conditions) have been defined,
        documented and, verified by \glspl{sme} (process for evaluation and acceptance defined)
  \item Business rules are aligned with in-scope use cases and specific user stories
  \item Business rules are standardized and registered into a repository with a defined mechanism for logging
        additions and performing updates

\end{level-assessment}

\begin{level-assessment}{c-3-1}{2}

  \item A defined architecture exists to translate business rules into machine-executable code (some rules will be
        OWL\index{OWL} expressions, some will be SHACL shapes\index{SHACL} and constraints, some will be translated into
        workflow logic)
  \item Business \iindex{provenance} and \iindex{lineage} are traceable across the \iindex{data!supply chain} and
        evaluated against defined business rules (all business rules must be traceable and understandable in
        context\,---\,must understand the purpose and importance of the rule)
  \item Business rules for in-scope use cases are implemented in the \gls{ekg}

\end{level-assessment}

\begin{level-assessment}{c-3-1}{3}

    \item \textbf{Metrics}\,---\,The measurement criteria are defined for data quality business rules (which rules are executed,
          how often, improvement)
    \item \textbf{Performance}\,---\,The value of business rules are related to business concepts (products, financial performance,
          organizational objectives)\,---\,able to trace the core relationship between the business objectives and
          the data quality business rules (correlation between rules and outcomes are known, able to be queried and
          traceable within the \gls{ekg})

\end{level-assessment}

\begin{level-assessment}{c-3-1}{4}

    \item Business rules are combined with \gls{ai} capability for compliance (dynamic optimization of
          business rules)
    \item Model-driven (senior management can begin to optimize business objectives using business rules in
          the \gls{ekg}\,---\,i.e. alignment of business rules with “what if” scenarios)

\end{level-assessment}

\begin{level-assessment}{c-3-1}{5}

    \item All business rules are driven by business objectives (objectives are in the \gls{ekg} with appropriate scorecards)

\end{level-assessment}


\chapter{Data Governance}\label{ch:ekg-mm-c-4}\label{ch:ekg-mm-data-governance} % C.4 Data Governance

The \nameref{ch:ekg-mm-c-4} component has the following capabilities:

\begin{itemize}[leftmargin=.5in]
  \ekgmmCapabilityItem{c-4-1}
  \ekgmmCapabilityItem{c-4-2}
  \ekgmmCapabilityItem{c-4-3}
  \ekgmmCapabilityItem{c-4-4}
\end{itemize}

%
% C.4.1 Data-management Operating Model
%
\ekgmmCapability{c-4-1}{data-management-operating-model}{Data-management \Glsfmttext{operating-model}}

\ekgmmCapabilityContributionToEnterprise{c-4-1}

\ekgmmCapabilityContributionToEKG{c-4-1}

\begin{maturity-dimensions}

  \item Have the underlying \iindex{principles} (and challenges) of data management been established and
        accepted by involved stakeholders
  \item Does the \gls{odm} have authority to enforce adherence to policy
  \item Has the data management funding model been established and implemented
  \item Has the organization identified and recruited stakeholders with sufficient skill sets to
        implement the data management program
  \item Are involved stakeholders held accountable to data management program deliverables
  \item Has the organization defined and validated the \glspl{operating-model},
        and workflows necessary to implement the data management program
  \item Are metrics and KPIs captured and actively used to improve the data management operating process
  \item Is the data management \gls{operating-model} audited for compliance and effectiveness

\end{maturity-dimensions}


\input{../../ekg-mm/sections/c-4-data-governance-2-data-management-policy.tex}
\input{../../ekg-mm/sections/c-4-data-governance-3-classification-management.tex}
%
% C.4.4 Risk and Control Environment
%
\ekgmmCapability{c-4-4}{risk-and-control-environment}{Risk \& Control Environment}

\ekgmmCapabilityContributionToEnterprise{c-4-4}

\ekgmmCapabilityContributionToEKG{c-4-4}

\begin{maturity-dimensions}

  \item Has the organization developed a structured framework outlining the \iindex{principles} of how
        operational risk is identified, assessed, monitored and controlled
  \item Has the operational risk framework been adopted by executive management and verified by
        internal audit
  \item Have oversight mechanisms been adopted to ensure compliance with operational risk
        control policies and governance procedures
  \item Are technology resiliency and continuity plans in place to ensue systems integrity,
        security and availability during mergers, acquisitions and consolidations

\end{maturity-dimensions}



