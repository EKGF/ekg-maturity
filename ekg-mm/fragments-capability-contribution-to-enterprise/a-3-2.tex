%
% A.3.3 Enterprise Architecture -- Contribution to the Enterprise
%
An enterprise architecture is fundamental for execution since it blueprints business processes, data, and
technology to reach the required standardization and integration levels.
Therefore, implementing an Enterprise Architecture may bring several business benefits:

\begin{itemize}
    \item \textbf{Reduced IT costs:} more discipline about standards may help companies increase their
          ROI on IT investments.
          This would be particularly useful to reduce maintenance, integration, and development costs.
    \item \textbf{Increased IT responsiveness:} departing from a stable set of standardization and integration standards
          helps organizations to reduce time during development and also to increase their approval afterwards.
          Additionally, standardization might be very useful to reuse existing business modules for new markets,
          new products and to take advantage of new business opportunities faster.
          In turn, this might result in greater product leadership for companies while offering more strategic agility.
    \item \textbf{Increased management satisfaction:} By departing from an Operating Model and Enterprise Architecture
          defined by the business and for the business, business executives will see more consistent results
          for a lower price and at a faster rate.
          So, this is a by-product of the other benefits.
    \item \textbf{Better operational excellence:} In certain cases, an Enterprise Architecture may result in better
          coordination across end-to-end business processes reducing production costs and waste.
    \item \textbf{More costumer intimacy:} since it is easier to identify customer and integrate their data,
          organizations implementing an Enterprise Architecture may be able to increase the customer value
          and provide consistent results across customer touchpoints.
\end{itemize}

