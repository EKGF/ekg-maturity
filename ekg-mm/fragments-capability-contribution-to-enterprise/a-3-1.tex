%
% A.3.1 Operating Model -- Contribution to the Enterprise
%
The operating model\index{business!operating model} could help organizations define their requirements
and expectations in terms of standardization and integration across business units.
This would be useful to harmonize business capabilities and glossaries\index{glossary}
across companies and their business units.

An operating model gives a company better guidance for developing IT and business process capabilities.
\index{business!process capability}
It also serves as a stable foundation for strategic endeavours such as mergers and acquisitions.
\index{mergers \& acquisitions}
This foundation enables IT to be more proactive in identifying possible strategic opportunities.
In order to define an operating model, management needs to define the role of business process
standardization and integration.
This also requires management to identify the company’s key business processes that create a sustainable
competitive advantage for the company.
As a result, an operating model offers a company the possibility to create and possess reusable capabilities
for long-term growth.
In this context, an operating model could be seen as the main driver of strategy at a corporate or business level.
In addition, an operating model plays a major role in defining the required architecture, practices,
management thinking, policies, and processes as they may be different for each operating model.
In other words, an operating model could be a key driver in the design of separate organization units.
Therefore, a proper translation from an operating model to a data model might be critical for digital transformation.

Furthermore, an operating model may serve as input to the development of an Enterprise Architecture\index{enterprise architecture}
(see~\ref{sec:ekg-mm-enterprise-architecture}) that serves as the organizing logic for business processes
and IT infrastructure based on the company’s standardization and integration requirements.

