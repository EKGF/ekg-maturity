%
% A.3.4 Risk Management -- Contribution to the Enterprise
%
Considering the above on the state, imperatives and opportunities in an enterprise’s risk management practices,
we highlight below a few key focus areas where EKG can help/enable enterprises:

\paragraph*{\textit{Holistic, inter-related, strategy aligned and performance monitoring integrated risk management}}

The topical approaches to risk management is a direct result of the silos of operations in various enterprise units.
As outlined in the earlier sections, through Business Identities,
\glspl{ekg} can enable enterprises to align business strategy with the operating model and the performance management,
thus creating bridges between and aligning the silos of operations to enterprise strategy.
The inter-relationship, in the context of strategic objectives,
can enable risk management to evaluate and monitor risk in a holistic manner.
Qualitative and subjective metrics to more quantitative and objective metrics based risk management practices:
Narrowly focused risk factors, owing to siloed specialisations have led to use of predominantly qualitative metrics
in risk monitoring.
\Glspl{ekg} can help align risk factors with metrics in performance management,
thus helping in use of more effective quantitative objective metrics in risk monitoring.

\paragraph*{\textit{Evaluation and prioritisation of emerging risks through explorative analysis:}}

Specialised areas of risk dominate focus of risk management, while strategic issues require a coordinated action
across multiple operational areas.
In helping to align risk monitoring more closely to the strategic issues,
\glspl{ekg} can enable enterprises to evaluate and prioritise key emerging risks affecting the strategy
through enterprise wide, exploratory scenario-based analysis.
