%
% B.2.1 Ontologies -- Contribution to the EKG
%
Ontologies are the basis for Principle \Ref{sec:ekg-principle-meaning}\,---\,\Nameref{sec:ekg-principle-meaning}:

\somequote{%
    The meaning of every data point must be directly resolvable to a
    machine-readable definition in verifiable formal logic.%
}{EKGF}{https://www.ekgf.org/principles}\index{principles!Meaning}

The link to precise meaning serves to mitigate problems created using the same word with multiple definitions;
and the challenges of expressing conceptual nuance using multiple informal sentences.
In the other direction, from ontology to \iindex{vocabulary}, it should be possible to generate a
\iindex{business glossary} directly from ontologies for a given scope.
Since they should capture the meaning of concepts applicable to an organization, or an even broader ecosystem,
the choice of concepts to include in an \gls{ekg} should be driven by business use cases.
And different overlapping ontologies may be included and mapped to cover different relevant aspects.
Likewise, it should be possible to generate\,---\,and map to\,---\,models for more conventional tools from ontologies,
by applying technology-specific rules.

Semantic modeling also eliminates the problem of hard-coding assumptions about the world into a single data model.
And while multiple ontologies may coexist, they are able to be mapped and connected to each other.
In a mature environment, the data modeling process drives technology implementation,
by defining the detailed data structures and associated \glspl{api}.
These components\,---\,along with functional code\,---\,are included as part of the testing suite
within the \gls{ekg} to facilitate rapid deployment.

Different types of external data models are not needed in \gls{ekg} but can be mapped to or generated.
In fact, physical data modelers are a community with their own vocabulary.

Constraints/shapes for models are applied by context (use case)\,---\,there is no \glsfirst{svot}
for the \gls{ekg} as a whole.
Different ontologies may be used for different contexts and mapped to each other in the underlying knowledge graph.

Ontology elements are linked to by vocabularies (see \ref{sec:ekg-mm-business-vocabularies}) and mapped to other
data models and datasets (\ref{sec:ekg-mm-datasets}) to provide their meaning;
and from Use Cases to provide their scope.
These aspects are covered by those respective capabilities.
