%
% B.2.2 Vocabularies (formerly called Business Terminology (formerly called Business Glossary))
%
\ekgmmCapability{b-2-2}{vocabularies}{Vocabularies}
\index{business!vocabularies}
\index{business!glossary}
\index{business!terminology}

\ekgmmCapabilitySectionContributionToEnterprise

Parts of large organizations, and the others they communicate with, often have their own local business language.
Likewise, operational systems\,---\,including their data warehouses and content repositories\,---\,use
data models and predefined physical schemas that are designed to serve specific purposes.
The ability of the knowledge graph to link this terminology with its actual meaning defined via ontologies at
its most granular (atomic) level\index{granular data} eliminates confusion over meaning when data is validated
or transformed.
Standalone business glossaries provide useful input but the terms need to be reconciled to their
enterprise-level meaning via the \gls{ekg}.

\ekgmmCapabilitySectionDimensions

%\begin{core-questions}
%
%  \item [\thesection.1] Are there multiple (official) business glossaries for lines of business or functions
%  \item [\thesection.2] Are there internal standards for content, format, and use of the business glossaries
%  \item [\thesection.3] Is the business meaning of the data verified by accountable subject matter experts
%  \item [\thesection.4] If the organization maintains multiple business glossaries, how are they reconciled,
%                        verified, maintained and governed
%  \item [\thesection.5] Are the glossaries mapped to all expressions of the data in various systems
%                        (how is this maintained)
%  \item [\thesection.4] Are the business glossaries accessible and searchable from the corporate intranet
%
%\end{core-questions}

\ekgmmCapabilitySectionLevelsOneFive

Core Data Management Requirements: The organization has one (or many) officially designated glossaries,
front-to-back \gls{sme} review and verification, official glossaries are captured in an inventory,
there is a defined policy mandating that all glossaries have verified definitions,
all glossaries are maintained with a defined mechanism for \iindex{change management},
all in-scope domains are covered (and all content is included), mapped to processes and expressions
(lineage and transformation), \iindex{governance} (\iindex{ownership}, \iindex{transformation}, logic,
allowable values, \iindex{audit trail})

\ekgmmscoringlevelOne

\begin{scoring}

  \item Business terms for \glspl{ekg:use-case} are captured and mapped to an \iindex{ontology}
        (possibly as simple labels)

%  \item The organization has one (or many) officially designated glossaries
%  \item Front-to-back SME review and verification
%  \item Official glossaries are captured in an inventory
%  \item There is a defined policy mandating that all glossaries have verified definitions
%  \item All glossaries are maintained with a defined mechanism for change management
%  \item All in-scope domains are covered (and all content is included)
%  \item Mapped to processes and expressions (lineage and transformation)
%  \item Governance (ownership, transformation, logic, allowable values, audit trail)

\end{scoring}

\ekgmmscoringlevelTwo

\begin{scoring}

  \item Relevant terms (for \glspl{ekg:use-case}) are associated with the use case independently of
  their corresponding ontologies

%  \item All concepts for any KG use case are covered by the glossary
%  \item Process exists to add new terms (and updating) into the official glossaries

\end{scoring}

\ekgmmscoringlevelThree

\begin{scoring}

  \item Existing glossaries within the organization, within the scope of supported use cases,
        are mapped to ontologies and imported into the \gls{ekg}
  \item Terms are grouped into vocabularies for reuse in different communities
  \item Natural language used for definitions links to other terms used
  \item Ontology logic is presented to business as natural language
  \item Terms are searchable via the knowledge graph interface

%  \item Relevant terms (for KG use cases) are expressed as RDF and imported into the knowledge graph
%  \item Relevant terms are linked to their corresponding ontologies

\end{scoring}

\ekgmmscoringlevelFour

\begin{scoring}

  \item \gls{ekg} is the authoritative source for all terms, scoped by community, context and use case
  \item The governance processes for all new terms (and changes) are managed directly within the \gls{ekg}
  \item The results of ontology inferencing are presented in business natural language
  \item Data usage (per term) is accessible

%  \item All glossaries within the organization are expressed as RDF and imported into the knowledge graph
%  \item All glossaries are searchable via the knowledge graph interface
%  \item The governance process (including all metadata) for glossaries are on-boarded into the knowledge graph
%        [passive, linked to existing governance tool, read-only]

\end{scoring}

\ekgmmscoringlevelFive

\begin{scoring}

  \item Terminology is used to support \gls{nlp} of unstructured data for the \gls{ekg}

%  \item The \glsxtrlong{ekg:platform} is the authoritative source for all glossaries
%  \item All glossary terms are derived directly from the ontologies
%  \item The governance processes for all new terms (and changes) are managed directly within the \glsxtrshort{ekg}
%  \item Data usage (per glossary term) is monitored and reported

\end{scoring}

