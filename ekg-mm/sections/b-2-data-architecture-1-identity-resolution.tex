\section{Identity Resolution}\label{sec:ekg-mm-b-2-1} % B.2.1 Identity Resolution

Identity resolution is a process of combining multiple identifiers across devices, spreadsheets, repositories, and
platforms into a cohesive profile.
The process includes searching across disparate datasets and analyzing content to find (and resolve) matches based
on available data records and attributes.
Identity resolution is complicated by distinctions in both structure and meaning because various information systems
can vary in quality, completeness, format, and nomenclature.

\ekgmmContextSection

Without standards for describing identity attributes, the process of resolution can be both time-consuming and risky.
The knowledge graph uses formal descriptions (ontologies) of the terms in any domain as well as the relationship
between terms.
The use of ontologies and graph capability not only allows you to resolve multiple identifiers into a single,
harmonized profile\,---\,it allows you to create associations across multiple identities with one master ID.
The knowledge graph becomes the Rosetta stone for identity resolution.

\kgmmcorequestionssection

\begin{core-questions}

  \item [\thesection.1] Is there a standard process for identifying (inventory) and resolving (cross-referencing)
                        identification schemes
  \item [\thesection.2] Does the firm maintain an inventory of identifiers for all strategic data assets
  \item [\thesection.3] Is the identification and meaning of data assets linked to the authorized \glspl{sor}
  \item [\thesection.4] Are versioning and time stamp management consistent across all data assets
  \item [\thesection.5] Is the method of generating and controlling identifiers aligned with governance process
                        (i.e. defined by policy, verified, implemented, and audited)

\end{core-questions}

\subsection*{Common Principles for \glsfmtshort{ekg:iri}}

\begin{itemize}
  \item All URLs have no meaning (opaque and meaningless)
  \item Never have only one central service (authority) to "mint" the URL - everyone can create their own
        \glsfmtshort{ekg:iri}s based on the policy
  \item Goal is to maximize proliferation of \glsfmtshort{ekg:iri}s across the organizational ecosystem
  \item \glsfmtshort{ekg:iri}s are universally unique and permanent (no reuse and no elimination)
  \item \glsfmtshort{ekg:iri} naming conventions (domain and host names) have to be sustainable forever
        (i.e.\ don't use company name because it might change if a merger or acquisition happens)
\end{itemize}

\kgmmscoringsection

\kgmmscoringlevelOne

\begin{scoring}

  \item [concepts] Describe and agree on the identifying properties of the entity (primary key
        attributes) and how IRIs will be constructed
  \item [registration] Establish host and domain names for \glsfmtshort{ekg:iri}s (per \gls{ekg:platform} deployments)
  \item [policy] Align approach with the \glsfmtshort{ekg:iri} policy (add establish policy to policy section) --
        i.e.\ must have one, structure of \glsfmtshort{ekg:iri}/hashing standard, maintain in inventory --
        goal is as short as possible (Short, opaque, meaningless) --
        must decide and agree on standard (format and encoding)
  \item [mapping] All onboarding of in-scope data sources and their \glsxtrshort{etl} pipelines must implement policy
        (generate the right \glsfmtshort{ekg:iri}s).
        Ensure that all \glsfmtshort{ekg:iri}s look the same (create and implement rules/formula for conversion)
  \item [resolution] Define and implement (\href{https://www.w3.org/TR/owl-ref/\#sameAs-def}{owl:sameAs}) process for
        resolving identical objects from multiple datasets into merged \glsfmtshort{ekg:iri} (fully supporting
        multiple \glsfmtshort{ekg:iri}s per object)

\end{scoring}

\kgmmscoringlevelTwo

\begin{scoring}

  \item [strategy] Develop strategy for ALL data (no longer compartmentalized) –- strategy covers all domains across
        the ecosystem
  \item [creation] Implement the model the IRI properties (per concept) into the \glsfmtshort{ekg}\todo{Improve sentence}
  \item Mapping of content to \glsfmtshort{ekg:iri} is standardized using standard software routines
  \item Other systems also generate \glsfmtshort{ekg:iri}s (extend mapping of \glsfmtshort{ekg:iri}s to other systems and
        applications within the organization to prepare for enterprise conversion)
  \item Implement ability to look-up existing \glsfmtshort{ekg:iri}s (reference hub) to allow other systems to
        link/integrate to KG (prerequisite for enterprise-wide KG)
  \item [policy] Policy for the proliferation of \glsfmtshort{ekg:iri}s (if the goal is a standard for creating for
        identification\,---\,is the standard endorsed, mandated and implemented)\todo{Improve sentence}\,---\,includes policy
        for domain names (unchangeable) and commitment from executive management to register the top-level domain
  \item [policy] policy for synchronization of \glsfmtshort{ekg:iri} with policy for objects (ontologies)\todo{unclear}
  \item [policy] policy for creation and control (only authorized systems are allowed) and policy for who is
        eligible to create \glsfmtshort{ekg:iri}s (i.e. a certificate needed to assign and only certain processes are
        eligible for certificates)

\end{scoring}

\kgmmscoringlevelThree

\begin{scoring}

  \item Security procedures (i.e. specific crypto-security certificates) implemented to ensure control over
        \glsfmtshort{ekg:iri} assignment process
  \item Implement standard \glsfmtshort{ekg:iri} class hierarchy structure to enable multi-location linkage (shift from
        URL “flavor” to URN “flavor” as defined in the IRI standard)\todo{needs work}
  \item All inbound data flows (into the \glsfmtshort{ekg}) are using a "Lookup Service" to convert "Strings to Things"
        (i.e. mapping traditional identifiers to \glsfmtshort{ekg:iri}s in a model-driven way)

\end{scoring}
