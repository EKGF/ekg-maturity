%
% B.2.1 Ontologies
%
\ekgmmCapability{b-2-1}{ontologies}{Ontologies}

An ontology is about:

\begin{itemize}
    \item explicit meanings and relationships; the terms used are less important
    \item a combination of definitions in both text and logic
\end{itemize}

An ontology can be the basis of, but is broader than:
\begin{itemize}
    \item a taxonomy\index{taxonomy}
    \item a vocabulary\index{vocabulary}
    \item a data or object model\index{data!model}\index{object model}
    \item a conceptual model\index{conceptual model}
    \item a specific serialization format\index{serialization format}
\end{itemize}

Ontologies can be expressed at different levels of sophistication, with different scopes,
and in a combination of languages.
The basic structures include:

\begin{itemize}[leftmargin=.8in,font=\itshape]
    \item [individual] a representation of a business object or item which is the subject of information to be managed.
                       An individual\index{individual} usually has a unique identity.\index{identity}
                       For example \lstinline|Person X| or \lstinline|Shipment Y|.
                       Many such individuals might represent the same real world object.
    \item [data value] strings, numbers, dates which represent the data.
    \item [property]   a type of data point that may be associated with individuals.
                       An individual, a property\index{property} and a value\,---\,which may be a data value or another
                       individual\,---\, form a triple.\newline
                       For example person \lstinline|X hasBirthDate D| or person \lstinline|X hasMother Y|.
                       Triples whose value is another individual form relationships.
                       Properties may have generalizations, for example \lstinline|hasMother| is a
                       \lstinline|subProperty| of \lstinline|hasParent|.
    \item [class]      a category applied to individuals, that determines what you can do with them,
                       the properties you can expect to see, and the rules that might apply;
                       an individual may be a member of many classes associated;
                       classes may have generalizations.
                       Note that, unlike more traditional approaches, properties are independent of
                       classes\index{class} though they may be used to infer\index{inference} class membership.
                       For example, given the triple \lstinline|X hasMother Y| you may be able to infer that both
                       \lstinline|X| and \lstinline|Y| are members of the class \lstinline|Person|,
                       or at least \lstinline|Animal|.
    \item [ontology]   grouping of the above for management and identification purposes.
\end{itemize}

\ekgmmCapabilityContributionToEnterprise{b-2-1}

\ekgmmCapabilityContributionToEKG{b-2-1}

\begin{maturity-dimensions}

    \item How much of the enterprise data is covered by ontologies?
    \item How well is ontology coverage mapped to business need?\index{business!needs}
    \item To what extent are concepts independent of but mapped to terminology/vocabulary?
    \item Level of sophistication of textual and logic definitions
    \item Level of tooling is available and used
    \item Level of training and trained people
    \item Level of process (including change management), guidelines and standards
    \item Level of modularity and reuse\,---\,internal and external
    \item Extent of examples and tests
    \item Extent of \iindex{traceability} with different logical and physical data models\index{data!physical models}
    
\end{maturity-dimensions}

\ekgmmCapabilitySectionLevels

The following criteria for each level are abbreviated: each item is shorthand for:

\begin{itemize}
    \item documented process
    \item trained participants
    \item implemented process and/or technology
    \item monitoring and improvement
\end{itemize}

\ekgmmCapabilitySectionLevelsOneFive

\begin{level-assessment}{b-2-1}{1}

    \item Minimal ontologies which could be as simple as a list of classes and properties used in graphs
    \item Basic metadata (definition, label) for each class and property
    \item Each individual (in data) has at least one explicit class
    \item Ontology coverage for each use case in scope of the project;
          project includes minimal number of ontologies and classes not justified by a use case
    \item Definitions catalogued and under change management

\end{level-assessment}

\begin{level-assessment}{b-2-1}{2}

    \item Ontologies expressed in a standard ontology language (could be as simple as \gls{ekg:rdf-schema})
    \item Common (shared or mapped) concepts across \gls{ekg} projects
    \item Ability to see ontology usage by use cases, vocabularies and datasets
    \item Namespace scheme established and used for new ontologies in the \gls{ekg}
    \item Ontology guidelines in place and implemented, including common metadata
    \item Documented approach for external ontologies, including selection and adaptation
    \item Annotated example files for documentation and training
    \item Test files based on use cases covering all used ontology elements
    \item Ontology change management includes impact analysis and stakeholder approval
    \item Tooling for ontology diagrams and documentation
    \item Automated basic checking of ontology syntax
    \item Access to at least one trained \gls{ekg:ontologist}

\end{level-assessment}

\begin{level-assessment}{b-2-1}{3}

    \item Modeling of required data and constraints by use case, including for stored and communicated data
    \item Automated validation of ontologies (for guideline compliance, and for logical consistency),
          with results as triples\index{triple}
    \item Automated testing and validation of test data with ontologies (per use case)\index{use case}
    \item Separation of concerns to support enterprise management such as
          bi-temporality\index{temporality}\index{temporality!bi-temporality}, transactions and events
    \item Automated transformation of ontologies to use common serialization and metadata
    \item Automated checking of ontologies against different profiles (e.g. OWL-RL)\index{OWL}
          to check for technology support
    \item Automated checking of ontologies against different best practices
    \item Ontology source changes linked to automated~\nameref{sec:ekg-mm-kg-operations}
          (see~\ref{sec:ekg-mm-kg-operations}) for testing and deployment
    \item Impact analysis identifies ontology breaking changes which require fixes to existing \gls{ekg} data
    \item \Gls{ekg}-wide ontology browsing and searching
    \item \textit{Follow-your-nose}\index{follow-your-nose} UI starting from any
          ontology element \glsfmtshort{uri}\footnote{\label{foot:predicate-iri}see also \gls{ekg:iri:predicate}}
    \item \textit{Follow-your-nose} \gls{api} starting from any
          ontology element \glsfmtshort{uri}\footnoteref{foot:predicate-iri}
    \item Trained ontologist\index{ontology!expert} available to each project (ideally via the \gls{ekg:coe})

\end{level-assessment}

\begin{level-assessment}{b-2-1}{4}

    \item Separation of ontologies from vocabularies, with multiple vocabularies for different communities
          mapped to the same concepts
    \item Ontology architecture management process, including use of patterns and modularity
    \item Generation of logic into business language
    \item Automated fixes to existing \gls{ekg} data in response to ontology breaking changes
    \item Basic ontology metrics and reporting, including usage in data
    \item Generation of ontologies/shapes for external interchange

\end{level-assessment}

\begin{level-assessment}{b-2-1}{5}

    \item Sophisticated ontology metrics\index{ontology!metrics} and reporting, including trends
    \item Matching and differencing of ontologies from different sources
    \item Automated matching of ontologies with vocabularies
    \item Generation of validation code for external interchange
    \item Wizard for developing ontologies from business questions
    \item Inducing of ontologies from instance data

\end{level-assessment}
