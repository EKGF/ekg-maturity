%
% B.2.3 Datasets
%
\ekgmmCapability{b-2-3}{datasets}{Datasets}

\ekgmmCapabilityContributionToEnterprise{b-2-3}

\ekgmmCapabilityContributionToEKG{b-2-3}

\begin{maturity-dimensions}

  \item Do one or more data inventories exist
  \item Is the inventory based on defined standards (for both meaning and format)
  \item Is defined and in-scope data (both breadth and depth) covered in the inventories
  \item Is the data inventory linked to \glspl{sor} and authorized data distribution points
  \item Is the inventory linked to the business meaning of the data and expressed using standards
  \item Is the creation and maintenance of the data inventory mandated by policy and incorporated
        into the data strategy
  \item Is the quality of the content in the inventory measured, reported to involved stakeholders,
        and used for process enhancement

\end{maturity-dimensions}

\ekgmmCapabilitySectionLevelsOneFive

(concepts are inherited as levels progress)

\begin{level-assessment}{b-2-3}{1}

  \item Sources, data sets, and metadata are onboarded and expressed as formal ontologies
  \item Authoritative (upstream) data sources and (downstream) consumers are documented and verified by users,
        data, and technology
  \item The inventory of applications is defined and selected for graph applications
  \item Requirements and dependencies for each outbound data flow are documented and verified (implementation in the
        graph is not a requirement)
  \item Business glossaries for in-scope use cases are defined and verified in the graph (including a list of
        data sources and datasets)
  \item Policy implemented mandating inventory maintenance and only authorizing the use of data that has been logged
        into the inventory

\end{level-assessment}

\begin{level-assessment}{b-2-3}{2}

  \item \glspl{uct} are defined, standardized, and implemented
  \item All upstream data sources are linked to the authorized systems of record and distribution points
  \item Policy mandating the use of \glspl{sor} and documentation of data flow is implemented
  \item Entitlements have been defined in the graph (governing access to sources of data in the inventory)
  \item Classifications (i.e. criticality, security, privacy) are aligned with the use case tree and captured in the
        knowledge graph
  \item Governance requirements (i.e. use cases, \iindex{accountability}, data sources, data flows, \glspl{sla})
        are modeled and registered into the knowledge graph

\end{level-assessment}

\begin{level-assessment}{b-2-3}{3}

  \item Data inventory is centralized in the graph and linked to governance for defined use cases
  \item Ontologies and data models (including change history and transformations) are registered in the knowledge graph
  \item Entitlements are calculated within the inventory and enforced at the datapoint level
  \item Data Quality\index{data!quality} is automatically calculated (fine-grained with dynamic value resolution) within
        the inventory for each use case
  \item Data retention rules are registered in the graph and automatically enforced
  \item Full audit trail for all upstream and downstream data usage is registered in the graph
  \item Data elements, calculation methods, and \glspl{cde} are linked to individual regulatory requirements

\end{level-assessment}

\begin{level-assessment}{b-2-3}{4}

  \item Connected inventory has been extended to include real-time (transactional) data
  \item Inventory is extended to external suppliers and third parties along the supply chain
  \item The inventory is fully integrated with machine learning to optimize data flow
  \item The “value of data” is calculated and classified within the organizational inventory

\end{level-assessment}

