\section{Business Glossary}\label{sec:ekgmm-b-2-4} % B.2.4 Business Glossary

Business data definitions are non-technical descriptions of an organization’s data assets.
The glossary harmonizes the details from applications-specific dictionaries providing transparency into definitions
that helps promote consistency of business terms, identify synonyms and link business content to technical
definitions stored in operational systems.
Agreement on granular meaning by involved stakeholders is essential to managing conceptual nuance across
shared applications and linked processes.

\ekgmmContextSection

Operational systems (including their data warehouses and content repositories) use data models and predefined
physical schemas that are designed to serve specific purposes.
In large organizations, this process can result in multiple business glossaries that reflect the
local “business language” corresponding to these functions.
Mapping these glossaries to each other (and to the consuming applications) is complex and best accomplished
using technology and standards that are specifically designed for this purpose.
The ability of the knowledge graph to link data with metadata ensures that data concepts are defined at their
most granular (atomic) level and eliminates confusion over meaning when data is validated or transformed.

\kgmmcorequestionssection

\begin{core-questions}

  \item [\thesection.1] Are there multiple (official) business glossaries for lines of business or functions
  \item [\thesection.2] Are there internal standards for content, format and use of the business glossaries
  \item [\thesection.3] Is the business meaning of the data verified by accountable subject matter experts
  \item [\thesection.4] If the organization maintains multiple business glossaries, how are they reconciled,
                        verified, maintained and governed
  \item [\thesection.5] Are the glossaries mapped to all expressions of the data in various systems
                        (how is this maintained)
  \item [\thesection.4] Are the business glossaries accessible and searchable from the corporate intranet

\end{core-questions}

\kgmmscoringsection

Core Data Management Requirements: The organization has one (or many) officially designated glossaries,
front-to-back SME review and verification, official glossaries are captured in an inventory,
there is a defined policy mandating that all glossaries have verified definitions,
all glossaries are maintained with a defined mechanism for change management,
all in-scope domains are covered (and all content is included), mapped to processes and expressions
(lineage and transformation), governance (ownership, transformation, logic, allowable values, audit trail)

\kgmmscoringlevelOne

\begin{scoring}

  \item The organization has one (or many) officially designated glossaries
  \item Front-to-back SME review and verification
  \item Official glossaries are captured in an inventory
  \item There is a defined policy mandating that all glossaries have verified definitions
  \item All glossaries are maintained with a defined mechanism for change management
  \item All in-scope domains are covered (and all content is included)
  \item Mapped to processes and expressions (lineage and transformation)
  \item Governance (ownership, transformation, logic, allowable values, audit trail)

\end{scoring}

\kgmmscoringlevelTwo

\begin{scoring}

  \item All concepts for any KG use case are covered by the glossary
  \item Process exists to add new terms (and updating) into the official glossaries

\end{scoring}

\kgmmscoringlevelThree

\begin{scoring}

  \item Relevant terms (for KG use cases) are expressed as RDF and imported into the knowledge graph
  \item Relevant terms are linked to their corresponding ontologies

\end{scoring}

\kgmmscoringlevelFour

\begin{scoring}

  \item All glossaries within the organization are expressed as RDF and imported into the knowledge graph
  \item All glossaries are searchable via the knowledge graph interface
  \item The governance process (including all metadata) for glossaries are on-boarded into the knowledge graph
        [passive, linked to existing governance tool, read-only]

\end{scoring}

\kgmmscoringlevelFive

\begin{scoring}

  \item The \glsxtrlong{ekg:platform} is the authoritative source for all glossaries
  \item All glossary terms are derived directly from the ontologies
  \item The governance processes for all new terms (and changes) are managed directly within the \glsxtrshort{ekg}
  \item Data usage (per glossary term) is monitored and reported

\end{scoring}

