\section{Inventory Management}\label{sec:ekgmm-b-2-3} % B.2.3 Inventory Management

A \iindex{Data Inventory} is a repository of data assets\index{Data Assets} for the organization (also known as a
data catalog\index{Data Catalog} or metadata repository\index{Metadata Repository}).
The inventory contains information about what data exists, where it resides (“data-at-rest”)\index{Data-at-Rest},
responsible parties, upstream/downstream usage, how it moves (“data in motion”)\index{Data-in-Motion}, classifications,
quality designations, availability and other useful metrics.
The contents are catalogued at physical, logical and business levels using defined organizational standards.
Consistently documenting data is the backbone of an effective data management program.
Policies and controls are required to ensure the inventory remains accurate and relevant.

\kgmmekgrationalesection

Linking data inventory to the knowledge graph (and business concepts) ensures precision of meaning at the most
granular level.
Data in the knowledge graph is traceable to all application usage allowing users to find data of interest through
assisted and contextual search and navigation.
The unique ability of the knowledge graph to connect data with metadata enables users to perform flexible queries
in ways that were not previously possible.

\kgmmcorequestionssection

\begin{core-questions}

  \item [\thesection.1] Do (one of more) data inventories exist
  \item [\thesection.2] Is the inventory based on defined standards (for both meaning and format)
  \item [\thesection.3] Is defined and in-scope data (both breadth and depth) covered in the inventories
  \item [\thesection.4] Is the data inventory linked to \glspl{sor} and authorized data distribution points
  \item [\thesection.5] Is the inventory linked to the business meaning of the data and expressed using standards
  \item [\thesection.6] Is the creation and maintenance of the data inventory mandated by policy and incorporated
                        into the data strategy
  \item [\thesection.7] Is the quality of the content in the inventory measured, reported to involved stakeholders
                        and used for process enhancement

\end{core-questions}

\kgmmscoringsection

(concepts are inherited as levels progress)

\kgmmscoringlevelOne

\begin{scoring}

  \item Sources, data sets and metadata are onboarded and expressed as formal ontologies
  \item Authoritative (upstream) data sources and (downstream) consumers are documented and verified by users,
        data and technology
  \item The inventory of applications is defined and selected for graph applications
  \item Requirements and dependencies for each outbound data flow are documented and verified (implementation in the
        graph is not a requirement)
  \item Business glossaries for in-scope use cases are defined and verified in the graph (including a list of
        data sources and datasets)
  \item Policy implemented mandating inventory maintenance and only authorizing the use of data that has been logged
        into the inventory

\end{scoring}

\kgmmscoringlevelTwo

\begin{scoring}

  \item Use case trees\index{Use Case Tree} (taxonomies and dependencies) are defined, standardized and implemented
  \item All upstream data sources are linked to the authorized systems of record and distribution points
  \item Policy mandating the use of \glspl{sor} and documentation of data flow is implemented
  \item Entitlements have been defined in the graph (governing access to sources of data in the inventory)
  \item Classifications (i.e. criticality, security, privacy) are aligned with the use case tree and captured in the
        knowledge graph
  \item Governance requirements (i.e. use cases, accountability, data sources, data flows, SLAs) are modeled and
        registered into the knowledge graph

\end{scoring}

\kgmmscoringlevelThree

\begin{scoring}

  \item Data inventory is centralized in the graph and linked to governance for defined use cases
  \item Ontologies and data models (including change history and transformations) are registered in the knowledge graph
  \item Entitlements are calculated within the inventory and enforced at the datapoint level
  \item Data quality is automatically calculated (fine-grained with dynamic value resolution) within the inventory
        for each use case
  \item Data retention rules are registered in the graph and automatically enforced
  \item Full audit trail for all upstream and downstream data usage is registered in the graph
  \item Data elements, calculation methods and CDEs are linked to individual regulatory requirements

\end{scoring}

\kgmmscoringlevelFour

\begin{scoring}

  \item Connected inventory has been extended to include real-time (transactional) data
  \item Inventory is extended to external suppliers and third parties along the supply chain
  \item The inventory is fully integrated with machine learning to optimize data flow
  \item The “value of data” is calculated and classified within the organizational inventory

\end{scoring}

