%
% B.2.2 Business Vocabularies (formerly called Business Terminology (formerly called Business Glossary))
%
\ekgmmCapability{b-2-2}{business-vocabularies}{Business Vocabularies}
\index{business!vocabularies}
\index{business!glossary}
\index{business!terminology}

\ekgmmCapabilityContributionToEnterprise{b-2-2}

\ekgmmCapabilityContributionToEKG{b-2-2}

\begin{maturity-dimensions}

  \item The organization has one (or many) officially designated glossaries\index{glossary}\index{glossary!designated glossary}
  \item Level of review by \glspl{sme}
  \item Level of change management including \iindex{versioning}, approval, \iindex{audit trail}
  \item Level of communication and ready availability
  \item Terms have a full business definition
  \item Business definitions also use a vocabulary with cross references
  \item Definitions traced to their source with change triggering
  \item Level of examples including edge case instances, which can expose vagueness or ambiguity
  \item All terms mapped to an \iindex{ontology} definition\index{ontology!definition}
  \item Level of review by ontology experts\index{ontology!experts}
  \item Level of review by application/data experts
  \item Level of coverage of use cases\index{use case}
  \item Relationship to official business glossaries\index{business!glossary} or data dictionaries\index{data!dictionary}
  \item Mapping to business models including processes, objectives, \glspl{kpi}
  \item Segregation of vocabularies by community
  \item Reuse of glossaries across communities
  \item Support for different natural languages
  \item Searchability
  \item Completeness of reverse mapping from ontologies to in-scope glossary for terms and definitions
  \item Generation of glossaries from ontologies
  \item Relationship amongst terms (synonyms, abbreviations)
  \item Export to alternative representations e.g. web page, spreadsheet
  \item Managing proper nouns (names of business entities, products etc) and mapping to individuals in the \gls{ekg}
  
\end{maturity-dimensions}

\ekgmmCapabilitySectionLevelsOneFive

\begin{level-assessment}{b-2-2}{1}

  \item Business terms for \glspl{ekg:use-case} are captured and mapped to an \iindex{ontology}
        (possibly as simple labels)

\end{level-assessment}

\begin{level-assessment}{b-2-2}{2}

  \item Relevant terms (for \glspl{ekg:use-case}) are associated with the use case independently of
  their corresponding ontologies

\end{level-assessment}

\begin{level-assessment}{b-2-2}{3}

  \item Existing glossaries within the organization, within the scope of supported use cases,
        are mapped to ontologies and imported into the \gls{ekg}
  \item Terms are grouped into vocabularies for reuse in different communities
  \item Vocabularies include local textual definitions as well as being mapped to ontologies
  \item Community level vocabularies able to import common vocabularies reusable in many communities:
        allow for multiple levels from local to global
  \item Support for synonyms and abbreviations within vocabularies
  \item Vocabularies use general principles of modularity
  \item Ontologies presented to different communities in their own language
  \item Natural language used for definitions links to other terms used
  \item Terms are searchable via the knowledge graph interface
  \item Generate vocabularies from existing data models or ontologies
  \item Ontology logic is presented to business as natural language

\end{level-assessment}

\begin{level-assessment}{b-2-2}{4}

  \item Ontology logic\index{ontology!logic} is presented to business as natural language\index{natural language}
  \item \Gls{ekg} is the \iindex{authoritative source} for all terms, scoped by community, context and \iindex{use case}
  \item The governance processes for all new terms (and changes) are managed directly within the \gls{ekg}
  \item The results of ontology inferencing are presented in business natural language
  \item Use of ontology logic to validate use of vocabulary terms
  \item Management of homonyms and disambiguation of terms\,---\,through being mapped to different concepts
  \item Use of business terms in natural language queries\index{natural language!queries}
  \item Techniques for disambiguation e.g. allowing business users to select the intended meaning
  \item Use of textual analytics to suggest modifications to vocabulary/ontology (lemmatization)
  \item Use of vocabularies to match internal databases or external data sources for linked data,
        and establish \lstinline|sameAs| links\index{\lstinline|sameAs|}

\end{level-assessment}

\begin{level-assessment}{b-2-2}{5}

  \item Terminology is used to support \gls{nlp} of unstructured data for the \gls{ekg}
  \item Use of reference data from KG to support entity resolution

\end{level-assessment}

