\section{Data Management Policy}\label{sec:ekgmm-b-4-2} % B.4.2 Data Management Policy

Data management policies are the mandated requirements to ensure effective management of an organization’s
information assets.
Policies are implemented via operating models, standards and documented procedures.
Data management policies define requirements related to data access (e.g. authorized sources, entitlements,
usage and redistribution), data quality (e.g. metadata, definitions, modeling and business rules),
organizational governance (e.g. accountabilities and alignment with business processes) and
control functions (e.g. retention, security, privacy and change management).

\ekgmmContextSection

Data management policies and standards should be crafted to cover the specifics of the knowledge graph.
Candidate areas for \glsxtrshort{ekg}-related policies include those associated with the management of data catalogs,
the resolution of identity and data meaning, the standards for data modeling and metadata,
the use of \glspl{sor}/authorized distribution points and the control over data entitlements.

\kgmmcorequestionssection

\begin{core-questions}

  \item [\thesection.1] Are policies and standards in alignment with data management strategy
  \item [\thesection.2] Are data management policies documented, complete and operational
  \item [\thesection.3] Are data management policies linked to operational control functions as well as the
                        SDLC process of the organization
  \item [\thesection.4] Have data policies been created in collaboration with (and approved by) business, technology
                        and operational stakeholders
  \item [\thesection.5] Are data management policies aligned with the requirements of the \glsxtrshort{ekg}
  \item [\thesection.6] Have data management policies been approved by executive management and audit

\end{core-questions}

