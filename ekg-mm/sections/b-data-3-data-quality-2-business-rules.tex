\section{Data Quality Business Rules}\label{sec:b-3-2} % B.3.2 Business Rules

Data Quality business rules ensure that the data is fit for its intended purpose.
Subject matter experts specify the criteria used to validate and enforce data integrity.
The criteria are translated into agreed specifications (i.e. business rules) which are later codified for data profiling or measuring conformity.
Data quality business rules can also be embedded into data capture systems to ensure validity at source.

\subsection*{\glsfmtshort{ekg} Rationale}

Data quality is an integrated feature of the knowledge graph.
Data quality rules are linked to structured business vocabularies to ensure that meaning is shared and not obscured by vague terms or cryptic codes.
The logic of business rules and policies are captured and expressed as executable models and consistently enforced across all systems and processes.
These quality constraints (models) allow firms to measure the quality of the data and perform verification across disparate systems.
In the mature \acrshort{ekg}, violations of logic or integrity are identified and prevented before data enters the system.

\subsection*{Core Questions}

\begin{itemize}[leftmargin=.5in]

  \item [\thesection.1] Are data quality business rules specified, formalized and expressed in a standardized manner
  \item [\thesection.2] Is there a clear line of business accountability (owned, funded and governed) for the data quality rules
  \item [\thesection.3] Is there a centrally managed repository of business rules
  \item [\thesection.4] Is there a clearly defined mechanism for logging additions and performing updates
  \item [\thesection.5] Are the business rules aligned with business applications and traceable to source systems
  \item [\thesection.6] Are the data quality business rules automated and expressed in a machine-executable format

\end{itemize}
