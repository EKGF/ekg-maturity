\section[Business Rules]{Data Quality Business Rules}% B.3.2 Business Rules
\label{sec:ekgmm-b-3-2}
\label{sec:ekg-mm-data-quality-business-rules}
\index{data quality}

Data Quality business rules ensure that the data is fit for its intended purpose.
Subject matter experts specify the criteria used to validate and enforce data integrity.
The criteria are translated into agreed specifications (i.e. business rules) which are later codified for
\iindex{data profiling} or measuring conformity.
Data quality business rules can also be embedded into data capture systems to ensure validity at source.

\ekgmmContextSection

Data quality is an integrated feature of the knowledge graph.
Data quality rules are linked to structured business vocabularies\todo{or better: ontologies} to ensure that meaning
is shared and not obscured by vague terms or cryptic codes.
The logic of business rules and policies are\change{is} captured and expressed as executable models and consistently
enforced across all systems and processes.
These quality constraints (models) allow firms to measure the quality of the data and perform verification across
disparate systems.
In the mature \gls{ekg}, violations of logic or integrity are identified and prevented before data
enters the system.

\kgmmcorequestionssection

\begin{core-questions}

  \item [\thesection.1] Are data quality business rules specified, formalized, and expressed in a standardized manner
  \item [\thesection.2] Is there a clear line of business \iindex{accountability} (owned, funded, and governed) for the
                        \iindex{data quality} rules
  \item [\thesection.3] Is there a centrally managed repository of business rules\improve{explain that it does not
                        necessarily have to be central as long as it is agreed and enforced at the right scope}
  \item [\thesection.4] Is there a clearly defined mechanism for logging additions and performing updates
  \item [\thesection.5] Are the business rules aligned with business applications and traceable to source systems
  \item [\thesection.6] Are the data quality business rules automated and expressed in a machine-executable format

\end{core-questions}

\kgmmscoringsection

\kgmmscoringlevelOne

\begin{scoring}

  \item \hyperref[sec:ekg-mm-data-quality-business-rules]{Data quality business rules} (conditions) have been defined,
        documented and, verified by \glspl{sme} (process for evaluation and acceptance defined)
  \item Business rules are aligned with in-scope use cases and specific user stories
  \item Business rules are standardized and registered into a repository with a defined mechanism for logging
        additions and performing updates

\end{scoring}

\kgmmscoringlevelTwo

\begin{scoring}

  \item A defined architecture exists to translate business rules into machine-executable code (some rules will be
        OWL\index{OWL} expressions, some will be SHACL shapes\index{SHACL} and constraints, some will be translated into
        workflow logic)
  \item Business \iindex{provenance} and \iindex{lineage} are traceable across the \iindex{data supply chain} and
        evaluated against defined business rules (all business rules must be traceable and understandable in
        context\,---\,must understand the purpose and importance of the rule)
  \item Business rules for in-scope use cases are implemented in the \gls{ekg}

\end{scoring}

\kgmmscoringlevelThree

\begin{scoring}

    \item [Metrics] The measurement criteria are defined for data quality business rules (which rules are executed,
          how often, improvement)
    \item [Performance] The value of business rules are related to business concepts (products, financial performance,
          organizational objectives)\,---\,able to trace the core relationship between the business objectives and
          the data quality business rules (correlation between rules and outcomes are known, able to be queried and
          traceable within the \gls{ekg})

\end{scoring}

\kgmmscoringlevelFour

\begin{scoring}

    \item Business rules are combined with \gls{ai} capability for compliance (dynamic optimization of
          business rules)
    \item Model-driven (senior management can begin to optimize business objectives using business rules in
          the \gls{ekg}\,---\,i.e. alignment of business rules with “what if” scenarios)

\end{scoring}

\kgmmscoringlevelFive

\begin{scoring}

    \item All business rules are driven by business objectives (objectives are in the \gls{ekg} with appropriate scorecards)

\end{scoring}
