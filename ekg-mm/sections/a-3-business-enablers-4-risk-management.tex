%
% A.3.4 Risk Management
%
\ekgmmCapability{a-3-4}{risk-management}{Risk Management}
\index{risk}

Risk Management has traditionally been an integral, and sometimes implicit, part of an Enterprise’s
Business Governance practices.\index{business governance}\index{governance!business}
Though there are guidelines for business governance in every country, not all are formulated as regulations;
some are only generally accepted norms of conduct\autocite{integration_of_risk_management}.
Some examples of well-known governance and risk frameworks are

\begin{itemize}
    \item The BASEL Accord framework (for financial services organizations) used by regulators globally for their
          specific jurisdiction directives covering management practices around operational, credit and market risks
    \item COSO, a private sector initiative (sector agnostic) dedicated to helping organizations improve
          performance by developing thought leadership that enhances internal control, risk management,
          governance and fraud deterrence
\end{itemize}

Enterprises use a variety of internal control systems, both formally and informally,
to monitor their governance mechanisms.
The COSO model\autocite{erm_integrating_with_strategy_and_performance},
for example, defines internal control as
\textit{...designed to provide reasonable assurance of the achievement of objectives in the following categories:}

\begin{itemize}
    \item \textit{Operational Effectiveness and Efficiency}
    \item \textit{Financial Reporting Reliability}
    \item \textit{Applicable Laws and Regulations Compliance}
\end{itemize}

A recent study\autocite{erm_integrating_with_strategy_and_performance}
on the practices around \gls{erm} has highlighted the following:

\begin{itemize}
    \item Effective risk management is a priority among boards of directors;
            organizations are facing pressures from a number of stakeholders to provide more risk information and
            business leaders want to be better prepared when unexpected risk events emerge to avoid being surprised
    \item Most organizations continue to struggle to integrate their risk management and strategic planning efforts;
            risk tolerances are not formally articulated as part of their strategic management activities
    \item There are a number of impediments to advancing an organization’s risk management processes,
            with the belief that "risks are managed in other ways besides \gls{erm}" dominating the list
    \item There is a heavy emphasis on risks related to technology, legal/compliance and financial issues,
            with ERM processes less focused on emerging strategic/market/industry risks or risk related to reputation
    \item There are opportunities to reposition an entity's risk management process to ensure risk insights
            generated are focused on the most important strategic issues
\end{itemize}

\ekgmmCapabilityContributionToEnterprise{a-3-4}

\ekgmmCapabilityContributionToEKG{a-3-4}

PREVIOUS CONTENT SHOWN BELOW: WHAT TO DO?

Risk is the probability that an event can occur either negative or positive.
Negative risk can be realized into events that can be an issue,
defect or problem that happened while operating the business.
Positive events are opportunities.\improve{JAG>Are negative events not opportunities either? Opportunities to improve?}
If opportunities are not anticipated or planned, then there are impacts to the business as well.
Unanticipated opportunities might not be able to be taken advantage of if the business cannot react quickly enough.

Example types of risk include the probability of:
\begin{itemize}
    \item insufficient staff to produce or support increased product demand,
    \item wrong product strategy leading to reduced purchases impacting revenue,
    \item disruptive political events in a key market geography,
    \item interest rate change impacting liquidity,
    \item facility damage impacting operational capability,
    \item supply chain disruptions impacting ability to produce product,
    \item data security breaches causing harm to the organization and customers, and many more.
\end{itemize}

Common categories are:
\begin{itemize}
    \item Strategic risk,\index{risk!strategic risk}
    \item Operational risk (people, process, systems, external events),\index{risk!operational risk}
    \item Financial risk, and\index{risk!financial risk}
    \item Reputation risk.\index{risk!reputational risk}
\end{itemize}

Risks are mitigated through process change to a lower level of risk that can be accepted.
Accepted risks should ideally have a documented \iindex{reaction plan}, which is a set of processes and procedures.
These are not \glsfirst{bau} processes, but are executed in response to an issue, defect or problem, or opportunity.
It is a best practice to define your reaction plans before an event occurs.

A reaction plan can include:
\begin{itemize}
    \item how to obtain more material from existing or new suppliers,
    \item increase in staff to service an opportunity,
    \item increase in product production equipment,
    \item expanded or new facilities,
    \item expansion of delivery channels and additional logistic complexity,
    \item alternate site plans in case of facility damage, and more.
\end{itemize}

Products and processes have an inherent level of risk\,---\,the risk that exists if no risk mitigation control
processes are implemented.
Aircraft manufacturing/operations is inherently more risky than creation and use of board games.
If the value of inherent risk is high, then more complex and costly risk mitigation is justified,
and might not be wise to cut back for cost savings.
Inherent risk is mitigated by process change to a new lower level of residual risk that can be accepted,
or plan to mitigate to a level of risk that is acceptable.
The value of risk can be calculated by taking the probability of an event occurring,
multiplied by the expected monetary value impacted (up or down) if that risk is realized.
Risks should be mitigated to a level of accepted risk that fits a cost/benefit balance.
The value of risk can be used to justify investment in process change and tools.

\ekgmmCapabilitySectionContributionToEnterprise

For processes that have risks modeled that are an accepted risk by the business,
reaction processes can be modeled in the \gls{ekg} and associated to that risk.
If that risk materializes into an actual event,
then the reaction process and associated procedures can be quickly found and carried out.
An \gls{ekg} can hold the knowledge to quickly determine impacts when an issue happens by looking at
associated risks aligned to the processes and resources for a specific risk.

\ekgmmCapabilitySectionContributionToEKG

The \gls{ekg} can use risk management processes to mitigate risk that the \gls{ekg}
stays up to date and relevant,
is developed aligned with the scope and purpose of the \gls{ekg}\footnote{%
    i.e. is aligned with the scope and purpose of each use case that the \gls{ekg:platform} serves%
},
has input from multiple points of view across the business to prevent skewing towards business silo views and use.

These risks to the \gls{ekg} are mitigated with
overall governance processes and standards for managing the \gls{ekg}\footnote{%
    by applying a methodological approach to work methods, development approach, and best practices,
    as part of one or more \glspl{ekg:coe}. See also \gls{ekg:method}%
},
data governance processes to ensure the \gls{ekg} model and data continues to support needed knowledge,
and that the knowledge is used for business decisions or knowledge reuse.
