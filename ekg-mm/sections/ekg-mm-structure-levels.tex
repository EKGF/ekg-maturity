\section{The five maturity levels}\label{sec:the-five-maturity-levels}

This document will refer\,---\,for each capability\,---\,to the five maturity levels.

\subsection{Level 1: \ekgmmLevelOneLabel}

The domain of internal \glspl{poc}, pilots or\,---\,ideally\,---\,"\glspl{lighthouse-project}".
The focus is usually on specific targeted baseline use cases constructed using isolated ontologies.
The champions are visionaries who have assembled a specialist team for implementation\,---\,possibly the
start of the \glsfirst{ekg:coe}.
Funding is likely to be project-based and designed to demonstrate capabilities.

\begin{itemize}[leftmargin=1in,font=\bfseries]

    \item[Business]     Stakeholders recognize business opportunities in scaling and amplifying capabilities
                        through \glspl{ekg}.
                        The first internal champion is seeking to socialize strategic business cases,
                        supports innovation, and is willing to take on the disruption challenge.
    \item[Organization] Champions are internal visionaries who have assembled a specialist team for implementation.
                        The pilot is sanctioned and funded.
                        Knowledge acceleration is being addressed.
                        Overall organizational support is emerging.
    \item[Data]         Core data management capabilities (\gls{operating-model}, inventory, data architecture,
                        \hyperref[sec:ekg-mm-business-vocabularies]{vocabularies, business glossary or terminology},
                        pipeline management, etc.) are being performed.
                        Specific use cases are being implemented with specialist teams for the pilot initiative.
    \item[Technology]   Technology strategy is focused on experimentation and innovation.
                        Manual data transformation and targeted ETL is underway for the pilot.
                        Limited infrastructure and dedicated efforts to build initial knowledge graph components.
\end{itemize}

\subsection{Level 2: \ekgmmLevelTwoLabel}

The domain of parallel knowledge graph activities, implementing multiple (related) use cases on the same platform.
The organization is creating reusable architecture\,---\,i.e. "the \gls{ekg:platform}"\,---\,based
on \citefield{ekgprinciples}{title}.\index{principles}
The \glsfirst{ekg:coe} is created.
Funding is likely to be at the \gls{lob} level and starts to (partly) come from \gls{bau} budgets.

\begin{itemize}[leftmargin=1in,font=\bfseries]

    \item[Business]     Stakeholders adopt a “knowledge-centric” mindset in their tactics to strengthen focus on
                        strategic business value.
                        Management elevates the knowledge graph as an organizational and funding priority.
    \item[Organization] \Gls{operating-model} of collaboration is implemented to support the knowledge graph.
                        The Center of Excellence and DataOps environment is initiated.
                        Budget and implementation strategy are based on agile and synchronized with the
                        use case tree methodology.
    \item[Data]         Critical data elements are prioritized in the ontology.
                        Approach to identity and meaning resolution is established.
                        Use case trees are defined and modeled to capture shared data relationships.
                        The knowledge graph is becoming the central point for integration.
    \item[Technology]   Reusable architecture based on \citefield{ekgprinciples}{title}.\index{principles}
                        Core software development design approaches are being established and incorporated
                        into strategy.
                        CTO focuses on extending pilot initiatives for additional leverage.
\end{itemize}

\subsection{Level 3: \ekgmmLevelThreeLabel}

The establishment of a secure, scalable and resilient \gls{ekg:platform} for business-critical strategic use cases.
Resources for the design and build of operational systems are defined and coordinated.
The knowledge graph is now really an \myuline{Enterprise} Knowledge Graph (EKG) that serves as the
semantic data fabric\index{data fabric} for the organization.
Ownership, governance, and funding are managed at the enterprise level and coordinated by the \gls{ekg:coe} that
oversees the full life-cycle of use cases from inception to deployment and beyond.
Long-term "operate \& optimize" processes are in place.

\begin{itemize}[leftmargin=1in,font=\bfseries]

    \item[Business]     Strong collaboration between various business and support units to prioritize
                        strategic business cases.
    \item[Organization] The \gls{ekg} is recognized as a core service for the enterprise.
                        Enterprise-wide ownership and funding processes are operational.
                        The \gls{ekg:coe} is a stand-alone \gls{bau} department.
    \item[Data]         Inventory is embedded into the \gls{ekg} and linked to governance.
                        Data is expressed as formal ontologies, onboarded into the \gls{ekg} and searchable.
                        Data flows are defined and modeled.
                        The \gls{ekg} is the authoritative source for data.
    \item[Technology]   Commitment to the \gls{ekg} as the strategic infrastructure for the organization.
                        \Gls{iac} and \iindex{continuous deployment} are adopted and implemented.
                        Cloud architecture defined for elasticity.
                        Data point\index{data point} security and authentication processes are implemented.
\end{itemize}

\subsection{Level 4: \ekgmmLevelFourLabel}

The \gls{ekg} is understood as strategic infrastructure\,---\,as an operational utility\,---\,for the organization
and the authoritative source for most data (except for data that originates from core legacy systems).
It supports structural application rationalization\index{rationalization}, high-level and high-quality \gls{ai}
that can take over many tasks from humans and process automation.
Strategic funding is based on the vision of executive management and fully embraced by the Board of Directors.
All core data management capabilities have been achieved.

\subsection{Level 5: \ekgmmLevelFiveLabel}

The \gls{ekg} is central to systems and business processes.
It has been fully integrated into both internal operations and external supply chain partners.
Workflows and approval steps are fully automated.
Entitlements and access rights are controlled by the \gls{ekg}.
Inference and reasoning capabilities are used for advanced \gls{ai} use cases.
