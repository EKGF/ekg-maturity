\section{Risk \& Control Environment}\label{sec:ekgmm-b-4-6} % B.4.6 Risk and Control Framework

Operational risk is the result of inadequate or failed internal procedures,
systems or policies that are not driven by external forces such as economic, political or systemic events.
Understanding how data flows into decision making and operational processes is one of the key components of
operational risk mitigation.
Many firms are working to establish strong control environments to identify, approve and monitor operational risk.
Control processes include a combination of strong governance, standards for ensuring interoperability and
quality assurance across the full data lifecycle.

\ekgmmContextSection

Control process in the \glsxtrshort{ekg} environment are managed as a structured set of executable business rules
and automatically enforced across all applications.
The meaning of the data is structurally validated at the point of data capture to prevent bad data from
entering the system\improve{avoid "the system", use "EKG"} at ingestion.
In the knowledge graph, data and metadata are fully integrated to ensure reusability across systems
and operational processes.
The \glsxtrshort{ekg} is able to map data flow including all dependencies and transformations to verify that
data is obtained from authorized \glspl{sor}, identified according to access rights and
aligned with intended usage.

\kgmmcorequestionssection

\begin{core-questions}

  \item [\thesection.1] Has the organization developed a structured framework outlining the principles of how
                        operational risk is identified, assessed, monitored and controlled
  \item [\thesection.2] Has the operational risk framework been adopted by executive management and verified by
                        internal audit
  \item [\thesection.3] Have oversight mechanisms been adopted to ensure compliance with operational risk
                        control policies and governance procedures
  \item [\thesection.4] Are technology resiliency and continuity plans in place to ensue systems integrity,
                        security and availability during mergers, acquisitions and consolidations

\end{core-questions}

