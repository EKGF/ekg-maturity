%
% A.3.3 Performance Management
%
\ekgmmCapability{a-3-3}{performance-management}{Performance Management}

\begin{wrapfigure}[3]{r}{0.5\textwidth}
    \vspace{-42pt}
    \begin{center}
        \begin{tcolorbox}[colback=secondary!5,colframe=secondary!60,left=2pt,right=2pt]
            \itshape\large\enquote{%
                Making a distinction between strategy and execution can do great damage to a corporation.%
            }%
            \begin{flushright}\textcite{the_execution_trap}\end{flushright}%
        \end{tcolorbox}
    \end{center}
\end{wrapfigure}

Performance Management at Enterprises, for various reasons of convenience and expediency, 
have operated at (at-least!) three different levels:

\begin{enumerate}
    \item \textbf{Periodic reporting cycles used by the executive leadership in evaluating progress on strategic goals}
    \begin{itemize}
        \item These have been traditionally dominated by lag indicators such as receivables, payables, market share with
              intuition and judgment driven evaluation of lead indicators such as outages, downtimes, revenue pipeline,
              attrition etc.
              The Business Intelligence movement and the rapid digitalisation of key activities has improved the
              ability of most enterprises to collate, aggregate and report on various metrics of interest in such
              review cycles.
    \end{itemize}
    \item \textbf{Operations and Delivery management reporting used by front-line and operations staff to identify,
        analyse and fix problems and deviations}
    \begin{itemize}
        \item Improvements in the planning and scheduling activities, through multiple generations of resource planning
              and workflow management, have helped most enterprises to rapidly configure chains of activities,
              optimise resource capital and monitor in near real-time the fulfilment and deviations in operations.
    \end{itemize}
    \item \textbf{Personnel evaluation and reporting towards Role management for career and aspirations management of
          staff}
    \begin{itemize}
        \item Performance appraisal has received a lot of attention in recent years, with shifts from intuition and
              heuristics based once-a-year activity to statistical profiling mechanisms and of late,
              continuous conversations focused on metrics such as productivity, retention and capability development.
    \end{itemize}
\end{enumerate}

While the Operating Model can address many of the disconnects that arise between these levels,
Enterprises still struggle with bridging the gap in aligning performance management for reasons such as:

\begin{itemize}
    \item \textbf{Differences in focus between Governance and Operating procedures:}
    \begin{itemize}
        \item Metrics in governance focus on enforcement whereas metrics in operations focus on output and progress.
        The feedback mechanisms which aim to align the two are dominated by subjective assessments resulting in
        high exception workloads.
    \end{itemize}
    \item \textbf{Fulfilment focused integration leaving significant silos between information technology and
        Operations}
    \begin{itemize}
        \item Most use of information technology is focused on information flows and aggregation whereas operations
        technology is focused on individual units of activity. The relative prioritisation is dominated by
        committee-based subjective assessments resulting in high backlogs,
        long wish lists of technology enablement and proliferation of tactical approaches in operations.
    \end{itemize}
    \item \textbf{Unclear alignment of incentives, especially between Operations and Role performance}
    \begin{itemize}
        \item While performance appraisals and role management have become increasingly automated in evidence
        management with links to operations and competency development, work assignment and activities seldom
        remain aligned to start-of-period goals, leading to subjective adjustments through mechanisms such as
        continuous conversations.
        The key challenge of mapping enterprise strategy to role specific goals remains mostly a
        subjective and judgmental exercise.
    \end{itemize}
\end{itemize}

\ekgmmCapabilityContributionToEnterprise{a-3-3}

\ekgmmCapabilityContributionToEKG{a-3-3}


