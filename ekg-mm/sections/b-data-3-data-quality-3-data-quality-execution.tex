\section[Execution]{Data Quality Execution}% \thesection.3 Data Quality Execution
\label{sec:ekgmm-b-3-3}
\label{sec:ekg-mm-data-quality-execution}
\index{data quality}

Implementation of the data quality program is a control process mechanism for organizations.
It is driven by the execution of business rules at the point of data capture (validation).
key components include data profiling, root cause analysis, data remediation and issue management.
The primary objective of the data quality program is to accurately and consistently represent the concepts and
values needed to meet the needs of consumers.
A well-orchestrated data quality program requires communication and operational coordination across the
organizational landscape.

\ekgmmContextSection

Data in the knowledge graph is aligned to precise meaning so that errors and definitional conflicts are
verified at source before they are introduced into operational systems.
In the knowledge graph data, rules and metadata are co-resident and reusable without impedance mismatch.
Quality execution is rules-based and unhooked from both schemas and data models that are tailored to
specific applications.
Metadata is embedded into the content so that users always understand what the data represents as it moves across
organizational boundaries.

\kgmmcorequestionssection

\begin{core-questions}

  \item [\thesection.1] Is there a defined process for executing data quality across linked applications
  \item [\thesection.1] Is there a link between business rules and execution
  \item [\thesection.1] Is the governance structure (and funding mechanisms) for managing data quality issues
                        operational
  \item [\thesection.1] Have service level and data sharing agreements been defined and verified by consumers
  \item [\thesection.1] Is there a defined (and repeatable) process for tracing data quality issues to
                        systems of record and downstream applications
  \item [\thesection.1] Is data quality remediation coordinated across the data production and consumption process
  \item [\thesection.1] Are data quality metrics and scorecards captured and reported to involved stakeholders

\end{core-questions}

