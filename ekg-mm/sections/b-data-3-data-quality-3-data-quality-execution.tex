\section{Data Quality Execution}\label{sec:b-3-3} % \thesection.3 Data Quality Execution

Implementation of the data quality program is a control process mechanism for organizations.
Executing business Rules effectively and consistently is the goal, to do this, governed business Rules must be turned into executable logic, and interjected at capture time (validation) and/or on the population of existing data; in which case any violations should be reported to the owning parties for remediation.
Data Quality Issue Management is also a key part of execution agenda. key components include data profiling, root cause analysis and data remediation.
The primary objective of the data quality program is to accurately and consistently represent the concepts and values needed to meet the needs of consumers.
A well-orchestrated data quality program requires communication and operational coordination across the organizational landscape.

\subsection*{\glsfmtshort{ekg} Rationale}

Data in the knowledge graph is aligned to precise meaning so that errors and definitional conflicts are verified at source before they are introduced into operational systems.
In the knowlege graph, the data, the rules and the metadata are co-resident and therefore reusable without impedance mismatch.
Everything is available including the results and the data quality issue management process: root cause is traversable, analysis of sources and errors and impact analysis of root cause failures.
Quality execution is rules-based and unhooked from both schemas and data models that are tailored to specific applications.
Metadata is embedded in the content so that users always understand what the data represents as it moves across organizational boundaries.

\subsection*{Core Questions}

\begin{itemize}[leftmargin=.5in]

  \item [\thesection.1] Is there a defined process for executing data quality rules across linked applications, and gathering the results in a standardized manner (JEP>is this 2 questions - I think so!)
  \item [\thesection.2] Are the executed rules linked back to the business rules.
  \item [\thesection.3] Is the governance structure (and funding mechanisms) for managing data quality issues operational
  \item [\thesection.4] Have service level and data sharing agreements been defined and verified by consumers
  \item [\thesection.5] Is there a defined and repeatable process for tracing data quality issues to systems of record, conversely when a root cause is found is there a way to find all downstream dependencies?
  \item [\thesection.6] Is data quality remediation coordinated across the data production and consumption process
  \item [\thesection.7] Are data quality metrics and scorecards captured and reported to involved stakeholders

\end{itemize}
