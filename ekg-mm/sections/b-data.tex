\bookmarksetup{bold=true, color=blue}
\part{Data}\label{pt:ekgmm-b} % B Data

The pillar "Data" has the following components:

\begin{itemize}[leftmargin=.5in]
  \item [\ref{ch:ekgmm-b-1}] \nameref{ch:ekgmm-b-1}
  \item [\ref{ch:ekgmm-b-2}] \nameref{ch:ekgmm-b-2}
  \item [\ref{ch:ekgmm-b-3}] \nameref{ch:ekgmm-b-3}
  \item [\ref{ch:ekgmm-b-4}] \nameref{ch:ekgmm-b-4}
\end{itemize}

\paragraph{Levels}

\begin{description}[nosep,font=\bfseries]

    \item [1. \glsfmtshort{ekg} Initiation, \glsfmtshort{mvp}]
    Core data management capabilities (operating model, inventory, data architecture, business
    glossary, pipeline management, etc.) are being performed.
    Specific use cases are being implemented with specialist teams for the pilot initiative.
    
    \item [2. Extensible Platform (reusable components)]
    Critical data elements are prioritized in the ontology.
    Approach to identity and meaning resolution is established.
    Use case trees are defined and modeled to capture shared data relationships.
    The knowledge graph is becoming the central point for integration.

    \item [3. Enterprise Ready]
    Inventory is embedded into the \gls{ekg} and linked to governance.
    Data is expressed as formal ontologies, onboarded into the \gls{ekg} and searchable.
    Data flows are defined and modeled.
    The \gls{ekg} is the authoritative source for data.

\end{description}

\chapter{Data Strategy}\label{ch:ekgmm-b-1} % B.1 Data Strategy

We welcome your input here.

The \nameref{ch:ekgmm-b-1} component has the following capabilities:

\begin{itemize}[leftmargin=.5in]
    \item[B.1.1] Goals and Objectives -- \gls{ekg} as the semantic data fabric for the organization
    \item[B.2.2] Business Case -- Business rationale, justification and \gls{roi} of the \gls{ekg}
\end{itemize}


\chapter{Data Architecture}\label{ch:ekgmm-b-2} % B.2 Data Architecture

The \nameref{ch:ekgmm-b-2} component has the following capabilities:

\begin{itemize}[leftmargin=.5in]
  \item [\ref{sec:ekgmm-b-2-1}] \nameref{sec:ekgmm-b-2-1} -- Standards for ensuring content is identified and resolvable
  \item [\ref{sec:ekgmm-b-2-2}] \nameref{sec:ekgmm-b-2-2} -- Standards and tools for ensuring the shared meaning of data
  \item [\ref{sec:ekgmm-b-2-3}] \nameref{sec:ekgmm-b-2-3} -- Cataloguing content, sources, provenance, and lineage in the \gls{ekg}
  \item [\ref{sec:ekgmm-b-2-4}] \nameref{sec:ekgmm-b-2-4} -- Process for ensuring that glossaries are linked to ontologies
  \item [\ref{sec:ekgmm-b-2-5}] \nameref{sec:ekgmm-b-2-5} -- Patterns and approaches for moving and transforming data
\end{itemize}


\section{Identity Resolution}\label{sec:ekgmm-b-2-1} % B.2.1 Identity Resolution

Identity resolution is a process of combining multiple identifiers across devices, spreadsheets, repositories and
platforms into a cohesive profile.
The process includes searching across disparate datasets and analyzing content to find (and resolve) matches based
on available data records and attributes.
Identity resolution is complicated by distinctions in both structure and meaning because various information systems
can vary in quality, completeness, format and nomenclature.

\kgmmekgrationalesection

Without standards for describing identity attributes, the process of resolution can be both time-consuming and risky.
The knowledge graph uses formal descriptions (ontologies) of the terms in any domain as well as the relationship
between terms.
The use of ontologies and graph capability not only allows you to resolve multiple identifiers into a single,
harmonized profile -- it allows you to create associations across multiple identities with one master ID.
The knowledge graph becomes the Rosetta stone for identity resolution.

\kgmmcorequestionssection

\begin{itemize}[leftmargin=.5in]

  \item [\thesection.1] Is there a standard process for identifying (inventory) and resolving (cross-referencing)
                        identification schemes
  \item [\thesection.2] Does the firm maintain an inventory of identifiers for all strategic data assets
  \item [\thesection.3] Is the identification and meaning of data assets linked to the authorized \glsxtrfullpl{sor}
  \item [\thesection.4] Are versioning and time stamp management consistent across all data assets
  \item [\thesection.5] Is the method of generating and controlling identifiers aligned with governance process
                        (i.e. defined by policy, verified, implemented and audited)

\end{itemize}

\subsection*{Common Principles for \glsfmtshort{kgiri}}

\begin{itemize}
  \item All URLs have no meaning (opaque and meaningless)
  \item Never have only one central service (authority) to “mint” the URL - everyone can create their own
        \glsfmtshort{kgiri}s based on the policy
  \item Goal is to maximize proliferation of \glsfmtshort{kgiri}s across the organizational ecosystem
  \item \glsfmtshort{kgiri}s are universally unique and permanent (no reuse and no elimination)
  \item \glsfmtshort{kgiri} naming conventions (domain and host names) have to be sustainable forever
        (i.e. don’t use company name because it might change if a merger or acquisition happens)
\end{itemize}

\kgmmscoringsection

\kgmmscoringlevelOne

\begin{itemize}[leftmargin=1.5in]

  \item [concepts] Describe and agree on the identifying properties of the entity (primary key
        attributes) and how IRIs will be constructed
  \item [registration] Establish host and domain names for \glsfmtshort{kgiri}s (per EKG platform deployments)
  \item [policy] Align approach with the \glsfmtshort{kgiri} policy (add establish policy to policy section) -- i.e.
        must have one, structure of \glsfmtshort{kgiri}/hashing standard, maintain in inventory –- goal is as short
        as possible (Short, opaque, meaningless) –- must decide and agree on standard (format and encoding)
  \item [mapping] All onboarding of in-scope data sources and their \glsxtrshort{etl} pipelines must implement policy
        (generate the right \glsfmtshort{kgiri}s).
        Ensure that all \glsfmtshort{kgiri}s look the same (create and implement rules/formula for conversion)
  \item [resolution] Define and implement (\href{https://www.w3.org/TR/owl-ref/#sameAs-def}{owl:sameAs}) process for
        resolving identical objects from multiple datasets into merged \glsfmtshort{kgiri} (fully supporting
        multiple \glsfmtshort{kgiri}s per object)

\end{itemize}

\kgmmscoringlevelTwo

\begin{itemize}[leftmargin=1.5in]

  \item [strategy] Develop strategy for ALL data (no longer compartmentalized) –- strategy covers all domains across
        the ecosystem
  \item [creation] Implement the model the IRI properties (per concept) into the \glsfmtshort{ekg}\todo{Improve sentence}
  \item Mapping of content to \glsfmtshort{kgiri} is standardized using standard software routines
  \item Other systems also generate \glsfmtshort{kgiri}s (extend mapping of \glsfmtshort{kgiri}s to other systems and
        applications within the organization to prepare for enterprise conversion)
  \item Implement ability to look-up existing \glsfmtshort{kgiri}s (reference hub) to allow other systems to
        link/integrate to KG (prerequisite for enterprise-wide KG)
  \item [policy] Policy for the proliferation of \glsfmtshort{kgiri}s (if the goal is a standard for creating for
        identification -- is the standard endorsed, mandated and implemented)\todo{Improve sentence} -– includes policy
        for domain names (unchangeable) and commitment from executive management to register the top-level domain
  \item [policy] policy for synchronization of \glsfmtshort{kgiri} with policy for objects (ontologies)\todo{unclear}
  \item [policy] policy for creation and control (only authorized systems are allowed) and policy for who is
        eligible to create \glsfmtshort{kgiri}s (i.e. a certificate needed to assign and only certain processes are
        eligible for certificates)

\end{itemize}

\kgmmscoringlevelThree

\begin{itemize}[leftmargin=1.5in]

  \item Security procedures (i.e. specific crypto-security certificates) implemented to ensure control over
        \glsfmtshort{kgiri} assignment process
  \item Implement standard \glsfmtshort{kgiri} class hierarchy structure to enable multi-location linkage (shift from
        URL “flavor” to URN “flavor” as defined in the IRI standard)\todo{needs work}
  \item All inbound data flows (into the \glsfmtshort{ekg}) are using a "Lookup Service" to convert "Strings to Things"
        (i.e. mapping traditional identifiers to \glsfmtshort{kgiri}s in a model-driven way)

\end{itemize}

Nothing (yet) beyond level 3

\section{Ontologies \& Data Models}\label{sec:ekgmm-b-2-2} % B.2.2 Ontologies and Data Models

Data modeling is a formal process designed to describe the data needed to support the business functions within
an organization.
Effective data modeling is a communications mechanism to ensure a shared understanding of requirements between
business stakeholders and applications developers.
Data modeling (via either ontology design or more traditional techniques) results in agreed terminology, precise
definitions and alignment of the models to applicable business rules.

\kgmmekgrationalesection

Well-designed data models describe what the data means as well as how concepts are connected.
Conceptual data models include the expression of high-level concepts and capture business rules to provide a starting
point for the development of operational ontologies.
These conceptual models should link to business glossaries and be directly translated into physical data structures.
The link to precise meaning serves to mitigate problems created using the same word with multiple definitions and
the challenges of expressing conceptual nuance using a variety of words.

Semantic modeling also eliminates the problem of hard-coding assumptions about the world into a single data model.
And while multiple models may co-exist, they are able to be mapped and connected to each other.
In a mature environment, the data modelling process drives technology implementation, by defining the detailed
data structures and associated APIs.
These components (along with functional code) are included as part of the testing suite within the knowledge graph
to facilitate rapid deployment.

\kgmmcorequestionssection

\begin{core-questions}

  \item [\thesection.1] Which areas of the business use well-defined data models
  \item [\thesection.2] Are the data models of the organization linked to the business glossary
  \item [\thesection.3] Are the data models directly linked to the data structures of the consuming applications'
                        data structures
  \item [\thesection.4] How are data models synchronized and aligned across the organization
  \item [\thesection.5] How are the links between data modeling and systems implementation approved and tested
                        (i.e. standard process for approval, change management logs)
  \item [\thesection.6] Are the models used to generate code-specific artifacts (i.e. APIs)
  \item [\thesection.7] Are the data models searchable from the corporate intranet
  \item [\thesection.8] Can the adoption of shared meaning be demonstrated and verified by audit
  \item [\thesection.9] Is the organization using (or mapping to) industry-standard reference points to
                        accelerate development

\end{core-questions}

\begin{members-only}

\kgmmscoringsection

\kgmmscoringlevelOne

\begin{scoring}

  \item [meaning] The granular business meaning of all onboarded data from source systems is verified by SMEs and
        captured as an ontology (machine-readable; every property has its predicate-IRI)
  \item [implementation] Ensure that all onboarded data is tagged with the predicate-IRIs
  \item [basic ontology structure] Define and agree use case specific (target) ontologies for expressing shared and
        foundational concepts [practical target ontologies to support data mapping, harmonization and lineage]
  \item [mapping] Apply transformations from source-to-target ontologies (convert ETL pipeline to standard ontologies)\todo{rephrase}
  \item [governance] Implement governance process for ontology ownership and maintenance (must have responsible parties
        for each domain to facilitate approvals and support change management)
  \item [testing] Implement automated testing and validation of ontologies (per use case) - logical structure, coverage,
        no circular reasoning
  \item [testing] Every change to any ontology and every mapping must be tested (thou shalt never do anything if it
        cannot be tested)
  \item [DevOps] Implement the DevOps environment for ontology lifecycle management with source and version control
        (i.e. Git) - every change will be in Git and tested - do it right from the beginning

\end{scoring}

\kgmmscoringlevelTwo

\begin{scoring}

  \item [enhanced ontologies] Enhance target ontologies with \iindex{OWL axioms}, \gls{bitemporality}, to support
        transactions and events (for reasoning)
  \item [reasoning] Develop upper ontologies and abstractions for shared concepts (prepare for reasoning capabilities
        as a building block)
  \item Domain taxonomies are created and annotated to use cases and related ontologies (must know what ontologies
        exist and for which domain they are targeted)
  \item [governance] Onboard governance activities (full audit trail on changes and approvals) into the knowledge graph
  \item [testing] Onboard all test scripts and validation of tests into the knowledge graph for quality assurance ranking
  \item [architecture governance] Develop and agree to process for ontology policies and codification of architectural
        decisions and patterns

\end{scoring}

\TODO[inline]{Finish scoring levels for EKG/MM capability level 3 and higher}

\end{members-only}


\section{Inventory Management}\label{sec:ekgmm-b-2-3} % B.2.3 Inventory Management

A \iindex{Data Inventory} is a repository of data assets\index{Data Assets} for the organization (also known as a
data catalog\index{Data Catalog} or metadata repository\index{Metadata Repository}).
The inventory contains information about what data exists, where it resides (“data-at-rest”)\index{Data-at-Rest},
responsible parties, upstream/downstream usage, how it moves (“data in motion”)\index{Data-in-Motion}, classifications,
quality designations, availability and other useful metrics.
The contents are catalogued at physical, logical and business levels using defined organizational standards.
Consistently documenting data is the backbone of an effective data management program.
Policies and controls are required to ensure the inventory remains accurate and relevant.

\kgmmekgrationalesection

Linking data inventory to the knowledge graph (and business concepts) ensures precision of meaning at the most
granular level.
Data in the knowledge graph is traceable to all application usage allowing users to find data of interest through
assisted and contextual search and navigation.
The unique ability of the knowledge graph to connect data with metadata enables users to perform flexible queries
in ways that were not previously possible.

\kgmmcorequestionssection

\begin{itemize}[leftmargin=.5in]

  \item [\thesection.1] Do (one of more) data inventories exist
  \item [\thesection.2] Is the inventory based on defined standards (for both meaning and format)
  \item [\thesection.3] Is defined and in-scope data (both breadth and depth) covered in the inventories
  \item [\thesection.4] Is the data inventory linked to \glsxtrfullpl{sor} and authorized data distribution points
  \item [\thesection.5] Is the inventory linked to the business meaning of the data and expressed using standards
  \item [\thesection.6] Is the creation and maintenance of the data inventory mandated by policy and incorporated
                        into the data strategy
  \item [\thesection.7] Is the quality of the content in the inventory measured, reported to involved stakeholders
                        and used for process enhancement

\end{itemize}

\kgmmscoringsection

(concepts are inherited as levels progress)

\kgmmscoringlevelOne

\begin{itemize}[leftmargin=1.5in]
  \item Sources, data sets and metadata are onboarded and expressed as formal ontologies
  \item Authoritative (upstream) data sources and (downstream) consumers are documented and verified by users,
        data and technology
  \item The inventory of applications is defined and selected for graph applications
  \item Requirements and dependencies for each outbound data flow are documented and verified (implementation in the
        graph is not a requirement)
  \item Business glossaries for in-scope use cases are defined and verified in the graph (including a list of
        data sources and datasets)
  \item Policy implemented mandating inventory maintenance and only authorizing the use of data that has been logged
        into the inventory
  \item Access control and entitlement is implemented for in-scope data and related use cases (at the named graph level)
\end{itemize}

\kgmmscoringlevelTwo

\begin{itemize}[leftmargin=1.5in]
  \item Use case trees\index{Use Case Tree} (taxonomies and dependencies) are defined, standardized and implemented
  \item All upstream data sources are linked to the authorized systems of record and distribution points
  \item Policy mandating the use of \glsxtrfullpl{sor} and documentation of data flow is implemented
  \item Entitlements have been defined in the graph (governing access to sources of data in the inventory)
  \item Classifications (i.e. criticality, security, privacy) are aligned with the use case tree and captured in the
        knowledge graph
  \item Governance requirements (i.e. use cases, accountability, data sources, data flows, SLAs) are modeled and
        registered into the knowledge graph
\end{itemize}

\kgmmscoringlevelThree

\begin{itemize}[leftmargin=1.5in]
  \item Data inventory is centralized in the graph and linked to governance for defined use cases
  \item Ontologies and data models (including change history and transformations) are registered in the knowledge graph
  \item Entitlements are calculated within the inventory and enforced at the datapoint level
  \item Data quality is automatically calculated (fine-grained with dynamic value resolution) within the inventory
        for each use case
  \item Data retention rules are registered in the graph and automatically enforced
  \item Full audit trail for all upstream and downstream data usage is registered in the graph
  \item Data elements, calculation methods and CDEs are linked to individual regulatory requirements
\end{itemize}

\kgmmscoringlevelFour

\begin{itemize}[leftmargin=1.5in]
  \item Connected inventory has been extended to include real-time (transactional) data
  \item Inventory is extended to external suppliers and third parties along the supply chain
  \item The inventory is fully integrated with machine learning to optimize data flow
  \item The “value of data” is calculated and classified within the organizational inventory
\end{itemize}

\section{Business Data Glossary}\label{subsec:b-2-4} % B.2.4 Business Glossary

Business data definitions are non-technical descriptions of an organization’s data assets.
The glossary harmonizes the details from applications-specific dictionaries providing transparency into definitions that promote consistency of business terms, identify synonyms and link business content to technical definitions stored in operational systems.
Agreement on granular meaning by involved stakeholders is essential to managing conceptual nuance across shared applications and linked processes.

\subsection*{\glsfmtshort{ekg} Rationale}

Operational systems (including their data warehouses and content repositories) use data models and predefined physical schemas that are designed to serve specific purposes.
In large organizations, this process can result in multiple business glossaries that reflect the local “business language” corresponding to these functions.
Mapping these glossaries to each other (and to the consuming applications) is complex and best accomplished using technology and standards that are specifically designed for this purpose.
The ability of the knowledge graph to link data with metadata ensures that data concepts are defined at their most granular (atomic) level and eliminates confusion over meaning when data is validated or transformed.

\subsection*{Core Questions}

\begin{itemize}[leftmargin=.5in]
  \item [\thesection.1] Are there multiple (official) business glossaries for lines of business or functions
  \item [\thesection.2] Are there internal standards for content, format and use of the business glossaries
  \item [\thesection.3] Is the business meaning of the data verified by accountable subject matter experts
  \item [\thesection.4] If the organization maintains multiple business glossaries, how are they reconciled, verified, maintained and governed
  \item [\thesection.5] Are the glossaries mapped to all expressions of the data in various systems (how is this maintained)
  \item [\thesection.4] Are the business glossaries accessible and searchable from the corporate intranet

\end{itemize}

\subsection*{Scoring Criteria}

Core Data Management Requirements
\begin{itemize}
  \item The organization has one (or many) officially designated glossaries
  \item Front-to-back SME review and verification
  \item Official glossaries are captured in an inventory
  \item There is a defined policy mandating that all glossaries have verified definitions
  \item All glossaries are maintained with a defined mechanism for change management
  \item All in-scope domains are covered (and all content is included)
  \item Mapped to processes and expressions (lineage and transformation)
  \item Governance (ownership, transformation, logic, allowable values, audit trail)
\end{itemize}

\kgmmscoringlevelOne

\begin{itemize}
  \item All concepts for any KG use case are covered by the glossary
  \item Process exists to add new terms (and updating) into the official glossaries
\end{itemize}

\kgmmscoringlevelTwo

\begin{itemize}
  \item Relevant terms (for KG use cases) are expressed as RDF and imported into the KG platform
  \item Relevant terms  are linked to their corresponding ontologies
\end{itemize}

\kgmmscoringlevelThree

\begin{itemize}
  \item All glossaries within the organization are expressed as RDF and imported into the KG platform
  \item All glossaries are searchable via the knowledge graph interface
  \item The governance process (including all metadata) for glossaries are on-boarded into the knowledge graph [passive, linked to existing governance tool, read-only]
\end{itemize}

\kgmmscoringlevelFour

\begin{itemize}
  \item The \acrlong{ekgp} is the authoritative source for all glossaries
  \item All glossary terms are derived directly from the ontologies
  \item The governance processes for all new terms (and changes) are managed directly within the \acrshort{ekg}
  \item Data usage (per glossary term) is monitored and reported
\end{itemize}

\kgmmscoringlevelFive

Nothing additional

\section{Data Integration}\label{sec:ekgmm-b-2-5} % B.2.5 Data Integration

Data integration is the process of combining data from different sources into a single, unified view for
business consumption and enhanced utility.
The process of integrating data from multiple sources begins with the ingestion process and may include activities
such as data profiling, cleansing/remediation, cross-referencing transformation, and field mapping.

\ekgmmContextSection

Data integration is challenging because data from multiple sources can have different file formats, data structures,
definitions, and contextual meanings.
Using the knowledge graph during data integration standardizes the meaning of data and makes the content understandable
to both humans and machines.
Embedding referenceable meaning into the data using machine-readable standards facilitates automatic validation and
assurance of \iindex{data quality}.
Registering data integration activities into the knowledge graph generates full data transparency across
linked processes.
Finally, ontology-based metadata representations make it possible to embed business rules and accommodate
different values, identities, and definitions that existed at various times in the entity lifecycle.

\kgmmcorequestionssection

\begin{core-questions}

  \item [\thesection.1] Are the data integration activities, their systems, repositories, and connections
                        known and tracked
  \item [\thesection.2] Are data integration activities linked to data inventory, business glossaries, and data models
  \item [\thesection.3] Are all data integration input and output datasets documented, tracked, and governed
  \item [\thesection.4] Are there reusable standards and defined business rules for performing data integration
  \item [\thesection.5] Are data integration patterns, tools, and technologies defined, governed, and used
  \item [\thesection.6] Has the firm established a central data integration function
                        (i.e. integration Center of Excellence) to manage \glsxtrshort{etl} across both
                        internal and external data pipelines
\end{core-questions}

\kgmmcorequestionssection

The goal is not always a single source of data - but rather the ability to choose the right authoritative source
for the appropriate context.

\kgmmscoringsection

\kgmmscoringlevelOne

\begin{scoring}

    \item All data sources are identified and documented for in-scope use cases
    \item Do we know the authoritative source for each data set (should not be able to do integration without
          using approved authoritative source)
    \item Does everyone agree that we are using the right sources (the right source for every context) --
          link to governance
    \item Do we have an approved list of what each source feeds (precise description at the entity level that
          we can get from an approved source\,---\,must know if this is the primary source of the data per the
          use case context).
          For any given entity do I have all the potential sources and for a specific context do I know which
          is authorized.
    \item There is a defined governance process for change management and testing (clear picture of all the
          dependencies for data integration).
          If there are changes to authoritative sources\,---\,do we know the downstream implications (tracked and tested)
    \item Are all \glspl{technology-stack} known and supported by current teams (are all key systems under the
          management and governance of the organization\,---\,should not have ghost systems that are not controlled
          as part of the integration process)
    \item Entitlement policies and classification rules (i.e. security, PII, business sensitive) are
          defined and verified
    \item \iindex{Data quality} requirements are defined, documented, and verified

\end{scoring}

\kgmmscoringlevelTwo

\begin{scoring}

    \item All information (above) are identified, precisely defined, and on-boarded into the knowledge graph
    \item Able to do datapoint lineage (detailed and complete view of the data integration landscape)
    \item Start making the \gls{ekg} the central point for data integration (the \gls{ekg} becomes the Rosetta stone of
          integration)\,---\,onboard systems, convert to RDF, integrate into \gls{ekg} (defined as the
          data integration strategy\,---\,not necessarily complete)
    \item All data sets that are on-boarded into the \gls{ekg} are coming from the authoritative sources.
          There are no man-in-the-middle systems.
          The goal is direct from the authoritative source to the target system for in-scope use cases.
          Must get the most granular data directly from the authoritative sources.
    \item All datasets are "\glspl{ekg:sdd}".
    \item [policy] all data is obtained from the \gls{ekg} as the authoritative source.
          Do not go directly to the originating source of the data.
    \item Entitlement policies and classification requirements are on-boarded into the \gls{ekg}
    \item \hyperref[sec:ekg-mm-data-quality-business-rules]{Data quality business rules}
          are on-boarded into the \gls{ekg}

\end{scoring}

\kgmmscoringlevelThree

\begin{scoring}

    \item Data is precisely defined (granular level) - expressed as formal ontologies - and on-boarded into the \gls{ekg}
    \item All data flows are modeled, defined, and registered in the \gls{ekg} (full lineage in the \gls{ekg} for all
          in-scope applications)
    \item Start to make the \gls{ekg} the authoritative source (set-up to facilitate decommissioning of systems).
          The \gls{ekg} is structured to become the “new” system for in-scope applications (as soon as all
          connections emanate from the \gls{ekg}).
    \item Entitlements are automatically managed enforced

\end{scoring}

\kgmmscoringlevelFour

\begin{scoring}

    \item [policy] All downstream client systems are using authoritative sources as the only source of information
          for in-scope datasets (\gls{ekg} is in the middle of all data flows)
    \item All “\iindex{cottage industry} systems” are replaced by the \gls{ekg}
          (and \gls{ekg} is able to perform all the requirements of any system it replaced --
          reporting, entitlement, quality control)

\end{scoring}



\chapter{Data Quality}\label{ch:ekgmm-b-3} % B.3 Data Quality

The \currentname capability has the following sub-capabilities:

\begin{itemize}[leftmargin=.5in]
  \item [\ref{sec:ekgmm-b-3-1}] \nameref{sec:ekgmm-b-3-1}
  \item [\ref{sec:ekgmm-b-3-2}] \nameref{sec:ekgmm-b-3-2}
  \item [\ref{sec:ekgmm-b-3-3}] \nameref{sec:ekgmm-b-3-3}
\end{itemize}

\section{Data Quality Framework}\label{sec:ekgmm-b-3-1} % B.3.1 Data Quality Framework

Data quality is a measurement of the degree to which any dataset is fit for its intended purpose.
It is based on an understanding of application requirements and derived by reverse engineering of the
data production process\index{Data Production Process}.
A data quality framework is an agreed methodology including operational controls, governance processes and
measurement mechanisms.
The framework is designed to support organizational priorities for data quality based on criticality and business value.

\ekgmm-context-section

Data in the knowledge graph is structured around interconnectivity and precision.
The goal is assurance of granular meaning so that users have confidence they are getting all the information they
need to understand context and solve ad hoc business questions.
In a semantic environment, every datapoint is resolved to a universally unique identifier, ensuring discoverability
across repositories and domains.
Data and metadata are connected so that logical errors and data inconsistencies are detected before they enter
the system.
From a compliance perspective, data in the graph is immutable because lineage\index{Lineage} can be traced and
nothing can be deleted except by policy.

\kgmmcorequestionssection

\begin{core-questions}

  \item [\thesection.1] Has the data management strategy for the organization been defined, verified and aligned
                        with the operating model
  \item [\thesection.2] Have business requirements been captured and linked to granular data concepts
  \item [\thesection.3] Have relevant data sources been identified and linked to \glspl{sor}
  \item [\thesection.4] Is data precisely defined and mapped to enterprise models\todo{or any semantic
                        machine-readable model i.e. ontology?}
  \item [\thesection.5] Is the governance infrastructure (roles, responsibilities, funding requirements,
                        measurement criteria) imple\-mented and operational
  \item [\thesection.6] Are communications mechanisms in place to ensure that quality issues are verified,
                        addressed at source and linked to consuming applications
  \item [\thesection.7] Are end users getting the data they need (and able to use it without the need for
                        reconciliation or manual transformation)

\end{core-questions}

\subsection*{Concepts to Discuss}

\begin{enumerate}

  \item have business requirements been captured and verified in the form of use case trees and business user stories
  \item are all data sources identified; defined in the knowledge graph; SOR and authorized distribution points;
  \item all data in machine-readable form and linked to ontologies

\end{enumerate}

\section[Business Rules]{Data Quality Business Rules}% B.3.2 Business Rules
\label{sec:ekgmm-b-3-2}
\label{sec:ekg-mm-data-quality-business-rules}
\index{data quality}

Data Quality business rules ensure that the data is fit for its intended purpose.
Subject matter experts specify the criteria used to validate and enforce data integrity.
The criteria are translated into agreed specifications (i.e. business rules) which are later codified for
\iindex{data profiling} or measuring conformity.
Data quality business rules can also be embedded into data capture systems to ensure validity at source.

\ekgmmContextSection

Data quality is an integrated feature of the knowledge graph.
Data quality rules are linked to structured business vocabularies\todo{or better: ontologies} to ensure that meaning
is shared and not obscured by vague terms or cryptic codes.
The logic of business rules and policies are\change{is} captured and expressed as executable models and consistently
enforced across all systems and processes.
These quality constraints (models) allow firms to measure the quality of the data and perform verification across
disparate systems.
In the mature \gls{ekg}, violations of logic or integrity are identified and prevented before data
enters the system.

\kgmmcorequestionssection

\begin{core-questions}

  \item [\thesection.1] Are data quality business rules specified, formalized, and expressed in a standardized manner
  \item [\thesection.2] Is there a clear line of business \iindex{accountability} (owned, funded, and governed) for the
                        \iindex{data quality} rules
  \item [\thesection.3] Is there a centrally managed repository of business rules\improve{explain that it does not
                        necessarily have to be central as long as it is agreed and enforced at the right scope}
  \item [\thesection.4] Is there a clearly defined mechanism for logging additions and performing updates
  \item [\thesection.5] Are the business rules aligned with business applications and traceable to source systems
  \item [\thesection.6] Are the data quality business rules automated and expressed in a machine-executable format

\end{core-questions}

\kgmmscoringsection

\kgmmscoringlevelOne

\begin{scoring}

  \item \hyperref[sec:ekg-mm-data-quality-business-rules]{Data quality business rules} (conditions) have been defined,
        documented and, verified by \glspl{sme} (process for evaluation and acceptance defined)
  \item Business rules are aligned with in-scope use cases and specific user stories
  \item Business rules are standardized and registered into a repository with a defined mechanism for logging
        additions and performing updates

\end{scoring}

\kgmmscoringlevelTwo

\begin{scoring}

  \item A defined architecture exists to translate business rules into machine-executable code (some rules will be
        OWL\index{OWL} expressions, some will be SHACL shapes\index{SHACL} and constraints, some will be translated into
        workflow logic)
  \item Business \iindex{provenance} and \iindex{lineage} are traceable across the \iindex{data supply chain} and
        evaluated against defined business rules (all business rules must be traceable and understandable in
        context\,---\,must understand the purpose and importance of the rule)
  \item Business rules for in-scope use cases are implemented in the \gls{ekg}

\end{scoring}

\kgmmscoringlevelThree

\begin{scoring}

    \item [Metrics] The measurement criteria are defined for data quality business rules (which rules are executed,
          how often, improvement)
    \item [Performance] The value of business rules are related to business concepts (products, financial performance,
          organizational objectives)\,---\,able to trace the core relationship between the business objectives and
          the data quality business rules (correlation between rules and outcomes are known, able to be queried and
          traceable within the \gls{ekg})

\end{scoring}

\kgmmscoringlevelFour

\begin{scoring}

    \item Business rules are combined with \gls{ai} capability for compliance (dynamic optimization of
          business rules)
    \item Model-driven (senior management can begin to optimize business objectives using business rules in
          the \gls{ekg}\,---\,i.e. alignment of business rules with “what if” scenarios)

\end{scoring}

\kgmmscoringlevelFive

\begin{scoring}

    \item All business rules are driven by business objectives (objectives are in the \gls{ekg} with appropriate scorecards)

\end{scoring}

\section[Execution]{Data Quality Execution}% \thesection.3 Data Quality Execution
\label{sec:ekgmm-b-3-3}
\label{sec:ekg-mm-data-quality-execution}
\index{data quality}

Implementation of the data quality program is a control process mechanism for organizations.
It is driven by the execution of business rules at the point of data capture (validation).
key components include data profiling, root cause analysis, data remediation and issue management.
The primary objective of the data quality program is to accurately and consistently represent the concepts and
values needed to meet the needs of consumers.
A well-orchestrated data quality program requires communication and operational coordination across the
organizational landscape.

\ekgmmContextSection

Data in the knowledge graph is aligned to precise meaning so that errors and definitional conflicts are
verified at source before they are introduced into operational systems.
In the knowledge graph data, rules and metadata are co-resident and reusable without impedance mismatch.
Quality execution is rules-based and unhooked from both schemas and data models that are tailored to
specific applications.
Metadata is embedded into the content so that users always understand what the data represents as it moves across
organizational boundaries.

\kgmmcorequestionssection

\begin{core-questions}

  \item [\thesection.1] Is there a defined process for executing data quality across linked applications
  \item [\thesection.1] Is there a link between business rules and execution
  \item [\thesection.1] Is the governance structure (and funding mechanisms) for managing data quality issues
                        operational
  \item [\thesection.1] Have service level and data sharing agreements been defined and verified by consumers
  \item [\thesection.1] Is there a defined (and repeatable) process for tracing data quality issues to
                        systems of record and downstream applications
  \item [\thesection.1] Is data quality remediation coordinated across the data production and consumption process
  \item [\thesection.1] Are data quality metrics and scorecards captured and reported to involved stakeholders

\end{core-questions}



\chapter{Data Governance}\label{ch:ekgmm-b-4} % B.4 Data Governance

The \nameref{ch:ekgmm-b-4} component has the following capabilities:

\begin{itemize}[leftmargin=.5in]
  \item [\ref{sec:ekgmm-b-4-1}] \nameref{sec:ekgmm-b-4-1} -- Structure and approach for the \gls{ekg} \gls{coe}
  \item [\ref{sec:ekgmm-b-4-2}] \nameref{sec:ekgmm-b-4-2} -- Policies, procedures and standards for managing the data lifecycle
  \item [\ref{sec:ekgmm-b-4-3}] \nameref{sec:ekgmm-b-4-3} -- Management of the data production and manufacturing process 
  \item [\ref{sec:ekgmm-b-4-4}] \nameref{sec:ekgmm-b-4-4} -- \gls{ekg} access control mechanisms
  \item [\ref{sec:ekgmm-b-4-5}] \nameref{sec:ekgmm-b-4-5} -- Identification and tracking of crtiical data attributes and uses
  \item [\ref{sec:ekgmm-b-4-6}] \nameref{sec:ekgmm-b-4-6} -- Creation and implementation of the data control environment
\end{itemize}

\section{Operating Model}\label{sec:ekgmm-b-4-1} % B.4.1 Operating Model

The people, processes, capabilities and tools that define the role of data management in delivering value to an
organization’s customers.
The operating model can help stakeholders understand the complexity of the data manufacturing process and how
components relate to each other.
It specifies the roles and responsibilities of the stakeholders involved in the data management program.
It provides a framework and policies to help align governance concepts with operating requirements and
organizational culture.
An effective operating model can both describe the way the organization does business today (“as is”) and
communicate a vision of how an operation will work in the future (“to be”).

\kgmmekgrationalesection

The operating model for the knowledge graph establishes a new way of running the organization to enable companies to
accelerate development and operate more efficiently.
The \glsxtrshort{ekg} governance framework focuses on combining new data capabilities (i.e. resolvable identity,
standard ontologies, open standards) as part of an integrated process.
A well defined operating model specifies architecture components that support flexible and reusable data to help
organizations achieve improvements in revenue, customer experience and cost.

\kgmmcorequestionssection

\begin{core-questions}

  \item [\thesection.1] Have the underlying principles (and challenges) of data management been established and
                        accepted by involved stakeholders
  \item [\thesection.2] Does the \gls{odm} have authority to enforce adherence to policy
  \item [\thesection.3] Has the data management funding model been established and implemented
  \item [\thesection.4] Has the organization identified and recruited stakeholders with sufficient skill sets to
                        implement the data management program
  \item [\thesection.5] Are involved stakeholders held accountable to data management program deliverables
  \item [\thesection.6] Has the organization defined and validated the operating models, and workflows necessary to
                        implement the data management program
  \item [\thesection.7] Are metrics and KPIs captured and actively used to improve the data management operating process
  \item [\thesection.8] Is the data management operating model audited for compliance and effectiveness

\end{core-questions}


\section{Data Management Policy}\label{sec:ekgmm-b-4-2} % B.4.2 Data Management Policy

Data management policies are the mandated requirements to ensure effective management of an organization’s
information assets.
Policies are implemented via operating models, standards and documented procedures.
Data management policies define requirements related to data access (e.g. authorized sources, entitlements,
usage and redistribution), data quality (e.g. metadata, definitions, modeling and business rules),
organizational governance (e.g. accountabilities and alignment with business processes) and
control functions (e.g. retention, security, privacy and change management).

\kgmmekgrationalesection

Data management policies and standards should be crafted to cover the specifics of the knowledge graph.
Candidate areas for \glsxtrshort{ekg}-related policies include those associated with the management of data catalogs,
the resolution of identity and data meaning, the standards for data modeling and metadata,
the use of \glspl{sor}/authorized distribution points and the control over data entitlements.

\kgmmcorequestionssection

\begin{core-questions}

  \item [\thesection.1] Are policies and standards in alignment with data management strategy
  \item [\thesection.2] Are data management policies documented, complete and operational
  \item [\thesection.3] Are data management policies linked to operational control functions as well as the
                        SDLC process of the organization
  \item [\thesection.4] Have data policies been created in collaboration with (and approved by) business, technology
                        and operational stakeholders
  \item [\thesection.5] Are data management policies aligned with the requirements of the \glsxtrshort{ekg}
  \item [\thesection.6] Have data management policies been approved by executive management and audit

\end{core-questions}


\section{Data Production \& Consumption}\label{sec:ekgmm-b-4-3} % B.4.3 Data Production Process

Data originates from a wide variety of business processes and flows through multiple systems before it is
ingested into applications and analytical functions.
As data flows across this “chain of supply”\,---\,it can be renamed, enriched, transformed and updated (many times).
Data producers and consumers must work collaboratively to ensure that the data is fit for its intended purposes.
This necessitates communications about the requirements for quality as well as an understanding of
how the data is being used, where it originates, how it has been transformed and its consistency with original intent.

\ekgmmContextSection

Management of the production/consumption process is one of the essential components of data management.
It is based on an understanding of what data is being exchanged and the ability to map dependencies and
transformations (\iindex{lineage}).
The knowledge graph is the logical distribution point for data manufacturing\,---\,leveraging the ontology to ensure
consistency of meaning across systems, people and processes.
The \glsxtrshort{ekg} can trace data flow and validate that defined use cases are aligned with agreed usage and
obtained from authoritative sources.
Entitlements and redistribution rights are automatically tracked and fully auditable.
\iindex{Data quality} rules are linked to business vocabulary and structurally enforced across systems and processes
to ensure consistency.

\kgmmcorequestionssection

\begin{core-questions}

  \item [\thesection.1] Is the data production/consumption process documented and verified from authoritative sources
                        to consuming applications
  \item [\thesection.2] Are the requirements for data quality (and criticality) defined, verified and expressed as SLAs
                        or other forms of data sharing agreement
  \item [\thesection.3] Is there an inventory of SLAs and is data production monitored against these agreements
  \item [\thesection.4] Is the data production/consumption process connected to ontologies and
                        official business glossaries
  \item [\thesection.5] Are entitlements, permissions and authorizations tracked as data flows across systems
  \item [\thesection.6] Can data quality errors and deficiencies be traced back to where the data originated
  \item [\thesection.7] Are the data production/consumption governance processes in place and operational
                        (\iindex{ownership} and \iindex{accountability})

\end{core-questions}


\section{Entitlement Management}\label{sec:ekgmm-b-4-4} % B.4.4 Entitlement Management

Entitlement management is technology that grants and enforces access rights to data, devices, systems and services.
Entitlement systems are linked to organizational policies (rules) governing access.
These systems track access to applications to ensure that actions are in line with policies and to provide data about access for audit purposes.
Entitlements need to be managed in sync with the goals for security, business continuity and compliance.
There can be multiple proprietary systems that all have unique entitlement structures and individuals can move across departments and perform a variety of roles.

\kgmmekgrationalesection

Classification and access rights within the EKG are linked to dataflow and managed at the granular level.  Lineage is automatically tracked and fully auditable by source, purpose and responsible party.  Security is embedded into the design of the data and not constrained by either systems or administrative complexity.  Rules can be modeled for all circumstances and controlled at both the datapoint and applications level.  Using a knowledge graph for fine-grained access control reduces complexity and enhances enforcement capability.

\kgmmcorequestionssection

\begin{core-questions}

  \item [\thesection.1] Are the rules for entitlement and access control documented and verified for defined circumstances
  \item [\thesection.2] Are entitlement rules and application logic expressed in machine-executable format
  \item [\thesection.3] How are entitlement permissions (and changes) administered, tracked and audited
  \item [\thesection.4] At what level are entitlements managed (i.e. datapoint, platform, applications, role)
  \item [\thesection.5] Are entitlement classifications linked to lineage, dataflow and transformation processes
  \item [\thesection.6] Are entitlements and access control linked to \glsxtrfullpl{sor}

\end{core-questions}


\section{Critical Data Elements}\label{sec:ekgmm-b-4-5} % B.4.5 Classification Management

Critical data elements (CDEs) are designated as having a material impact on a business process, regulatory report, financial calculation or risk measure.
The designation of a data element as critical means it is important enough to be governed by organizational policy.
CDE management is implemented as a negotiation process between producers and consumers that requires precision of meaning,
understanding of data flow, capture of business rules and the management of quality based on how the data will be used.
Many firms use the term CDE to describe various things such as data attributes, business concepts, derived processes and physical expressions.
This confusion over nomenclature about concepts makes it difficult to distinguish granular data from how it is assembled into business measures and how it is expressed in physical data repositories.

\kgmmekgrationalesection

In the \glsxtrshort{ekg}, critical data are defined by data quality business rules and expressed as machine-executable models
These rules are automatically executed across systems, processes and applications to ensure consistency.
Data is linked to the ontology and resolved to a single identifier to mitigate confusion about meaning when the data is onboarded or transformed.
The \glsxtrshort{ekg} is able to trace data flow and verify that criticality is aligned with \glsxtrfullpl{sor} and agreed usage.

\kgmmcorequestionssection

\begin{itemize}[leftmargin=.5in]

  \item [\thesection.1] Are critical data elements identified, verified and prioritized for specific use cases and applications
  \item [\thesection.2] How is the organization managing the distinctions between granular data attributes and derived/calculated business concepts
  \item [\thesection.3] Is the inventory of critical data elements implemented and linked to how the data is being consumed
  \item [\thesection.4] Are critical business elements connected to business glossaries, ontologies, \glsxtrfullpl{sor} and authorized distribution points
  \item [\thesection.5] Is the front-to-back flow of data defined, validated and linked to the designations of critical data
  \item [\thesection.6] Are the governance mechanisms for managing critical data defined and operational

\end{itemize}

\section{Risk \& Control Environment}\label{ch:b-4-6} % B.4.6 Risk and Control Framework

Operational risk is the result of inadequate or failed internal procedures,
systems or policies that are not driven by external forces such as economic, political or systemic events.
Understanding how data flows into decision making and operational processes is one of the key components of operational risk mitigation.
Many firms are working to establish strong control environments to identify, approve and monitor operational risk.
Control processes include a combination of strong governance, standards for ensuring interoperability and quality assurance across the full data lifecycle.

\subsection*{\glsfmtshort{ekg} Rationale}

Control process in the \acrshort{ekg} environment are managed as a structured set of executable business rules and automatically enforced across all applications.
The meaning of the data is structurally validated at the point of data capture to prevent bad data from entering the system at ingestion.
In the knowledge graph, data and metadata are fully integrated to ensure reusability across systems and operational processes.
The \acrshort{ekg} is able to map data flow including all dependencies and transformations to verify that data is obtained from authorized systems of record,
identified according to access rights and aligned with intended usage.

\subsection*{Core Questions}

\begin{itemize}[leftmargin=.5in]

  \item [\thesection.1] Has the organization developed a structured framework outlining the principles of how operational risk is identified, assessed, monitored and controlled
  \item [\thesection.2] Has the operational risk framework been adopted by executive management and verified by internal audit
  \item [\thesection.3] Have oversight mechanisms been adopted to ensure compliance with operational risk control policies and governance procedures
  \item [\thesection.4] Are technology resiliency and continuity plans in place to ensue systems integrity, security and availability during mergers, acquisitions and consolidations

\end{itemize}


