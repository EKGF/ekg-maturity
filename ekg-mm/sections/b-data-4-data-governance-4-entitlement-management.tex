\section{Entitlement Management}\label{sec:ekgmm-b-4-4} % B.4.4 Entitlement Management

Entitlement management is technology that grants and enforces access rights to data, devices, systems and services.
Entitlement systems are linked to organizational policies (rules) governing access.
These systems track access to applications to ensure that actions are in line with policies and to provide data about access for audit purposes.
Entitlements need to be managed in sync with the goals for security, business continuity and compliance.
There can be multiple proprietary systems that all have unique entitlement structures and individuals can move across departments and perform a variety of roles.

\kgmmekgrationalesection

Classification and access rights within the EKG are linked to dataflow and managed at the granular level.  Lineage is automatically tracked and fully auditable by source, purpose and responsible party.  Security is embedded into the design of the data and not constrained by either systems or administrative complexity.  Rules can be modeled for all circumstances and controlled at both the datapoint and applications level.  Using a knowledge graph for fine-grained access control reduces complexity and enhances enforcement capability.

\kgmmcorequestionssection

\begin{itemize}[leftmargin=.5in]
  \item [\thesection.1] Are the rules for entitlement and access control documented and verified for defined circumstances
  \item [\thesection.2] Are entitlement rules and application logic expressed in machine-executable format
  \item [\thesection.3] How are entitlement permissions (and changes) administered, tracked and audited
  \item [\thesection.4] At what level are entitlements managed (i.e. datapoint, platform, applications, role)
  \item [\thesection.5] Are entitlement classifications linked to lineage, dataflow and transformation processes
  \item [\thesection.6] Are entitlements and access control linked to \glsxtrfullpl{sor}
\end{itemize}
