%
% B.4.5 Classification Management
%
\ekgmmCapability{b-4-5}{classification-management}{Classification Management}

\ekgmmCapabilitySectionContributionToEnterprise

In the \glsxtrshort{ekg}, \glsfmtlongpl{cde} are defined by data quality business rules\index{data!quality!rules}
and expressed as machine-executable models\index{executable models}.\improve{i.e. ontologies}
\improve[inline]{%
    JAG>We should be more explicit about this by saying that \glspl{cde}
    basically do not exist as such anymore, a data element is deemed critical from the point of view of the
    enterprise perhaps but that's just one viewpoint, one context.
    In the \gls{ekg} we have to serve any viewpoint.
    For every use case there are data elements (concepts) that are critical which may be a whole
    different set of data elements than what enterprise level deems to be critical.
    All data elements are critical somewhere in some context.
    This has to do with the "\glsfirst{svot} delusion".
    It's of course fine to label certain data elements as "critical" but it should be tied to the use case.
    Data element X is critical for use case Y.},
These rules are automatically executed across systems, processes and applications to ensure consistency.
Data is linked to the ontology and resolved to a single identifier to mitigate confusion about meaning when the
data is onboarded or transformed.
The \glsxtrshort{ekg} is able to trace data flow and verify that criticality is aligned with \glspl{sor}
and agreed usage.

\ekgmmCapabilitySectionDimensions

\begin{core-questions}

  \item [\thesection.1] Are critical data elements identified, verified and prioritized for specific
                        use cases and applications
  \item [\thesection.2] How is the organization managing the distinctions between granular data attributes and
                        derived/calculated business concepts
  \item [\thesection.3] Is the inventory of critical data elements implemented and linked to how the
                        data is being consumed
  \item [\thesection.4] Are critical business elements connected to business glossaries, ontologies,
                        \glspl{sor} and authorized distribution points
  \item [\thesection.5] Is the front-to-back flow of data defined, validated and linked to the designations of
                        critical data
  \item [\thesection.6] Are the governance mechanisms for managing critical data defined and operational

\end{core-questions}

