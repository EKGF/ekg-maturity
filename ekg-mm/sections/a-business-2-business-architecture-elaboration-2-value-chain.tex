\section{Value Chain}\label{sec:ekgmm-a-2-2}

This capability is about having the Value Chain(s) defined, updated, used and managed.

\somequote{%
    A value chain is a business model that describes the full range of activities needed to create a product or
    service.
    For companies that produce goods, a value chain comprises the steps that involve bringing a product from
    conception to distribution, and everything in between—such as procuring raw materials,
    manufacturing functions, and marketing activities.
    \begin{itemize}
        \item A value chain is a step-by-step business model for transforming a product or service from idea to reality.
        \item Value chains help increase a business's efficiency so the business can deliver the most value for the
              least possible cost.
        \item The end goal of a value chain is to create a competitive advantage for a company by increasing
              productivity while keeping costs reasonable.
        \item The value-chain theory analyzes a firm's five primary activities and four support activities.
    \end{itemize}
}{Investopedia}{https://www.investopedia.com/terms/v/valuechain.asp}

We welcome your input here.

\ekgmmContextSection

\ekgmmHowEKGRequiresThisCapability

Having the Value Chain(s) of an organization defined can help with the selection, definition and prioritization of
the right use cases for the \gls{ekg}.

\ekgmmHowEKGAffectsThisCapability

In the \gls{ekg} context, a "digital twin" of a company's value chain can be modelled where all the various components
of the value chain are represented e.q. all details around logistics, supply chains, operations, services, marketing,
sales and all support activities.
At higher levels of maturity, all real-world details of these components are available as well and represent
the reality accurately and in "real time".

We welcome your input here.

