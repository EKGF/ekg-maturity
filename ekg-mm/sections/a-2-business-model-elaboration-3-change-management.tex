\section{Change Management}
\label{sec:ekg-mm-a-2-3}
\label{sec:ekg-mm-change-management}
\index{change management}

Adopt alternative organizational and operational approaches.
Manage operations change to new state of sustainable value.

\somequote{%
    Change management\,---\,The business process that coordinates and monitors all changes to the business processes
    and applications operated by the business, as well as to its internal equipment, resources, operating systems,
    and procedures.
    The change management discipline is carried out in a way that minimizes the risk of problems that will affect
    the operating environment and service delivery to the users.%
}{ASCM}{https://ASCM.org}

Change management guides the business for change required to react to problems or to proactively plan changes to
mitigate risk, obtain a good result for divesting and/or acquiring new areas of business, product changes for
shifts in customer needs, improve processes for efficiency and cost savings, stay ahead of competitors,
take advantage of new technologies, etc.

Change management is the discipline to guide realization and sustainment of business strategy intent.
Impact of changes in strategy should be modeled and serve as a basis for executing the change.
Change can be managed for any aspect of the business to include: geographic and facility footprint, market expansion,
product development, business organization and staffing, new sales channels, process improvement, technology change,
compliance to changes in regulations, and more.

\ekgmmContextSection

\paragraph*{How \glsfmtshort{ekg} requires this capability}
The \gls{ekg} should be updated to reflect changes made to the enterprise as a result of change projects.
\gls{ekg} governance\index{\glsfmtshort{ekg}!governance}\index{governance} is a set of change processes that should follow a
disciplined change approach to protect the integrity and value of the knowledge graph.

\paragraph{How \glsfmtshort{ekg} affects this capability}

The \gls{ekg} can hold the model of the enterprise to identify potential areas for change to include:
discover duplication across business silos, impact analysis of areas targeted to change,
impact of regulatory change, impact of entering new markets, impact of scaling the business,
impact from major incidents and issues, and can serve as a baseline for evaluating and executing
merger/acquisition of new enterprises.

The \gls{ekg} can hold the information on current state of the enterprise for reuse of information to start the
project, ideal state of the enterprise, intended new state of the enterprise after the change is deployed,
and updated for actual resulting business knowledge from deploying the change.

\paragraph*{More background}
Change is happening everywhere in organizations.
The \iindex{digitalization} of everything (\gls{iot}?, \iindex{edge computing}, \iindex{digital transformation})
has created a fluid context for change, and the need to decentralize decision-making to keep pace with
exponential change and complexity requires \iindex{non-linear thinking}, planning, and execution.

The current approaches for change management are usually built on linear models which do not scale nor can they collate
the facets of data\,---\,let alone semantic perspectives\,---\,that are overwhelming decision makers.

\Glspl{ekg} are the only way to build a scalable framework of understanding about change and capture relevant patterns
to guide option analysis and impact analysis.
The days of getting in a room and relying on meetings is quickly coming to an end.
\Gls{ai}, Bayesian Analysis\index{bayesian analysis}, Game Theory\index{game theory}, \gls{mcda},
and Heuristics can help, but \myuline{only} if we can guide them properly using \glspl{ekg}.

