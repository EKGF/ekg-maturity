%
% B.4.4 Entitlement Management
%
\ekgmmCapability{b-4-4}{entitlement-management}{Entitlement Management}

\ekgmmCapabilitySectionContributionToEnterprise

Classification and access rights within the \gls{ekg} are linked to dataflow and managed at the granular level.
\iindex{Lineage} is automatically tracked and fully auditable by source, purpose and responsible party.
Security is embedded into the design of the data and not constrained by either systems or administrative complexity.
Rules can be modeled for all circumstances and controlled at both the datapoint and applications level.
Using a knowledge graph for fine-grained access control reduces complexity and enhances enforcement capability.

\ekgmmCapabilitySectionDimensions

\begin{core-questions}

  \item [\thesection.1] Are the rules for entitlement and access control documented and verified for defined circumstances
  \item [\thesection.2] Are entitlement rules and application logic expressed in machine-executable format
  \item [\thesection.3] How are entitlement permissions (and changes) administered, tracked, and audited
  \item [\thesection.4] At what level are entitlements managed (i.e. datapoint, platform, applications, role)
  \item [\thesection.5] Are entitlement classifications linked to lineage, dataflow and transformation processes
  \item [\thesection.6] Are entitlements and access control linked to \glspl{sor}

\end{core-questions}

\ekgmmCapabilitySectionLevelsOneFive

\ekgmmscoringlevelOne

\begin{scoring}

  \item Manual alignment of law, policy and owner requirements to rule (source of the data that determines entitlement
        imported from systems like Active Directory or LDAP and the like “shadow IT”) –-
        KG is not yet the authoritative source
  \item Administration requires manual evaluation
  \item \glsfirst{rbac} (conventional state of the art) implemented for the limited KG use cases

\end{scoring}

\ekgmmscoringlevelTwo

\begin{scoring}

  \item Strategy (exist) to move from Active Directory to KG as the authoritative source
        (artifacts = documents and plans)
  \item Planning underway to convert the KG as the source of entitlement control
        (goal is to eliminate manual process for update)
  \item More advanced levels of granularity (data source owner’s policies) are captured in the KG
  \item Data source prioritization (based on data owners policies) incorporated into the graph –-
        must protect data that owners “manage data control in the same strict way that the data owner
        already has in place” (technical assurance to data owner that the sensitive and private data will be controlled)
  \item Entitlements are linked to governance processes and internal data process flow (auditable)

\end{scoring}

\ekgmmscoringlevelThree

\begin{scoring}

    \item \glsfmtshort{ekgbac} –- artifacts = demonstration of \gls{ekgbac}
          (onboarded the organizational management use case; organizational units, employees, contractors are in
          the KG and linked to \gls{hr}) –- KG must know who you are, roles and access rights before level 3
    \item All people, processes and data (organizational management landscape use case) are linked
    \item Law, regulation, internal policies and requirements of original data owner are modeled as ontology
          plus business logic (and corresponding data is onboarded)
    \item Entitlements are automatically enforced based (on executable policy)
    \item Must be able to push entitlement policies into the KG
    \item The \gls{ekg} covers all domains and all jurisdictions for the scope of its use cases –-
          complete overview of all related regulations, laws and policies;
          all factors that have an impact on enterprise decisions are known and part of the EKG
    \item \gls{ekg} uses data from many sources; systems owners will participate as long as they are confident that
          the requirements are equivalent (EKG and existing system must be kept in alignment for each data source
    \item Entitlement requirements are linked to the actual transactions and “things” that need to be protected

\end{scoring}

\ekgmmscoringlevelFour

\begin{scoring}

  \item Entitlements are linked to AI and reasoning capability to ensure that all entitlement decisions are
        in line with business objectives and organizational goals
  \item Entitlements become more “intelligent” and allow for flexible analysis on the value of
        activities and decisions (i.e. is this a smart thing for us to do)
  \item Artifacts = KG must use advanced AI capabilities; must see evidence of AI algorithms
        (show us that you have \gls{ai} capabilities (must have onboarded the organizational objectives,
        business objectives and roles –- all connected to the organizational management landscape;
        linked to \nameref{sec:ekg-mm-risk-management} (risk factors are linked to entitlement);
        linked to cost/profitability

\end{scoring}

\ekgmmscoringlevelFive

\begin{scoring}

  \item All mundane and repetitive tasks are implemented automatically
        (all workflows and approval steps are done by the EKG) –- artifacts = workflows and
        approval steps are fully automated
  \item Entitlements and pervasive and comprehensive for all activities
  \item Little need for human interaction; completely controlled environment about access rights and data usage

\end{scoring}
