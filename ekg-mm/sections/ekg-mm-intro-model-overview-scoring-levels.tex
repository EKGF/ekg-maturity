\section{Scoring Overview}\label{sec:ekg-mm-scoring-overview}

The \gls{ekgmm} is a capability model designed to promulgate best practices across the knowledge graph community.
It covers essential capabilities as well as standard evaluation criteria required for the design, implementation,
and maintenance of an \gls{ekg}.

The initial model was developed by \agnos and is being managed as an open source initiative by the
Enterprise Knowledge Graph Foundation.

\subsection{Level 1: \glsfmtshort{ekg} Initiation (minimum viable product)}

The domain of pilots and internal POCs.
The focus is usually on specific (targeted) use cases constructed using isolated ontologies.
The champions are visionaries who have assembled a specialist team for implementation.
Funding is likely to be project-based and designed to demonstrate capabilities.

\begin{itemize}[leftmargin=1in,font=\bfseries]

    \item[Business]     Stakeholders recognize the business liabilities from silos and data incongruence.
                        Internal champion is seeking to solve strategic use cases, supports innovation and is willing
                        to take on the disruption challenge.
    \item[Data]         Core data management capabilities (operating model, inventory, data architecture, business
                        glossary, pipeline management, etc.) are being performed.
                        Specific use cases are being implemented with specialist teams for the pilot initiative.
    \item[Technology]   Technology strategy is focused on experimentation and innovation.
                        Manual data transformation and targeted ETL is underway for the pilot.
                        Limited infrastructure and dedicated efforts to build initial knowledge graph components.
    \item[Organization] Champions are internal visionaries who have assembled a specialist team for implementation.
                        The pilot is sanctioned and funded.
                        Knowledge acceleration is being addressed.
                        Overall organizational support is emerging

\end{itemize}

\subsection{Level 2: Extensible Platform (reusable components)}

The domain of parallel knowledge graph activities.
The organization is creating reusable architecture based on expanded design principles.
The EKG Center of Excellence is created.
Funding is likely to be at the line-of-business level.

\begin{itemize}[leftmargin=1in,font=\bfseries]

    \item[Business]     Stakeholders adopt a “data centric” mindset focused on strategic business value.
                        Management elevates the knowledge graph as an organizational and funding priority.
    \item[Data]         Critical data elements are prioritized in the ontology.
                        Approach to identity and meaning resolution is established.
                        Use case trees are defined and modeled to capture shared data relationships.
                        The knowledge graph is becoming the central point for integration.
    \item[Technology]   Reusable architecture based on expanded design principles.
                        Core software development design approaches are being established and incorporated
                        into strategy.
                        CTO focuses on extending pilot initiatives for additional leverage.
    \item[Organization] Operating model of collaboration is implemented to support the knowledge graph.
                        The Center of Excellence and DataOps environment is initiated.
                        Budget and implementation strategy are based on agile and synchronized with the
                        use case tree methodology.

\end{itemize}

\subsection{Level 3: Enterprise Ready (default data hub)}

The establishment of a scalable and resilient platform for business-critical applications.
Resources for the design and build of operational systems are defined and coordinated.
The knowledge graph is now an “enterprise” knowledge graph (EKG) that serves as the data hub for the organization.
Ownership, governance and funding are managed at the enterprise level.

\begin{itemize}[leftmargin=1in,font=\bfseries]

    \item[Business]     Strong collaboration between business, data and technology to prioritize strategic
                        (mission-critical) use cases
    \item[Data]         Inventory is embedded into the \gls{ekg} and linked to governance.
                        Data is expressed as formal ontologies, onboarded into the \gls{ekg} and searchable.
                        Data flows are defined and modeled.
                        The \gls{ekg} is the authoritative source for data.
    \item[Technology]   Commitment to the \gls{ekg} as the strategic infrastructure for the organization.
                        IaC and continuous deployment adopted and implemented.
                        Cloud architecture defined for elasticity.
                        Datapoint security and authentication processes implemented
    \item[Organization] The \gls{ekg} is recognized as a core service for the enterprise.
                        Enterprise-wide ownership and funding processes are operational.
                        The EKG Center of Excellence is a stand-alone production department.

\end{itemize}

\subsection{Level 4: Strategic Asset (operational utility)}

The \gls{ekg} is understood as strategic infrastructure for the organization and the authoritative source
for all data.
It supports applications consolidation, AI and process automation.
Strategic funding is based on the vision of executive management and fully embraced by the Board of Directors.

\subsection{Level 5: Operational Ecosystem (continuous development)}

The \gls{ekg} is central to systems and business processes.
It has been fully integrated into both internal operations and external supply chain partners.
Workflows and approval steps are fully automated.
Entitlements and access rights are controlled by the \gls{ekg}.
Inference and reasoning capabilities are used for advanced AI applications.
