\section{Data Integration}\label{subsec:b-2-5} % B.2.5 Data Integration

Data integration is the process of combining data from different sources into a single, unified view for business consumption and enhanced utility.
The process of integrating data from multiple sources begins with the ingestion process and may include activities such as data profiling, cleansing/remediation, cross-referencing,  transformation and field mapping.

\subsection*{\glsfmtshort{ekg} Rationale}

Data integration is challenging because data from multiple sources can have different file formats, data structures, definitions and contextual meanings.
Using the knowledge graph during data integration standardizes the meaning of data and makes the content understandable to both humans and machines.
Embedding referenceable meaning into the data using machine-readable standards facilitates automatic validation and assurance of data quality.
Registering data integration activities into the knowledge graph generates full data transparency across linked processes.
Finally, ontology-based metadata representations make it possible to embed business rules and accommodate different values, identities and definitions that existed at various times in the entity lifecycle.

\subsection*{Core Questions}

\begin{itemize}[leftmargin=.5in]

  \item [\thesection.1] Are the data integration activities, their systems, repositories, and connections known and tracked
  \item [\thesection.2] Are data integration activities linked to data inventory, business glossaries and data models
  \item [\thesection.3] Are all data integration input and output data sets documented, tracked and governed
  \item [\thesection.4] Are there reusable standards and defined business rules for performing data integration
  \item [\thesection.5] Are data integration patterns, tools and technologies defined, governed and used
  \item [\thesection.6] Has the firm established a central data integration function (i.e. integration Center of Excellence) to manage \acrshort{etl} across both internal and external data pipelines

\end{itemize}
