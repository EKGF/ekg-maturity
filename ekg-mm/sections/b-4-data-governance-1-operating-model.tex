%
% B.4.1 Data-management Operating Model
%
\ekgmmCapability{b-4-1}{data-management-operating-model}{Data-management \Glsfmttext{operating-model}}

\ekgmmContextSection

The \gls{operating-model} for the \glsfirst{ekg:coe} establishes a new way of running the organization
to enable companies to accelerate development and operate more efficiently.
The \glsxtrshort{ekg} \iindex{governance framework}\index{\glsfmtshort{ekg}!governance framework} focuses on
combining new data capabilities (i.e. resolvable identity, standard ontologies, open standards)
as part of an integrated process.
A well defined \gls{operating-model} specifies architecture components that support flexible and reusable data to help
organizations achieve improvements in revenue, customer experience and cost.

\ekgmmcorequestionssection

\begin{core-questions}

  \item [\thesection.1] Have the underlying principles (and challenges) of data management been established and
                        accepted by involved stakeholders
  \item [\thesection.2] Does the \gls{odm} have authority to enforce adherence to policy
  \item [\thesection.3] Has the data management funding model been established and implemented
  \item [\thesection.4] Has the organization identified and recruited stakeholders with sufficient skill sets to
                        implement the data management program
  \item [\thesection.5] Are involved stakeholders held accountable to data management program deliverables
  \item [\thesection.6] Has the organization defined and validated the \glspl{operating-model},
                        and workflows necessary to implement the data management program
  \item [\thesection.7] Are metrics and KPIs captured and actively used to improve the data management operating process
  \item [\thesection.8] Is the data management \gls{operating-model} audited for compliance and effectiveness

\end{core-questions}

