%
% B.3.1 Data Quality Framework
%
\ekgmmCapability{b-3-1}{data-quality-framework}{Data Quality Framework}

\ekgmmContextSection

Data in the knowledge graph is structured around interconnectivity and precision.
The goal is assurance of granular meaning so that users have confidence they are getting all the information they
need to understand context and solve ad hoc business questions.
In a semantic environment, every datapoint is resolved to a universally unique identifier, ensuring discoverability
across repositories and domains.
Data and metadata are connected so that logical errors and data inconsistencies are detected before they enter
the system.
From a compliance perspective, data in the graph is immutable because \iindex{lineage} can be traced and
nothing can be deleted except by policy.

\ekgmmcorequestionssection

\begin{core-questions}

  \item [\thesection.1] Has the data management strategy for the organization been defined, verified and aligned
                        with the \gls{operating-model}
  \item [\thesection.2] Have business requirements been captured and linked to granular data concepts
  \item [\thesection.3] Have relevant data sources been identified and linked to \glspl{sor}
  \item [\thesection.4] Is data precisely defined and mapped to enterprise models\todo{or any semantic
                        machine-readable model i.e. ontology?}
  \item [\thesection.5] Is the governance infrastructure (roles, responsibilities, funding requirements,
                        measurement criteria) imple\-mented and operational
  \item [\thesection.6] Are communications mechanisms in place to ensure that quality issues are verified,
                        addressed at source and linked to consuming applications
  \item [\thesection.7] Are end-users getting the data they need (and able to use it without the need for
                        reconciliation or manual transformation)

\end{core-questions}

\subsection*{Concepts to Discuss}

\begin{enumerate}

  \item have business requirements been captured and verified in the form of use case trees and business user stories
  \item are all data sources identified; defined in the knowledge graph; SOR and authorized distribution points;
  \item all data in machine-readable form and linked to ontologies

\end{enumerate}
