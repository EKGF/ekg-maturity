%
% A.2.2 Value Chain
%
\ekgmmCapability{a-2-2}{value-chain}{Value Chain}
\index{value chain}
\index{digitalization}

Value chains have been a prominent way of organising and linking activities in an Enterprise to create value
for its customers, thorough provision of products and services.
Digitalization and the increasing focus on consumer-centricity\index{customer centric}
has pushed the imperatives of Value Chains to include more holistic collaborations in the
business ecosystem\,---\,reshaping provision of services and access to untapped pools of value.

\somequote{%
Most global companies are now actively considering the ecosystem business model given its value-generation potential:
growing the core business, expanding the network and portfolio, and generating revenues from new products and services.}
{McKinsey Digital}
{https://www.mckinsey.com/business-functions/mckinsey-digital/our-insights/how-do-companies-create-value-from-digital-ecosystems\#:~:text=Most\%20global\%20companies\%20are\%20now,from\%20new\%20products\%20and\%20services.}

\ekgmmContextSection

\ekgmmHowEKGRequiresThisCapability

Having the Value Chain(s) of an organization defined can help with the selection, definition and prioritization of
the right use cases for the \gls{ekg}.

\ekgmmHowEKGAffectsThisCapability

In the \gls{ekg} context, a "digital twin" of a company's value chain can be modelled where all the various components
of the value chain are represented e.q. all details around logistics, supply chains, operations, services, marketing,
sales and all support activities.
At higher levels of maturity, all real-world details of these components are available as well and represent
the reality accurately and in "real time".

We welcome your input here.

