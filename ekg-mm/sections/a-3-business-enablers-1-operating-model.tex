%
% A.3.1 Operating Model
%
\ekgmmCapability{a-3-1}{business-operating-model}{Operating Model}

At its simplest, an Operating Model enables an Enterprise to realise its vision and strategy
at the level of everyday execution\,---\,business operations.

An operating model\,---\,or lack of it\,---\,has a profound impact on how a company implements its
business processes\index{business!process} and organizes supporting technologies.
Depending on an enterprise's Business Models\index{business!model}, multiple operating models,
at corporate and unit/divisional levels, could co-exist, in pursuit of the business strategy.

In order to define an operating model at any of these different levels,
Enterprises need to answer the following two questions:

\begin{enumerate}
    \item To what extent is the successful completion of one business unit’s operations/transactions dependent on the
    availability, accuracy, and timeliness of other business unit’s information?
    \item To what extent does the enterprise benefit by having business units run their operations in the same way?
\end{enumerate}

The first question determines integration requirements while the second one covers the standardization requirements
of an Enterprise.

\begin{wrapfigure}[7]{r}{0.5\textwidth}
    \vspace{-23pt}
    \begin{center}
        \begin{tcolorbox}[colback=secondary!5,colframe=secondary!60,left=2pt,right=2pt]
            \itshape\large\enquote{%
                80\% of CEOs in one study claim to have transformations in place to make their businesses more digital;
                87\% expect to see a change in their operating models within three years.%
            }%
            \begin{flushright}\textcite{align_operating_model_to_strategy}\end{flushright}%
        \end{tcolorbox}
    \end{center}
\end{wrapfigure}

In today's modern dynamic economic environment, Enterprises have to contend with a variety of choices in the selection,
ongoing evaluation and transformation of their operating models.
Such choices are necessitated by a fast-changing competitive landscape and the accompanying need to constantly
align the workforce capacity and capabilities to business tactics.
Typical triggers of consideration include comparative metrics such as higher-than-peer Cost-to-Income Ratios
in key product portfolios, costs per transaction, Net Promoter Score/CSI, time-to-market for new products et al.

While practices around specifics of Operating Models vary across Enterprises, we highlight below a few key focus
areas in the setup and transformation of their operating models and in such a context,
how \glspl{ekg} can act as effective enablers:

\paragraph*{Integration: Effective coordination across silos of activities with disjoint goals in any
given external interaction (customer or stakeholder)}

In the Introductory section, we have introduced the idea of using Contracts as an intuitive anchor for
coordinating activities in an enterprise, towards its engagements with customers and stakeholders.
An Operating Model can leverage the context of "why" and "what" in the contracts and specify the "who" and "how"
to align and link various activities within a business unit, across business units and across enterprise boundaries.
Many enterprises are using the concept of "User/Customer Journeys" to align and link cross unit activities in their
Operating Models\,---\,such User/Customer Journeys can be considered to be instance models of Contract execution,
enabling Enterprises to assess and select appropriate features in their Operating Models.

\paragraph*{standardization: Trade-offs between Individual Efficiency and
Overall Effectiveness in the business processes}

A key challenge many enterprises face is in the choice between improving efficiencies focused on individual
business processes \myuline{and} effectiveness of the overall end-to-end business flow.
In the introductory section, we have introduced the idea of using Business Identities as intuitive anchor for
an Enterprise’s engagement with its stakeholders.
To address the trade-off challenge on individual vs collective efficiency in its business processes,
an Enterprise’s Business Identity can help in
(a) aligning the guidance of the enterprise's vision/mission/values to the trade-off and
(b) ensuring consistency in practices and tools, through standardization, in the execution of business processes.

In order to best support a company's strategy, a foundation for execution is necessary.
Such a foundation is the operating model of an organization.

See appendix \nameref{ch:operating-model} on page \pageref{ch:operating-model} for a summary of the theory around
business operating models.

\ekgmmCapabilityContributionToEnterprise{a-3-1}

\ekgmmCapabilityContributionToEKG{a-3-1}
