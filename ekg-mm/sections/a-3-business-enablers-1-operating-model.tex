%
% A.3.1 Operating Model
%
\ekgmmCapability{a-3-1}{business-operating-model}{Operating Model}

At its simplest, an Operating Model enables an Enterprise to realise its vision and strategy
at the level of everyday execution\,---\,business operations.

An operating model\,---\,or lack of it\,---\,has a profound impact on how a company implements its
business processes\index{business process} and organizes supporting technologies.
Depending on an Enterprise's Business Models\index{business model}, multiple operating models,
at corporate and unit/divisional levels, could co-exist, in pursuit of the business strategy.

In order to define an operating model at any of these different levels,
Enterprises need to answer the following two questions:

\begin{enumerate}
    \item To what extent is the successful completion of one business unit’s operations/transactions dependent on the
    availability, accuracy, and timeliness of other business unit’s information?
    \item To what extent does the enterprise benefit by having business units run their operations in the same way?
\end{enumerate}

The first question determines integration requirements while the second one covers the standardisation requirements
of an Enterprise.

\begin{wrapfigure}[8]{r}{0.5\textwidth}
    \vspace{-12pt}
    \begin{center}
        \begin{tcolorbox}[colback=secondary!5,colframe=secondary!60,left=2pt,right=2pt]
            \itshape\large\enquote{%
                80\% of CEOs in one study claim to have transformations in place to make their businesses more digital;
                87\% expect to see a change in their operating models within three years.%
            }%
            \begin{flushright}\textcite{align_operating_model_to_strategy}\end{flushright}%
        \end{tcolorbox}
    \end{center}
\end{wrapfigure}

In today's modern dynamic economic environment, Enterprises have to contend with a variety of choices in the selection,
ongoing evaluation and transformation of their operating models.
Such choices are necessitated by a fast-changing competitive landscape and the accompanying need to constantly
align the workforce capacity and capabilities to business tactics.
Typical triggers of consideration include comparative metrics such as higher-than-peer Cost-to-Income Ratios
in key product portfolios, costs per transaction, Net Promoter Score/CSI, time-to-market for new products et al.

While practices around specifics of Operating Models vary across Enterprises, we highlight below a few key focus
areas in the setup and transformation of their operating models and in such a context,
how \glspl{ekg} can act as effective enablers:

\paragraph*{Integration: Effective coordination across silos of activities with disjoint goals in any
given external interaction (customer or stakeholder)}

In the Introductory section, we have introduced the idea of using Contracts as an intuitive anchor for
coordinating activities in an enterprise, towards its engagements with customers and stakeholders.
An Operating Model can leverage the context of "why" and "what" in the contracts and specify the "who" and "how"
to align and link various activities within a business unit, across business units and across enterprise boundaries.
Many enterprises are using the concept of "User/Customer Journeys" to align and link cross unit activities in their
Operating Models\,---\,such User/Customer Journeys can be considered to be instance models of Contract execution,
enabling Enterprises to assess and select appropriate features in their Operating Models.

\paragraph*{Standardisation: Trade-offs between Individual Efficiency and
Overall Effectiveness in the business processes}

A key challenge many enterprises face is in the choice between improving efficiencies focused on individual
business processes \myuline{and} effectiveness of the overall end-to-end business flow.
In the introductory section, we have introduced the idea of using Business Identities as intuitive anchor for
an Enterprise’s engagement with its stakeholders.
To address the trade-off challenge on individual vs collective efficiency in its business processes,
an Enterprise’s Business Identity can help in
(a) aligning the guidance of the enterprise's vision/mission/values to the trade-off and
(b) ensuring consistency in practices and tools, through standardisation, in the execution of business processes.

In order to best support a company's strategy, a foundation for execution is necessary.
Such a foundation is the operating model of an organization.



%\citequote{%
%    The operating model is the necessary level of business process integration and standardization for delivering
%    goods and services to customers.
%    An operating model describes how a company wants to thrive and grow.
%    By providing a more stable and actionable view of the company than strategy,
%    the operating model drives the design of the foundation for execution.%
%}{ross_weill_robertson_2006}
%
%Defining an operating model in an organization is very important because it enables the
%adoption of different strategic initiatives as a foundation for execution.
%Such an operating model represents and reinforces the strategy to be supported.
%Therefore, the operating model\,---\,or the lack of it\,---\,has a profound impact on how a company
%implements business processes and IT.\index{business process}
%
%An operating model has two dimensions which are (1) business process standardization and (2) integration:
%
%\begin{enumerate}
%    \item According to Ross, Weil and Robertson (2006), \myuline{standardization} of business processes can be defined as;
%          \textit{“defining exactly how a process will be executed regardless of who is performing the process or
%          where it is completed.
%          Process standardization delivers efficiency and predictability across the company”}.
%          Business process standardization can lead to large increases in throughput and efficiency.
%          However, it comes at a price since the more business processes are standardized,
%          the more difficult it is to tailor services and products for specific customer needs hampering
%          innovation and business agility in the process.
%    \item Business process \myuline{integration} coordinates efforts and work across organizations through data.
%          Sharing data can occur between processes to enable end-to-end transaction processing,
%          or across processes to allow the company to present a single face to customers.
%          Some of the benefits of integration are increased efficiency, coordination, transparency, and agility.
%          However, integration entails a large amount of effort such as in end-to-end integration where companies
%          need to develop standard data definitions and formats that will be shared and used across different
%          business units and functions.
%\end{enumerate}
%
%Companies adopt an operating model at the enterprise level but could also adopt different operating models
%at the division, business region, or other level.
%In order to define an operating model at any of these different levels, companies need to answer the following
%two question made by Ross, Weil and Robertson (2006):
%
%\begin{enumerate}
%    \item \textit{To what extent is the successful completion of one business unit’s transactions dependent on the
%          availability, accuracy, and timeliness of other business unit’s data?}
%    \item \textit{To what extent does the company benefit by having business units run their operations in
%          the same way?}
%\end{enumerate}
%
%The first question determines integration requirements while the second one covers the standardization requirements
%of an organizations.
%Depending on these requirements, there are four different possible operating models as depicted in
%table~\ref{tab:ekg-mm-business-operating-model-quadrants}.
%
%%\newcommand{\tabitem}{\hspace*{0pt}~~\llap{\textbullet}~~}
%\newcommand{\tabitem}{\indent\llap{~~\textbullet~~}}
%
%\noindent\begin{table}[!htbp]
%    \centering
%    \let\freewidth\relax%
%    \newlength{\freewidth}%
%    \setlength{\freewidth}{\dimexpr (\textwidth-3em)-8\tabcolsep}%
%    \renewcommand{\arraystretch}{1.5}%
%    \begin{tabular}{
%        @{}p{0.03\freewidth}@{}
%        @{}p{0.03\freewidth}@{}
%        @{\hspace{1em}}p{0.5\freewidth}@{\hspace{1em}}
%        @{\hspace{1em}}p{0.5\freewidth}
%    }%
%        \cline{3-4}
%        \multirow{23}{2em}{\rotatebox[origin=c]{90}{\bf \Large Business Process Integration}}
%           & & \textbf{Coordination}
%             & \textbf{Unification} \\
%           &   \raisebox{-.5\normalbaselineskip}[0pt][0pt]{\rotatebox[origin=c]{90}{\bf \large High}}
%           &   \tabitem Shared customers, products or suppliers
%             & \tabitem Customers or suppliers may be local or global \\
%           & & \tabitem Impact on other business unit transactions
%             & \tabitem Globally integrated business processes often with support of enterprise systems \\
%           & & \tabitem Operationally unique business units or functions
%             & \tabitem Business units with similar or overlapping responsibilities \\
%           & & \tabitem Autonomous business management
%             & \tabitem Centralized management often applying functional/process/business unit matrix\\
%           & & \tabitem Business unit control over process design
%             & \tabitem High level process owners design standardized processes \\
%           & & \tabitem Shared customer/supplier data
%             & \tabitem Centrally mandated database \\
%           & & \tabitem Consensus process for design of IT infrastructure services
%             & \tabitem IT decisions made centrally \\
%           & & \tabitem IT application decisions made in business units & \\ [1em]
%        \cline{3-4}
%           & & \textbf{Diversification}
%             & \textbf{Replication} \\
%           & & \tabitem Few if any shared customers or suppliers\index{customer}\index{supplier}
%             & \tabitem Few if any shared customers or suppliers \\
%           & & \tabitem Independent transactions
%             & \tabitem Independent transactions are aggregated at high level \\
%           & & \tabitem Operationally unique business units
%             & \tabitem Operationally similar business units \\
%           & & \tabitem Autonomous business management
%             & \tabitem Autonomous business unit leaders with limited direction over processes \\
%           & & \tabitem Business unit control over process design
%             & \tabitem Centralized control over business process design \\
%           & & \tabitem Few data standards across business units
%             & \tabitem Standardized data definitions, but data locally owned with some aggregation at corporate level \\
%           & \raisebox{.5\normalbaselineskip}[0pt][0pt]{\rotatebox[origin=c]{90}{\bf \large Low}}
%           &   \tabitem Most IT decisions made within business units
%             & \tabitem Centrally mandated IT services \\
%        \cline{3-4}
%           & & {\bf \large Low} & \hfill {\bf \large High} \\
%           & & \multicolumn{2}{c}{\bf \Large Business Process Standardization} \\
%           & & & \hfill {\footnotesize (c)2005 MIT Sloan school of business} \\
%    \end{tabular}
%    \caption{Operating Model Quadrant}\label{tab:ekg-mm-business-operating-model-quadrants}
%\end{table}%
%
%Each one of these four operating model types is described below:
%
%\begin{basedescript}{%
%    \desclabelstyle{\multilinelabel}
%    \desclabelwidth{2.6cm}
%}
%    \item[Diversification] low standardization, low integration
%    \item[Coordination] low standardization, high integration
%    \item[Replication] high standardization, low integration
%    \item[Unification] high standardization, high integration
%\end{basedescript}
%
%\section*{Diversification}
%
%Diversification applies to companies that have different units with few common products, services, customers,
%or ways of doing business.
%
%Central management exerts relatively little control over business units that operate in a highly autonomous way
%offering their own products and services to their own customers.
%
%The Diversification model may offer synergies from related, but not integrated, business units.
%In this context, business units may create demand for one another or increase the company’s brand recognition
%which creates enterprise-wide value.
%Although there could be some synergies between business units, the success of companies with a Diversification model
%stem from the success of the individual business units and acquisitions of other related businesses.
%
%\section*{Coordination}
%
%Business units in a Coordination company usually share one of the following:
%customers, products, suppliers, and partners.
%Some of the benefits of the Coordination model are integrated customer service,
%cross-selling and transparency across supply chain processes.
%Whereas key business processes are tightly integrated, business units have unique processes and capabilities.
%
%In these companies, low cost is not the main driver as the main driver is delivering the best service and products
%to the customer while executing business processes in the most efficient way possible.
%Strong central management defines and prioritizes cooperation.
%
%Through integration without a high degree of standardization across business units or functions,
%growth can be achieved by offering already existing products or services to customer segments in new markets.
%Additionally, growth can also be achieved by improving services to meet new, but related, customer demands.
%
%\section*{Replication}
%
%The Replication model provides autonomy to business units but runs operations in a highly standardized way.
%The business units are not tightly integrated as they are not dependent on each other’s transactions while they
%implement a set of highly standardized business processes that can be easily repeated in new business units.
%McDonald’s\index{company!McDonald's} and other franchise companies are examples of a Replication company.
%The advantage of this model is that it enables organizations to build new business units from scratch
%with relatively little effort.
%
%\section*{Unification}
%
%Companies that operate as a highly optimized whole\,---\,around a highly standardized set of business processes\,---\,%
%may benefit from the Unification model.
%Business units in these companies may have relatively very little autonomy and they best maximize efficiencies and
%customer services by using integrated data and driving variability out of business processes.
%
%Unification companies typically have integrated supply chains with interdependencies between distributed business units.
%These business units share transaction data and standardized business processes.
%Therefore, these companies may benefit the most from enterprise-wide systems to support company standardization
%and integration requirements.
%
%Management in these companies is highly centralized and plays an important role in driving out inefficiencies to
%foster growth through economies of scale by introducing new products.
%Since variability must be minimized in these companies,
%this model is best suited for companies that compete on price such as those that offer commodities where
%innovation or customization are not key.
%
%\section*{Applying the operating model to attain growth}
%
%An operating model is the underlying logic of how an organization will enable and execute strategies.\improve{%
%  We should make the link here to \ref{sec:ekg-mm-business-goals} \nameref{sec:ekg-mm-business-goals} and
%  \ref{sec:ekg-mm-business-tactics} \nameref{sec:ekg-mm-business-tactics}
%}
%Each operating model entails its own opportunities for growth.
%
%The need to tightly integrate business processes make acquisitions and mergers\,---\,for both the buy-side as well
%as the sell-side\,---\,more challenging as disparate data definitions need to be reconciled.
%However, the tight process integration of the Coordination and Unification models offers opportunities of
%organic growth through expansion into new markets or extensions of current product lines.
%
%Process standardization, as in the Unification and Replication models, enables growth through a
%rip-an-replace approach to acquisitions.
%When an acquiring company wants to create a mirror image of itself out of an acquired company,
%it only has to replace the processes and systems of the acquired business with its own.
%However, both models do not offer much leverage when a company chooses to expand into operationally
%distinct lines of business as both models depend on leveraging already existing processes.
%
%The Diversification model imposes fewer constraints on the organic growth of individual business units and
%fewer limits for business acquisitions.
%However, it does not offer the benefits of integration and standardization across business units.
%The opportunities for growth of each operating model are shown in
%table~\ref{tab:ekg-mm-business-operating-model-different}.
%
%\noindent\begin{table}[!htbp]
%             \centering
%             \let\freewidth\relax%
%             \newlength{\freewidth}%
%             \setlength{\freewidth}{\dimexpr (\textwidth-3em)-8\tabcolsep}%
%             \renewcommand{\arraystretch}{1.5}%
%             \begin{tabular}{
%                 @{}p{0.04\freewidth}@{}
%                 @{}p{0.03\freewidth}@{}
%                 @{\hspace{1em}}p{0.5\freewidth}@{\hspace{1em}}
%                 @{\hspace{1em}}p{0.5\freewidth}
%             }%
%                 \cline{3-4}
%                 \multirow{9}{2em}{\rotatebox[origin=c]{90}{\bf \Large Business Process Integration}}
%                 & & \textbf{Coordination}
%                 & \textbf{Unification} \\
%                 &   \raisebox{-.5\normalbaselineskip}[0pt][0pt]{\rotatebox[origin=c]{90}{\bf \large High}}
%                 &   \tabitem \textbf{Organic:}
%                 stream of product innovations easily made available to existing customers
%                 using existing integrated channels.
%                 &   \tabitem \textbf{Organic:}
%                 leverage economies of scale by introducing existing products/services in new markets;
%                 grow product line incrementally. \\
%                 & & \tabitem \textbf{Acquisition:}
%                 can acquire new customers for existing products but must integrate data.
%                 &   \tabitem \textbf{Acquisition:}
%                 can acquire competitors to leverage existing foundation;
%                 must rip and replace infrastructure. \\ [1em]
%                 \cline{3-4}
%                 & & \textbf{Diversification}
%                 & \textbf{Replication} \\
%                 & & \tabitem \textbf{Organic:}
%                 small business units may feed core business;
%                 company grows through business unit growth
%                 &   \tabitem \textbf{Organic:}
%                 replicate best practices in markets;
%                 innovations extended globally \\
%                 & & \tabitem \textbf{Acquisition:}
%                 unlimited opportunities;
%                 must ensure shareholder value
%                 &   \tabitem \textbf{Acquisition:}
%                 can acquire competitors to expand market reach;
%                 must rip and replace \\ [1em]
%                 \cline{3-4}
%                 & \raisebox{1.5\normalbaselineskip}[0pt][0pt]{\rotatebox[origin=c]{90}{\bf \large Low}}
%                 & {\bf \large Low} & \hfill {\bf \large High} \\
%                 & & \multicolumn{2}{c}{\bf \Large Business Process Standardization} \\
%                 & & & \hfill {\footnotesize (c)2005 MIT Sloan Center for Information Systems Research.} \\
%             \end{tabular}
%             \caption{Different operating models position companies for different types of growth}
%             \label{tab:ekg-mm-business-operating-model-different}
%\end{table}%
%\index{acquisition}
%\index{product!line}
%\index{product!innovation}
%\index{data!integration}
%\index{rip and replace}
%\index{competition}
%
%Business units of a company can also adopt different operating models to respond to conflicting demands.
%For instance, Diversification companies may benefit from allowing their own business units to adopt their
%own operating models as these business units are highly independent.
%An example of a Diversification company whose business units adopted different operating models is
%Johnson \& Johnson.\index{company!Johnson \& Johnson}
%J\&J’s U.S. pharmaceutical group applies a Coordination operating model in which there is a single touchpoint
%with health-care practitioners while their subsidiary Janssen Pharmaceuticals\index{company!Janssen Pharmaceuticals}
%implements a Replication operating model in Europe with highly standardized, low cost processes.
%This gives freedom to each business unit to implement a different operating model depending on its
%own objectives while maintaining a relatively simple operating model at a corporate level.
%
%An operating model gives a company better guidance for developing IT and business process capabilities.
%It also serves as a stable foundation for strategic endeavours such as mergers and acquisitions.
%\index{mergers \& acquisitions}
%This foundation enables IT to be more proactive in identifying possible strategic opportunities.\improve{%
%    make the link to \ref{sec:ekg-mm-technology-strategy} \nameref{sec:ekg-mm-technology-strategy}
%}
%In order to define an operating model, management needs to define the role of business process standardization
%and integration.\index{business process!standardization}\index{business process!integration}
%This also requires management to identify the company’s key business processes that create a
%sustainable competitive advantage for the company.\index{competitive advantage}
%As a result, an operating model offers a company the possibility to create and possess reusable capabilities
%for long-term growth.
%In this context, an operating model could be seen as the main driver of strategy at a corporate or business level.
%In addition, an operating model plays a major role in defining the required architecture, practices,
%management thinking, policies, and processes as they may be different for each operating model.
%In other words, an operating model could be a key driver in the design of separate organization units.

\ekgmmContextSection

We welcome your input here.
