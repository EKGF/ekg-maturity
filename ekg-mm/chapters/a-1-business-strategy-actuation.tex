\chapter{Business Strategy Actuation}% A.1 Business Strategy Actuation
\label{ch:ekg-mm-a-1}
\label{ch:ekg-mm-business-strategy-actuation}
\index{business!strategy}
\index{business!strategy!actuation}
\index{actuation}

\textbf{\nameref{ch:ekg-mm-business-strategy-actuation}} is the process of identifying measurement values\,---\,e.g.
speed, time\,---\,for planned object change activities\,---\,e.g. add, remove, move\,---\,that will provide an
accurate way of assessing whether strategic objectives and associated business goals have been fulfilled.
\index{business!goals}

\textit{Actuation}\index{actuation} is an inspired term\,---\,in common use, it denotes the action of causing a
machine or device to start.
In the business world, \textit{Business Strategy} is about intent, and \textit{Strategy Actuation}
can be seen as activating that intent.
The broad business imperatives of \glsfirst{suda} for a modern, forward-looking Enterprise,
based on the "siren call" of its \textit{Business Identity}, have been established in the introductory section on page
\pageref{sec:ekg-mm-business-identity}.

This section suggests mechanisms wherein \gls{ekg} can help Businesses in aligning their Strategy Actuation to
the \gls{suda} imperatives.
While practices around Business Strategy vary across Enterprises (for historical reasons and management preferences),
they broadly cover the following key elements of Business Intent and its cascade within the Enterprise:

A \textbf{\nameref{sec:ekg-mm-business-vision}}\,---\,which communicates the essence of what a Business wants to
achieve\,---\,such as:
\begin{itemize}
  \item \textit{Create a better everyday life for many people}\,---\,%
        \href{https://www.ikea.com/gb/en/this-is-ikea/about-us/vision-and-business-idea-pub9cd02291}{IKEA}\index{ikea}
  \item \textit{Bring Inspiration and Innovation to every Athlete}\,---\,%
        \href{https://www.nike.com/gb/help/a/nikeinc-mission}{Nike}\index{nike}
  \item \textit{Create the most compelling car company of the 21st century\newline by driving the world's
        transition to electric vehicles}\,---\,%
        \href{https://visionarybusinessperson.com/tesla-mission-statement/}{Tesla}
\end{itemize}

An Enterprise’s Vision and Mission have been mechanisms for easy retention and recall within its rank and file\,---\,%
for guidance and motivation in activities\,---\,as well as its public image.
As communications and activities of Enterprises become more digital, leveraging a mix of human and machine scale
capabilities, \glspl{ekg} can help by enabling Enterprises to align the Vision and Mission in the context of activities,
through the Business Identity.
Importantly, the Business Identity anchored \gls{ekg} can provide a contextual lens for articulation,
validation and justification of Business Vision at any level of activity of an Enterprise\,---\,within as well as
across its boundaries.

\textbf{\nameref{sec:ekg-mm-business-goals}}, which specify the outcomes along the journey to achieve the
Business Vision.

Strategic Goals and their articulation have been human-facing primarily, with guidance and governance mechanisms
taking forms such as periodic reviews, monitoring and reporting.
While tools and workflows around SMART (Specific, Measurable, Achievable, Realistic, Timely) principles have
rendered more objectivity in the process, significant judgement is still required in assessing progress towards
Strategic Goals.
Such judgement involves a more holistic assessment of the progress taking a balanced view across objective measurements
and evidences of impact.
\Glspl{ekg} can help by enabling Enterprises align expectations, evidences and business metrics both in achieving
and assessing progress on Strategic Goals.

Carl Mattock>\textbf{\nameref{sec:ekg-mm-business-tactics}} are the actions performed that ensure
business change capabilities are focused on Business value realization.
To determine that added value is created by a particular business change, the key factors used for metric generation
can be made explicit as an EKG .

Avinash Patil>\textbf{\nameref{sec:ekg-mm-business-tactics}}, which specify the actions to focus and align
business activities and pursuits to the realisation of the Business Goals Practices around Agility\index{agile} in
business change, as well as real-time focus on monitoring of key internal and external \gls{bau} metrics have enabled
Enterprises to make frequent tactical corrections in their activities.
A significant component of such corrections is a judgement that relies on human \iindex{intuition}, \iindex{bias}
and \iindex{expertise}, guided by associated business metrics.
\Glspl{ekg} can help by enabling Enterprises to make faster tactical decisions, by providing an integrated view to
consistently reason more effectively around metrics.

To summarise, \textbf{\nameref{ch:ekg-mm-business-strategy-actuation}} is a business identity oriented process
that senses, understands and communicates \iindex{business outcome}-based assessment data to focus energy and
resources on strategic objectives.
Actuation measurement values are used to assess whether strategic objectives and associated business goals are
being fulfilled, such as:

\begin{itemize}
    \item Speed of awareness\,---\,%
          which contractual commitments will benefit from an enhanced ability to assess and respond?
    \item Speed of understanding impacts\,---\,%
          which commitments require a focus on demonstrating suitability through impact analysis?
    \item Speed of adapting to one or more changes\,---\,%
          which commitments are most at risk on the speed of adapting to changes?
    \item Speed of change\,---\,%
          which propositions, and their contractual commitments, will benefit from an enhanced speed of change?
    \item Ability to identify errors and understand how their impacts\,---\,%
          how do exceptions relate to commitments and how does exception management affect commitments?
    \item Ability to measure error propagation and assess scale
    \item Ability to manifest change to respond to errors
\end{itemize}

The \nameref{ch:ekg-mm-a-1} component has the following capabilities:

\begin{itemize}[leftmargin=.5in]
  \item [\ref{sec:ekg-mm-a-1-1}] \nameref{sec:ekg-mm-a-1-1}\,---\,Linking \gls{ekg} to strategic business and organizational objectives.
  \item [\ref{sec:ekg-mm-a-1-2}] \nameref{sec:ekg-mm-a-1-2}\,---\,Prioritization of strategic use cases and financial models.
  \item [\ref{sec:ekg-mm-a-1-3}] \nameref{sec:ekg-mm-a-1-3}\,---\,Incremental steps to address the challenges of data fragmentation.
  \item [\ref{sec:ekg-mm-a-1-4}] \nameref{sec:ekg-mm-a-1-4}\,---\,Performance-improving journeys that cut through organizational silos.
\end{itemize}

\section{Business Vision}% A.1.1 Business Vision
\label{sec:ekg-mm-a-1-1}
\label{sec:ekg-mm-business-vision}
\index{vision}
\index{business!vision}

The \nameref{sec:ekg-mm-business-vision} communicates the essence of what a Business wants to achieve.

We welcome your input here.

\ekgmmContextSection

We welcome your input here.

\section{Business Goals}\label{sec:ekg-mm-a-1-2}

We welcome your input here.

Example objectives:\footnote{Taken from a public presentation at Enterprise Data World '21 of LGT Private Banking
called "Enterprise Knowledge Graph in action at LGT"}

\begin{itemize}
    \item Stay Competitive.
          \begin{itemize}
              \item Comprehensive, client-focused services.
              \item Outstanding expertise and service quality.
          \end{itemize}
    \item Increase agility reacting to or delivering on Risk \& Regulatory requirements.\index{regulator}
          \begin{itemize}
              \item Comply with ethical, data privacy and security guidelines.
          \end{itemize}
    \item Drastically decrease \enquote{time to market} of the implementation and realization of new business ideas from
          inception to production.
          \begin{itemize}
              \item Offer more tailor-made services to each client.
          \end{itemize}
    \item Silos are holding us back: lack of \enquote{360 degree views} and not having the full picture is a
          drag on competitiveness, no more (new) silos.\index{silo}
\end{itemize}

\ekgmmContextSection

We welcome your input here.

\section{Business Tactics}\label{sec:ekgmm-a-1-3}

We welcome your input here.

\ekgmmContextSection

We welcome your input here.

\section[Operating Model]{Business \Glsfmttext{operating-model}}
\label{sec:ekg-mm-a-1-4}
\label{sec:ekg-mm-business-operating-model}

Performance-improving journeys that cut through organizational silos.

See \href{https://www.mckinsey.com/~/media/mckinsey/business%20functions/mckinsey%20digital/our%20insights/introducing%20the%20next-generation%20operating%20model/introducing-the-next-gen-operating-model.ashx}{McKinsey: Introducing the next-generation Operating Model}.

See also \secref{sec:ekg-mm-b-4-1}.

\ekgmmContextSection

We welcome your input here.

