\chapter*{Introduction \& Background}

\section*{The Business Case for Explorative Execution}

Business performance\index{business!performance} is traditionally measured with \iindex{metrics} emphasising
\iindex{financial performance}\,---\,typically, aspects of \iindex{revenue}, \iindex{profit}, growth and \iindex{risk}.
Mechanisms such as Balanced Scorecards\index{balanced scorecard} help enterprises align strategy execution to their
business performance, emphasising the causal chains coursing through organisational capability, operations and
customer propositions, eventually landing into financial performance as an outcome.
More recently, there has been an emphasis on key stakeholder metrics such as \glsxtrshort{esg} adding further
dimensions to responsible business performance.

The paradigm of business performance has largely hinged on organisations directly influencing their customer
engagements with their products and services.
Indeed, the prevalent use of the term "customer" is ingrained as a counterpart to the organisations’
products and services.
Dramatic advances in the digital economy, especially in the area of mobile and sensor technology over the last 15 years,
have highlighted the perception of value though insights into consumption/use.
In other words, the presumed value in organisations’ products and services is increasingly validated through
actual insights into consumption/use by consumers.

This shift in perspective of an organizations’ products and services as patterns (based on insights) of consumer use
is where movements such as \iindex{design thinking}, \iindex{customer journey maps} etc are increasingly considered
as important tools for developing and validating propositions.
The "aha" and "viola" moments from such exercises require further translation into realising the propositions,
which is where practices around agility, DevOps\index{DevOps} etc are increasingly mainstream.

All these themes and mechanisms of strategy execution through communication, product/service propositions,
development/deployment and its performance management largely depend on a human capital based on participation,
expertise and intuition.
Organisations wishing to amplify their human capital face few key dichotomies to be addressed:
\begin{itemize}
    \item \textbf{Customer vs Consumer}: Mindset and capability shift from ``Customer market share'' to
          ``share of (end) consumer value''
    \item \textbf{Enterprise vs Ecosystem}: Mindset and capability shift from
          “Proprietary control of customer experiences” to “open collaboration on consumer experiences”
    \item \textbf{Human Machine Continuum}: Mindset shift from Human scaled to machine scaled
          (intelligent and autonomous) business capability
\end{itemize}
Each of the above dichotomies highlight the limitations enterprises face today in extending business capability
beyond their organisational boundaries, be it in sourcing information beyond their own channels or collaborating
on shared experiences.

Addressing these challenges requires enterprises to adopt an exploratory style in executing their strategy,
which specifically focuses on sensing and understanding the consumer ecosystem better before taking decisions
and acting (to be aligned to hyperpersonalization, both in retail and business consumer scenarios)

To put this “Explorative Execution” into perspective, a useful mental model is \gls{suda}
(a slight variant of John Boyd’s OODA loop),
specifically focused on responding/acting in a dynamic evolving environment.
The choice of the mental model is to bring out its contrast with the imperative style mental model of \gls{pdca},
usually used for acting in an environment of more certainty and predictability.

%\begin{itemize}[leftmargin=1in,font=\bfseries]
\begin{basedescript}{%
    \desclabelstyle{\multilinelabel}
    \desclabelwidth{3cm}
}
    \item[SENSE] Accessing actual contents of need and use by consumers easily overwhelms most organisations
    for a variety of reasons, chief amongst which is the sheer volume of context-unique interactions which
    must be checked for links to an organisation’s business interests and intents.
    Most of this is practiced subjectively and accorded to human expertise and intuition.
    \item[UNDERSTAND] Unlike the carefully curated lens of its products \& services,
    inputs of interest for an organisation are generally ambiguous requiring significant efforts to
    disambiguate and synthesise.
    (John Boyd’s Orientation isn’t just a state you’re in; it’s a process. You’re always orienting.)
    \item[DECIDE] Disambiguated information still isn’t directly actionable as there is a significant element of
    hypothesis embedded within.
    This is especially challenging with high prevalence of conflicting hypothesis in many cases.
    \item[ACT] Filtered sensing, hypothesis-based understanding and scenario-driven decision making finally
    create an actionable context, which the organisational operations can execute.
\end{basedescript}

\textit{(List of \gls{ekg} principles, without mentioning \gls{ekg} itself,
    framed as key requirements traced to \gls{suda} narrative in a consumer ecosystem).}

A deeper analysis of the \gls{suda} loop of “Explorative Execution” bring into focus the following requirements of capability:

%\begin{itemize}[leftmargin=1in,font=\bfseries]
\begin{basedescript}{%
    \desclabelstyle{\multilinelabel}
    \desclabelwidth{3cm}
}
    \item[Identity] how can sensing be practiced in a subject domain of multiple identities?
    How can a shift from form-based to a function-based identity be enabled?
    \item[Self-describing] how can understanding be practiced in a subject domain with multiple descriptions?
    How can a shift from structure-based description to a behaviour-based description be enabled?
    \item[Open world ambiguity] how can decision-making be practiced in scenarios which require constant
    revalidation and conflict resolution, as new facts become known?
\end{basedescript}

\textit{(Framing \gls{ekg} as a enterprise capability for enabling explorative execution)}

An \gls{ekg} is a key organisational capability to enable Explorative Execution using a \gls{suda} mental model,
though the following core features:

%\begin{itemize}[leftmargin=1in,font=\bfseries]
\begin{basedescript}{%
    \desclabelstyle{\multilinelabel}
    \desclabelwidth{2.6cm}
}
    \item[SENSE]
        \begin{itemize}
            \item Co-existence of multiple identifications based on context \& interpretation instead of
                  the prevalent content based identification
            \item How do retail consumers and business consumers use products and services?
        \end{itemize}
    \item[UNDERSTAND]
        \begin{itemize}
            \item Emphasis on Meaning through association rather than static descriptions
            \item Enhanced collaboration with ecosystem partners through standards on context sharing
                  rather than lengthy negotiations on content specifics
            \item How do the products and services address needs of consumers?
        \end{itemize}
    \item[DECIDE]
        \begin{itemize}
            \item Significant scale in knowledge driven operations through autonomous reasoning
            \item Leverage of existing (diverse and federated) content sources instead of
                  separate curated content repositories
            \item How can products and services add more value through effective use?
            \item What changes can be made to make products and services more relevant
                  for (retail and business) consumers?
        \end{itemize}
\end{basedescript}

TODO: List of \gls{ekg} principles, without mentioning \gls{ekg} itself,
framed as key requirements traced to \gls{suda} narrative in a consumer ecosystem

TODO: Framing \gls{ekg} as a enterprise capability for enabling explorative execution