\chapter*{Introduction \& Background}

\section*{The Business Case for Explorative Execution}

\iindex{Business performance} is traditionally measured with \iindex{metrics} emphasising \iindex{financial performance}\,---\,%
typically, aspects of \iindex{revenue}, \iindex{profit}, growth and \iindex{risk}.
Mechanisms such as Balanced Scorecards\index{balanced scorecard} help enterprises align strategy execution to their
business performance, emphasising the causal chains coursing through organisational capability, operations and
customer propositions, eventually landing into financial performance as an outcome.
More recently, there has been an emphasis on key stakeholder metrics such as ESG adding further dimensions to
responsible business performance.

The paradigm of business performance has largely hinged on organisations directly influencing their customer
engagements with their products and services.
Indeed, the prevalent use of the term “customer” is ingrained as a counterpart to the organisations’
products and services.
Dramatic advances in the digital economy, especially in the area of mobile and sensor technology over the last 15 years,
have highlighted the perception of value though insights into consumption/use.
In other words, the presumed value in organisations’ products and services is increasingly validated through
actual insights into consumption/use by consumers.

This shift in perspective of an organizations’ products and services as patterns (based on insights) of consumer use
is where movements such as \iindex{design thinking}, \iindex{customer journey maps} etc are increasingly considered
as important tools for developing and validating propositions.
The ‘aha’ and ‘viola’ moments from such exercises require further translation into realising the propositions,
which is where practices around agility, DevOps\index{DevOps} etc are increasingly mainstream.

All these themes and mechanisms of strategy execution through communication, product/service propositions,
development/deployment and its performance management largely depend on a human capital based on participation,
expertise and intuition.
Organisations wishing to amplify their human capital face few key dichotomies to be addressed:
\begin{itemize}
    \item \textbf{Customer vs Consumer}: Mindset and capability shift from “Customer market share” to
          “share of (end) consumer value”
    \item \textbf{Enterprise vs Ecosystem}: Mindset and capability shift from
          “Proprietary control of customer experiences” to “open collaboration on consumer experiences”
    \item \textbf{Human Machine Continuum}: Mindset shift from Human scaled to machine scaled
          (intelligent and autonomous) business capability
\end{itemize}
Each of the above dichotomies highlight the limitations enterprises face today in extending business capability
beyond their organisational boundaries, be it in sourcing information beyond their own channels or collaborating
on shared experiences.

Addressing these challenges requires enterprises to adopt an exploratory style in executing their strategy,
which specifically focuses on sensing and understanding the consumer ecosystem better before taking decisions
and acting (to be aligned to hyperpersonalization, both in retail and business consumer scenarios)

TODO: SUDA narrative as a variation on John Boyd’s OODA loop

TODO: List of EKG principles, without mentioning EKG itself, framed as key requirements traced to SUDA narrative
in a consumer ecosystem

TODO: Framing EKG as a enterprise capability for enabling explorative execution