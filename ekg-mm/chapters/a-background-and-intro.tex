%\chapter*{Introduction \& Background}

The modern economy is often characterised as an Information Economy\index{information economy} or more specifically,
Knowledge Economy.\index{knowledge economy}
Seven of the top 10 firms globally by market capitalisation (Apple\index{company!apple}, Amazon\index{company!amazon},
Microsoft\index{company!microsoft}, Alphabet\index{company!alphabet}\index{company!google},
Facebook\index{company!facebook}, Tencent\index{company!tencent}, Alibaba\index{company!alibaba}),
are representative of this characterisation.
Interestingly, these seven do not figure in the list of the top 10 firms globally,
by revenue (though they are closing the gap!).
This underscores the broad expectation that future competitiveness is increasingly focused on embedding knowledge
as a key constituent of products, services and activities in business models of Enterprises.

What does embedding knowledge in products and services mean for Enterprises?
Prevalent characterisation of some business models offers some clue:
\begin{enumerate}[label=(\alph*)]
    \item Embedded Finance\,---\,the emphasis that convenience of financial products is increasingly determined not
          from clarity of information but simplifying use by minimising required knowledge.
    \item Subscription Services\,---\,the emphasis that utility of services is increasingly determined not from
          clarity of individual service but through similarity of use across a range of related goods and services
    \item Platform Services\,---\,the emphasis that experience of goods and services is increasingly determined not
          from broad contours of use over time but by specific contexts of use at any given time
\end{enumerate}

In the following sections, we explore a few illustrative capabilities that enable Enterprises to leverage and embed
knowledge in their propositions, products \& services, planning and activities.

\section*{The Business Case for Explorative Execution}

Business performance\index{business!performance} is traditionally measured with \iindex{metrics} emphasising
\iindex{financial performance}\,---\,typically, aspects of \iindex{business!revenue}, \iindex{business!profit},
growth\index{business!growth} and risk\index{business!risk}.
Mechanisms such as Balanced Scorecards\index{balanced scorecard} help enterprises align \iindex{strategy execution}
to their business performance, emphasising the causal chains coursing through organizational capability, operations and
customer propositions, eventually landing into financial performance as an outcome.
More recently, there has been an emphasis on key stakeholder metrics such as \glsxtrshort{esg} adding further
dimensions to responsible business performance.

The paradigm of business performance has largely hinged on organisations directly influencing their customer
engagements with their products and services.
Indeed, the prevalent use of the term "customer"\index{customer} is ingrained as a counterpart to the organisations’
products\index{business!products} and services\index{business!services}.
Dramatic advances in the digital economy, especially in the area of mobile and sensor technology over the last 15 years,
have highlighted the perception of value though insights into consumption/use.
In other words, the presumed value in organizations’ products and services is increasingly validated through
actual insights into consumption/use by consumers.

This shift in perspective of an organizations’ products and services as patterns (based on insights) of consumer use
is where movements such as \iindex{design thinking}, \iindex{customer journey maps} etc are increasingly considered
as important tools for developing and validating propositions.
The "aha" and "viola" moments from such exercises require further translation into realising the propositions,
which is where practices around agility, DevOps\index{DevOps} etc are increasingly mainstream.

All these themes and mechanisms of strategy execution through communication, product/service propositions,
development/deployment and its performance management\index{business!performance}\index{performance management}
largely depend on a human capital based on participation, expertise and intuition.
Organisations wishing to amplify their human capital face few key dichotomies to be addressed:

\begin{itemize}
    \item \textbf{Customer vs Consumer}: Mindset and capability shift from ``Customer market share'' to
          ``share of (end) consumer value''
    \item \textbf{Enterprise vs Ecosystem}: Mindset and capability shift from
          “Proprietary control of customer experiences” to “open collaboration on consumer experiences”
    \item \textbf{Human Machine Continuum}: Mindset shift from Human scaled to machine scaled
          (intelligent and autonomous) business capability
\end{itemize}
Each of the above dichotomies highlight the limitations enterprises face today in extending business capability
beyond their organisational boundaries, be it in sourcing information beyond their own channels or collaborating
on shared experiences.

Addressing these challenges requires enterprises to adopt an exploratory style in executing their strategy,
which specifically focuses on sensing and understanding the consumer ecosystem better before taking decisions
and acting (to be aligned to hyperpersonalization, both in retail and business consumer scenarios)

To put this “Explorative Execution” into perspective, a useful mental model is \gls{suda}
(a slight variant of John Boyd’s OODA loop),
specifically focused on responding/acting in a dynamic evolving environment.
The choice of the mental model is to bring out its contrast with the imperative style mental model of \gls{pdca},
usually used for acting in an environment of more certainty and predictability.

%\begin{itemize}[leftmargin=1in,font=\bfseries]
\begin{basedescript}{%
    \desclabelstyle{\multilinelabel}
    \desclabelwidth{3cm}
}
    \item[SENSE] Accessing actual contents of need and use by consumers easily overwhelms most organisations
    for a variety of reasons, chief amongst which is the sheer volume of context-unique interactions which
    must be checked for links to an organisation’s business interests and intents.
    Most of this is practiced subjectively and accorded to human expertise and intuition.
    \item[UNDERSTAND] Unlike the carefully curated lens of its products \& services,
    inputs of interest for an organisation are generally ambiguous requiring significant efforts to
    disambiguate and synthesise.
    (John Boyd’s Orientation isn’t just a state you’re in; it’s a process. You’re always orienting.)
    \item[DECIDE] Disambiguated information still isn’t directly actionable as there is a significant element of
    hypothesis embedded within.
    This is especially challenging with high prevalence of conflicting hypothesis in many cases.
    \item[ACT] Filtered sensing, hypothesis-based understanding and scenario-driven decision making finally
    create an actionable context, which the organisational operations can execute.
\end{basedescript}

A deeper analysis of the \gls{suda} loop of “Explorative Execution” bring into focus the following requirements of capability:

\begin{basedescript}{%
    \desclabelstyle{\multilinelabel}
    \desclabelwidth{3cm}
}
    \item[Identity] how can sensing be practiced in a subject domain of multiple identities?
    How can a shift from form-based to a function-based identity be enabled?
    \item[Self-describing] how can understanding be practiced in a subject domain with multiple descriptions?
    How can a shift from structure-based description to a behaviour-based description be enabled?
    \item[Open world ambiguity] how can decision-making be practiced in scenarios which require constant
    revalidation and conflict resolution, as new facts become known?
\end{basedescript}

\myuline{\textit{An \gls{ekg} is a key organisational capability to enable Explorative Execution using a \gls{suda} mental model}},\\
through the following core features:

\begin{basedescript}{%
    \desclabelstyle{\multilinelabel}
    \desclabelwidth{2.6cm}
}
    \item[SENSE]
        \begin{itemize}
            \item Co-existence of multiple identifications based on context \& interpretation instead of
                  the prevalent content based identification
            \item How do retail consumers and business consumers use products and services?
        \end{itemize}
    \item[UNDERSTAND]
        \begin{itemize}
            \item Emphasis on Meaning through association rather than static descriptions
            \item Enhanced collaboration with ecosystem partners through standards on context sharing
                  rather than lengthy negotiations on content specifics
            \item How do the products and services address needs of consumers?
        \end{itemize}
    \item[DECIDE]
        \begin{itemize}
            \item Significant scale in knowledge driven operations through autonomous reasoning
            \item Leverage of existing (diverse and federated) content sources instead of
                  separate curated content repositories
            \item How can products and services add more value through effective use?
            \item What changes can be made to make products and services more relevant
                  for (retail and business) consumers?
        \end{itemize}
\end{basedescript}

\section*{Business Identity}
\label{sec:ekg-mm-business-identity}

\nameref{sec:ekg-mm-business-identity} is a statement of a company's unique innovation, service, and/or product based
value proposition.\index{value proposition}
A \nameref{sec:ekg-mm-business-identity}, published to the world via business identifier unique \glsxtrlongpl{ekg},
may transparently use a data infrastructure that can support many distinct business activities.

An enterprise could have multiple Business Identities based on the value propositions underlying its
products and services\,---\,it is imperative that each must be perceived (as having differentiators) and
consumed (as unique offerings) by its customers and stakeholders.

\begin{itemize}
    \item The \textbf{Sense} imperative is addressed by using the Business Identity\index{business!identity}
          as an “attractor” amidst the voluminous stream of interactions.
          This is especially important in a Privacy conscious environment where interest in and intent of using
          information is clearly understood and agreed upon upfront.
    \item The \textbf{Understand} imperative is addressed by using the Business Identity as a common link between the
          complementary mechanisms of (a) fulfilment within an enterprise and (b) the collaboration contexts with
          the partners.
    \item The \textbf{Decide} imperative is addressed by using the Business Identity as an enterprise persona for scenario
          planning and evaluation - rather than viewing an enterprise as a structural entity.
\end{itemize}

Business Identity\index{business!identity} enables an (\gls{ekg} powered) enterprise to constantly assess and align its value propositions
both to

\begin{itemize}
    \item[(a)] needs and use contexts of consumers and
    \item[(b)] collaboration mechanisms with business partners for effective fulfilment.
\end{itemize}

\paragraph*{Aligning Business Identity with Activities in an Enterprise}\index{business!identity}

Enterprises use a variety of mechanisms today to organise “inventories” of their activities.
The artefacts take a variety of forms ranging from free-form documentation of manuals,
data and process models along different methodologies, data records and programming repositories.
Business Capability Models, Business Canvas, Business Reference Architectures etc have been some of the
methodologies used to serve as Organization-wide references of activities.

The objectives guiding such management are mostly around efficiency of categorising, storing and retrieval,
focused around “what is done” and “who does it”.
The key challenge in efficiency oriented approaches is that the perception can be quite subjective based
on the needs of the departments or teams - which leads to a variety of approaches across the length and
breadth of any enterprise.
There are several implications of using such a variety of approaches\,---\,chief amongst which is the
reconciliation effect expended every time a deviation has to be managed,
exceptions have to be handled or a change introduced in any activity.
Targeting efficiency of reconciliations through techniques such as focus-groups, standardisation of documentation etc
is a typical response of most organisations.

Taking the intuitive notion of a Business Identity (as a conceptual anchor for an organisation’s engagement with
its stakeholders) forward, enterprises can benefit from using a simple, Organization-wide concept anchor
focused around “why is it done” for its activities. This anchor need not be prescriptive to change the way an
organisation manages its activities but descriptive to enhance the ability to more effectively leverage information
within the artefacts.

One candidate for such an intuitive anchor is the Contract.
In fact, a contract is a memorandum of understanding (MOU), an agreement outlined in a formal document
that may become a legal Contract.
If we take the Business Identity\index{business!identity} as a basis for viewing an organisation’s product and services,
a contract can be seen as a specific engagement around which the activities of various departments and teams are
organised.
More specifically, a contract can be seen as a collection of related commitments, potentially spread over time,
which provides the necessary context for any activity in an organisation.
Interestingly, while its well-understood that a contract is a definitive source of specifics around any
agreement with stakeholders, its currently referred to only in the extreme case of legal recourse in
handling disagreements.
Another perception of Contracts is that it does not cover all collaboration contexts which guide activities
in an organisation.

Contracts have several features which help align the BBusiness Identity\index{business!identity} with the activities
within an Enterprise\,---\,and the wider Ecosystem, for that matter.

\begin{enumerate}[label=\alph*]
    \item Together with the legal system and societal norms, Contracts provide a contextual basis for
          belief/trust between an Enterprise and its stakeholders in the ecosystem
    \item Contracts are specific to the parties involved and provide an ideal contextual basis for
          (hyper)personalisation of products and services\,---\,through bounded simulation of contract performance
    \item Measures of performance in organisational activities, such as cost, operational efficiencies and
          innovation effectiveness can be aligned to contractual performance contexts
\end{enumerate}

We now look at two key elements of organizational activity (enabled by \gls{ekg}) aimed at orienting and
realising business strategy, through the lens of \nameref{sec:ekg-mm-business-identity}\index{business!identity}
and Contracts:

\begin{enumerate}
    \item %
% A.1 Business Strategy Actuation -- Summary
%
\textbf{\nameref{ch:ekg-mm-business-strategy-actuation}} is a business identity\index{business!identity} oriented
process that senses, understands and communicates business outcome-based assessment data to focus energy and
resources on \iindex{strategic objectives}.\index{business!outcome}
Actuation measurement values\index{measurement!values} are used to assess whether \iindex{strategic objectives}
and associated business goals\index{business!goals} are being fulfilled.

    See component~\ref{ch:ekg-mm-business-strategy-actuation} \nameref{ch:ekg-mm-business-strategy-actuation}
    on page \pageref{ch:ekg-mm-business-strategy-actuation}

    \item %
% A.2 Business Model Elaboration -- Summary
%
\textbf{\nameref{ch:ekg-mm-business-model-elaboration}}\index{business!model}\index{business!model elaboration}
is the process of providing further detail of business strategy\index{business!strategy}.
It may include \iindex{decision rationale} that explains how the business identity\index{business!identity}
data infrastructure\index{data!infrastructure} and algorithms enable the how and why planned actions
would create, deliver, and capture value.


    See component~\ref{ch:ekg-mm-business-model-elaboration} \nameref{ch:ekg-mm-business-model-elaboration}
    on page \pageref{ch:ekg-mm-business-model-elaboration}
\end{enumerate}

%
% What is Strategy?
%
\section*{What is Strategy?}

Competitive strategy is about standing out from competitors by offering a unique mix of value.
Michael E. Porter defines strategy as;
\emph{“Strategy is the creation of aunique valuable position; involving a different set of activities”}\autocite{what-is-strategy}.

Valuable positions emerge from three different sources.
First, positioning can be based onproducing a particular set of products or services.
Second, positioning can alsobe achieved by targeting the needs of a particular customer segment.
Third, thelast basis for positioning is based on the way customers can be reached as theway to reach a
particular segment might differ from other ones even thoughtheir needs are very similar.
For example, customer could be geographically spread or could be served only through specific channels.
Building a strategic position also involves trade-offs in terms of
(1) image or reputation,
(2) activities and resources, and
(3) priorities.

Even though these trade-offs might seem undesirable at first sight,
they are necessary for a sustainable strategic position according to Porter.
These trade-offs reduce the risk of imitation and straddling.
The former occurs when a competitor repositions itself to obtain the valuable strategic position of another company.
The latter is far more common and occurs when a straddler seeks to obtain the advantages of
a valuable strategic position while maintaining its existing position.
However, animitator or straddler may encounter many problems when trying to obtain astrategic position
different from the one it already has.
For example, a premium retailer may encounter many problems while trying to position itself as a low-price retailer
as both positions require different priorities, resources and activities.
This transition may even jeopardize the retailer’s current strategic position and confuse its consumers
by showing them two incongruent images.
Therefore, Porter also defines strategy as;
\emph{“Strategy is making trade-offs incompeting. The essence of strategy is choosing what not to do.”}

Positioning determines not only what activities a company will carry out and howit will set up individual
activities but also how activities relate to one another.
The latter characteristic is called fit and the most valuable fit is strategy-specific because it enhances a
position’s uniqueness and amplifies trade-offs.
Creating a fit between activities that is difficult to imitate also creates a sustainable strategic position.
Finally, Michael E. Porter also defines strategy as;
\emph{“Strategy is creating fit among a company’s activities.
The success of a strategy depends on doing many things well\,---\,not just a few\,---\,and
integrating among them.”}

TODO: What's the link to EKG
TODO: What's the link to Operating Model


\section*{Maturity Levels}

The first three overall maturity levels for the capabilities in the business pillar are:

\begin{itemize}[leftmargin=.48in,font=\bfseries]
    \item [Level 1] Stakeholders recognise business opportunities in scaling and amplifying capabilities through
          \glspl{ekg}.
          The first internal champion is seeking to socialise strategic business cases, supports innovation,
          and is willing to take on the disruption challenge.
    \item [Level 2] Stakeholders adopt a “knowledge-centric” mindset in their tactics to strengthen focus on
          strategic business value. Management elevates the knowledge graph as an organizational and funding priority.
    \item [Level 3] Strong collaboration between various business and support units to prioritise
          strategic business cases.
\end{itemize}
