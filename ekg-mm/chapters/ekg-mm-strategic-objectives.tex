\chapter{Strategic Objectives}
%
% TODO: Add all the concepts to the index
%
\begin{itemize}[leftmargin=2in,font=\bfseries]
  \item [Business Strategy] corporate objectives, use cases and organizational mechanisms necessary for sustainable business value from the \glsxtrfull{ekg}
  \begin{itemize}[labelwidth=1.5in,leftmargin=0in]
    \item [Corporate Goals] Alignment (shared vision) on why the organization is building a knowledge graph
    \item [Business Unit Goals] Support of the \glsfmtshort{ekg} value proposition\index{\glsfmtshort{ekg}!value proposition} by
          key \glsxtrshort{lob} \iindex{stakeholders}
    \item [Organizational Considerations] Operational and resource plans for \gls{ekg} strategy implementation and governance
  \end{itemize}
  \item [Data Strategy] enterprise data management framework (policies, target data architecture, quality assurance,
        governance) necessary to support the knowledge graph environment
    \begin{itemize}[labelwidth=1.5in,leftmargin=0in]
      \item [Data Goals \& Objectives] Importance of unique identification and the value of unambiguous shared meaning
      \item [Knowledge Graph Positioning] Role of the \glsxtrshort{ekg} as the underlying \iindex{data!fabric}
            for the organization
      \item [Business Case] \gls{roi} of linked data as an essential component of operational infrastructure
    \end{itemize}
  \item [Technology Strategy] environment for successful knowledge graph implementation including direction for
        \iindex{physical infrastructure}, applications and \iindex{process automation}
    \begin{itemize}[labelwidth=1.5in,leftmargin=0in]
      \item [Infrastructure Strategy] Physical infrastructure (i.e. \iindex{cloud}, \iindex{containerization},
            software layer) for the \glsxtrshort{ekg}
      \item [Application Strategy] Applications approach including rationalization, build vs. buy and standards adoption
      \item [Automation Strategy] Use of \glsfirst{ai} and \iindex{robotic process automation}
    \end{itemize}
\end{itemize}

~\\

\begin{description}[nosep,font=\bfseries]

  \item [High-Level Business Goals]
  Create integrated views, enhance product innovation, profile behavior, 
  determine preferences, understand relationships, implement target selling, 
  determine customer and product \gls{roi}, perform predictive modeling,
  define social connections, segment customers, enhance product satisfaction, 
  understand the dynamics of the market, operate with more agility, 
  maximize time-to-market.
  ~\\
  \begin{itemize}
    \item Do LOB stakeholders clearly understand the relationship between data management and business objectives
    \item Is all the data that is important to meet business priorities been defined and classified
    \item Is the data management strategy aligned with business priorities, implementation plans,
          technical capabilities and operational processes
    \item Has the data management strategy\index{data management!strategy} been mapped to
          business and organizational objectives\index{business!objectives}\index{organizational!objectives}
    \item Have business use cases and user stories been defined and aligned to data concepts
    \item Have LOB business outcomes (and dependencies) been defined and sequenced across the organizations
    \item Is there alignment between organizational/business objectives and data (concepts and repositories)
    \item Has the organization defined and aligned data metrics and \glspl{kpi} with business objectives
    \item Has the organization defined the business architecture\index{business!architecture}
          for both strategic and tactical objectives
    \item Is the data strategy aligned with lines of business objectives\index{business!objectives}
    \item Have service level agreements been defined and verified for critical systems and processes
    \item Are the lines of business engaged in (and understand the rationale of) the data management program
    \item Are lines of business and functional organizations committed and accountable to the data management objectives
  \end{itemize}

  ~\\

  \item [High-Level Technology Goals]
  Optimize infrastructure investments, automate business processes, 
  perform security surveillance, protect privacy, support continuous deployment. \\

  \begin{itemize}
    \item Does an integrated technology architecture strategy exist (and has it been implemented)
    \item Have IT and platform governance processes been defined and aligned with the data management strategy
    \item Do executive stakeholders have confidence in the ability of IT to manage realignment of fragmented architecture to meet strategic objectives (i.e. customer 360 and automated regulatory reporting)
    \item Has the data storage strategy been defined and aligned with the goals of data reconstruction, security and archive
  \end{itemize}

  ~\\

  \item [High-Level Data Goals]
  Adopt data management principles\index{principles!data management} of identity and meaning,
  \index{principles!Identity}\index{principles!Meaning}  monitor and ensure F-F-P quality,
  ensure data accessibility, eliminate data duplication, reduce reconciliation,
  govern the data lifecycle, control data at source, 
  control the data manufacturing process. \\

  \begin{itemize}
    \item Is there a clearly defined and sanctioned data strategy for the organization (aligned to organizational and business objectives)
    \item Does the organization have “data management delusions” and are they aware of their fallacy
    \item Does the organization have a plan on how to execute their data-centric strategy
    \item Has the data management business case been linked to organizational strategy and business pain points
    \item Are data requirements defined and aligned with funding processes
    \item Is there a mechanism for obtaining and verifying data management feedback from stakeholders
    \item Have the full suite of policies for data management been defined, approved and implemented
    \item Does the organization document and track the data production and consumption process
          (where data lives, the applications that are used, how it flows and where transformation occurs)
    \item Have logical and conceptual data models been defined and verified
    \item Is meaning of data in \glspl{sor} verified and locked down
    \item Are the criteria for designating criticality (and other classifications) consistent and scalable
    \item Does the organization have a data management and governance strategy to deliver against business objectives
    \item How does the organization evaluate the costs and effectiveness of the data strategy
    \item Has the data management strategy been translated into a operational roadmap
    \item Is funding and resource allocation plan in place to deliver against the data strategy
    \item Are communications, positioning and training programs about data management designed and operational
  \end{itemize}

  ~\\

  \item [High-level Organizational Goals]
  Perform flexible analysis, trust, operational resiliency, achieve efficiency/\-save money,
  comply with regulatory obligations, comply with contractual obligations, avoid fines,
  mitigate risk, preventing fraud, manage organizational change, enhance worker satisfaction,
  aggregate reporting, negotiate smart contracts with vendors, leverage capital,
  enhance market position, manage TCO and profitability. \\

  \begin{itemize}
    \item Does the organization view data as an instrument to transform the business
    \item Do the executive stakeholders within the organization understand the reasons why data is not harmonized across repositories and business processes (causes of incongruence)
    \item Do executive stakeholders understand the business rationale for establishing a “data control environment”
    \item Have high-level organizational goals been translated into data concepts
    \item Are knowledge workers focused on value-added activities
    \item Are key stakeholders committed to the \iindex{principles} and priorities of the data management program
    \item Does the organization have the skill sets and people talent to implement the data management strategy
  \end{itemize}

\end{description}










