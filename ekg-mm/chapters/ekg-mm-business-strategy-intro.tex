\chapter{What is Strategy?}\label{ch:ekg-mm-what-is-strategy}

Competitive strategy is about standing out from competitors by offering a unique mix of value.
Michael E. Porter defines strategy as;
\textit{“Strategy is the creation of a unique valuable position; involving a different set of activities”}
\autocite{what_is_strategy}.

Valuable positions emerge from three different sources.
First, positioning can be based on producing a particular set of products or services.
Second, positioning can also be achieved by targeting the needs of a particular customer segment.
Third, the last basis for positioning is based on the way customers can be reached as the way to reach a
particular segment might differ from other ones even though their needs are very similar.
For example, customer could be geographically spread or could be served only through specific channels.
Building a strategic position also involves trade-offs in terms of
(1) image or reputation,
(2) activities and resources, and
(3) priorities.

Even though these trade-offs might seem undesirable at first sight,
they are necessary for a sustainable strategic position according to Porter.
These trade-offs reduce the risk of imitation and straddling.
The former occurs when a competitor repositions itself to obtain the valuable strategic position of another company.
The latter is far more common and occurs when a straddler seeks to obtain the advantages of
a valuable strategic position while maintaining its existing position.
However, an imitator or straddler may encounter many problems when trying to obtain a strategic position
different from the one it already has.
For example, a premium retailer may encounter many problems while trying to position itself as a low-price retailer
as both positions require different priorities, resources and activities.
This transition may even jeopardize the retailer's current strategic position and confuse its consumers
by showing them two incongruent images.
Therefore, Porter also defines strategy as;
\textit{“Strategy is making trade-offs incompeting. The essence of strategy is choosing what not to do.”}

Positioning determines not only what activities a company will carry out and how it will set up individual
activities but also how activities relate to one another.
The latter characteristic is called fit and the most valuable fit is strategy-specific because it enhances a
position’s uniqueness and amplifies trade-offs.
Creating a fit between activities that is difficult to imitate also creates a sustainable strategic position.
Finally, Michael E. Porter also defines strategy as;
\textit{“Strategy is creating fit among a company’s activities.
The success of a strategy depends on doing many things well\,---\,not just a few\,---\,and
integrating among them.”}

\TODO[inline]{JAG>Carlos, would be good to have a section here linking it to EKG}

\TODO[inline]{JAG>Carlos, what's the link to Operating Model?}

