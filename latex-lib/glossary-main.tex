%%
%% The main glossary, use this for any terms that are NOT:
%%
%% - Concepts (use glossary-concepts.tex for that)
%% - Ontologies (use glossary-ontologies.tex for that)
%%
%% Do also not store any customer specific terms here, use
%% glossary-<customer-code>.tex for that.
%%

\newglossaryentry{master}{
    type=\glsdefaulttype,
    name={Master-data},
    description={Business critical data about parties, places and things}
}

\newglossaryentry{non-master}{
    type=\glsdefaulttype,
    name={Non-master data},
    description={Transactional data}
}

\newglossaryentry{temporal}{
    type=\glsdefaulttype,
    name={Temporal},
    description={Relating to Time}
}

\newglossaryentry{edgc}{
    type=\glsdefaulttype,
    name={Enterprise Data Governance Council},
    description={An enterprise governing body responsible for the data governance strategy,
    setting the organization-wide data policies and standards, and communicating them
    to enforce the data governance program}
}

\newglossaryentry{data-stewards}{
    type=\glsdefaulttype,
    name={Data Stewards},
    description={One who establishes data requirements and assesses the
    quality of the data in the data stores}
}

\newglossaryentry{bitemporality}{
    name={Bi-temporality},
    type=\glsdefaulttype,
    description={Bi-temporality is a feature of a system (or an Ontology) that allows for the recording of timestamps along two time lines:
    the time when the event was happening in the real world and the time when the event was recorded. See also "multi-temporality".}
}

\newglossaryentry{system-architecture}{
    type=\glsdefaulttype,
    name={system architecture},
    description={
        A system architecture is the conceptual model that defines the structure, behavior, and more views of a system.
    }
}

\newglossaryentry{bitemporality}{
    name={Bi-temporality},%
    type=\glsdefaulttype,%
    description={Bi-temporality is a feature of a system (or an Ontology) that allows for the recording of timestamps along two time lines:
    the time when the event was happening in the real world and the time when the event was recorded. See also "multi-temporality".}
}

\newglossaryentry{system-architecture}{
    type=\glsdefaulttype,
    name={system architecture},
    description={
        A system architecture is the conceptual model that defines the structure, behavior, and more views of a system.
    }
}

