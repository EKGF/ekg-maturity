%%
%% The main glossary, use this for any terms that are NOT:
%%
%% - Concepts (use glossary-concepts.tex for that)
%% - Ontologies (use glossary-ontologies.tex for that)
%%
%% Do also not store any customer specific terms here, use
%% glossary-<customer-code>.tex for that.
%%

\newglossaryentry{master}{
    type=\glsdefaulttype,
    name={Master-data},
    description={Business critical data about parties, places and things}
}

\newglossaryentry{non-master}{
    type=\glsdefaulttype,
    name={Non-master data},
    description={Transactional data}
}

\newglossaryentry{cde}{
    type=\glsdefaulttype,
    name={Critical data element},
    plural={Critical data elements},
    description={
        this concept is defined in many different ways. One way is to say that "critical data elements are
        those data elements that are critical to success in a specific business area".
        This could be a regulatory area such as BCBS 239 or the enterprise as a whole or any other use case area.
        In an \gls{ekg} context, especially at the higher levels of \gls{ekgmm} maturity,
        \myuline{all data is critical} since the criticality of data is always dependent on the context.
        For any given data point, there is a raison d'être in a given context.
        See also \chapterref{sec:ekgmm-b-4-5}{Critical Data Elements}{\glsfmtshort{ekgmm}}.
    }
}

\newglossaryentry{temporal}{
    type=\glsdefaulttype,
    name={Temporal},
    description={Relating to Time}
}

\newglossaryentry{edgc}{
    type=\glsdefaulttype,
    name={Enterprise Data Governance Council},
    description={%
        is an enterprise governing body responsible for the data governance strategy,
        setting the organization-wide data policies and standards, and communicating them
        to enforce the data governance program
    }
}

\newglossaryentry{data-steward}{
    type=\glsdefaulttype,
    name={Data Steward},
    plural={Data Stewards},
    description={%
        One who establishes data requirements and assesses the
        quality of the data in the data stores
    }
}

\newglossaryentry{bitemporality}{
    name={Bi-temporality},
    type=\glsdefaulttype,
    description={%
        a feature of a system (or an Ontology) that allows for the recording of timestamps along two timelines:
        the time when the event was happening in the real world and the time when the event was recorded.
        See also "multi-temporality".
    }
}

\newglossaryentry{system-architecture}{
    type=\glsdefaulttype,
    name={system architecture},
    description={%
        the conceptual model that defines the structure, behavior, and more views of a system.
    }
}

\newglossaryentry{organizational-accountability} {
    type=\glsdefaulttype,
    name={organizational accountability},
    plural={organizational accountabilities},
    description={%
        is about defining the company's mission, values, and goals,
        as well as everyone's role in working toward them.
        It's about holding employees and executives responsible for accomplishing these goals, completing assignments,
        and making decisions that deliver on these expectations.
    }
}

\newglossaryentry{technology-stack} {
    type=\glsdefaulttype,
    name={technology stack},
    plural={technology stacks},
    description={%
        ...
    }
}

\newglossaryentry{data-incongruence} {
    type=\glsdefaulttype,
    name={data incongruence},
    plural={data incongruences},
    description={%
        ...
    }
}

\newglossaryentry{operating-model} {
    type=\glsdefaulttype,
    name={operating model},
    plural={operating models},
    description={%
        is both an abstract and visual representation of how an organization delivers value to its customers or
        beneficiaries as well as how an organization actually runs itself.
    }
}
