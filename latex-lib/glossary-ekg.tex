%%
%% The main glossary (but only the ekg related terms of the main glossary),
%% use this for any terms that are NOT:
%%
%% - Concepts (use glossary-concepts.tex for that)
%% - Ontologies (use glossary-ontologies.tex for that)
%%
%% Do also not store any customer specific terms here, use
%% glossary-<customer-code>.tex for that.
%%
\newacronym[see={[Glossary:]{ekg:ekg}}]{ekg}{EKG}{Enterprise Knowledge Graph}% ekg the acronym
%\newglossaryentry{ekg}{% ekg the acronym
%    type=\acronymtype,
%    name={EKG},
%    description={Enterprise Knowledge Graph},
%    first={Enterprise Knowledge Graph (EKG)\glsadd{ekg:ekg}},
%    shortplural={EKGs},
%    longplural={Enterprise Knowledge Graphs},
%    see=[Glossary:]{ekg:ekg}
%}

\newglossaryentry{ekg:ekg}{% ekg:ekg the glossary entry
    type=\glsdefaulttype,
    name={EKG},
    description={%
        much like "the web" is a virtual concept, an EKG is a virtual concept that combines all
        information and knowledge of an enterprise or ecosystem.
    }
}

\newglossaryentry{ekg:coe}{
    type=\glsdefaulttype,
    name={\gls{coe} for the \gls{ekg}},
    description={%
        the group of people that is overseeing the \gls{ekg}
    }
}

\newglossaryentry{ekg:architecture:system}{
    type=\glsdefaulttype,
    name={EKG system architecture},
    description={%
        the logical \gls{system-architecture} of \gls{ekg} is divided into multiple layers or environments.
    }
}

\newglossaryentry{ekg:platform}{
    type=\glsdefaulttype,
    name={EKG/Platform},
    parent={ekg:architecture:system},
    description={%
        a logical \gls{system-architecture} component, the layer of software services that provide
        and serve the \gls{ekg} to end-users and other systems.
        The platform logically is a set services that enforce any of the specified policies in the
        \glspl{ekg:sdd} that have been published in the \gls{ekg}.
    }
}

\newglossaryentry{ekg:storage}{
    type=\glsdefaulttype,
    name={EKG/Storage},
    text={EKG/Storage},
    parent={ekg:architecture:system},
    description={%
        a logical \gls{system-architecture} component, the layer of data-storage services such as a "Triplestore"
        or an "Object Store" that serve the various other layers in the \gls{ekg:architecture:system}.
    }
}

\newacronym[see={[Glossary:]{ekg:storage:objectstore}}]{s3}{S3}{Simple Storage Service}

%\newglossaryentry{s3}{% s3 the acronym, refers to "object store"
%    type=\acronymtype,
%    name={S3},
%    text={S3},
%    description={Simple Storage Service},
%    first={Simple Storage Service (S3)\glsadd{ekg:storage:objectstore}},
%    shortplural={S3-services},
%    longplural={S3-services},
%    see=[Glossary:]{ekg:storage:objectstore}
%}
\newglossaryentry{ekg:storage:objectstore}{
    type=\glsdefaulttype,
    name={object store},
    text={object store},
    description={%
        "object store" in the context of \gls{ekg:architecture:system} refers to "object storage" or
        "cloud object storage"
        (such as \href{https://aws.amazon.com/what-is-cloud-object-storage/}{Amazon's AWS \gls{s3} service}).
        the \gls{ekg:storage} layer of \gls{ekg:architecture:system} always contains an object store (supporting the
        s3 api, usually the product \href{https://min.io}{MinIO} is used for that).
        Many backend storage mechanisms can be supported e.g. NAS, HDFS, Azure, Google Storage etc.
    }
}

\newglossaryentry{ekg:dataops}{
    type=\glsdefaulttype,
    name={EKG/DataOps},
    text={EKG/DataOps},
    first={EKG/DataOps\glsadd{ekg:dataops:practice}\glsadd{ekg:dataops:environment}},
    description={%
        the word EKG/DataOps has two meanings: a) it stands for a "practice" (see \gls{ekg:dataops:practice}) and
        b) it serves as the name of a logical \gls{system-architecture} component,
        the environment in which all "DataOps Pipelines" are running.
    }
}

\newglossaryentry{ekg:dataops:practice}{
    type=\glsdefaulttype,
    name={practice},
    text={EKG/DataOps Practice},
    parent={ekg:dataops},
    description={%
        the DataOps practice is one of the practices that the \gls{ekg:coe} executes...\improve{add definition of DataOps}
    }
}

\newglossaryentry{ekg:dataops:environment}{
    type=\glsdefaulttype,
    name={EKG/DataOps Environment},
    text={EKG/DataOps Environment},
    parent={ekg:architecture:system},
    description={%
        a logical \gls{system-architecture} component, the environment where \glspl{ekg:dataops:pipeline} run.
    }
}

\newglossaryentry{ekg:dataops:pipeline}{
    type=\glsdefaulttype,
    name={pipeline},
    text={EKG/DataOps Pipeline},
    parent={ekg:dataops},
    description={%
        a DataOps pipeline in the \gls{ekg:dataops:environment} is a series of programs, called "steps",
        that are run in sequence where the first step captures data in any given format from a given source
        and the last step produces an output file in any given format.
    }
}

\newglossaryentry{ekg:monitoring}{
    type=\glsdefaulttype,
    name={EKG/Monitoring},
    description={..todo..}
}

\newglossaryentry{ekg:sdd}{
    type=\glsdefaulttype,
    name={self-describing dataset},
    text={self-describing dataset},
    description={%
        An \gls{ekg} is logically composed of a set of \textit{self-describing datasets} that provide information about
        lineage, provenance, pedigree, maturity, quality, governance, entitlement policies, retention policies,
        security labels, IP policies, pricing policies, caching policies, organizational ownership, accountabilities,
        data quality feedback loop, issue management policies and so forth.
        Any given system owner that connects their data to the \gls{ekg} becomes in fact a publisher of a
        self-describing dataset and can therefore control how \myuline{their} data is handled by the \gls{ekg:platform}.
        The \gls{ekg:platform} consists of services that enforce any of the specified policies in the
        self-describing dataset.
        At the technical level, datasets can be files or streams or API definitions etc.
    }
}

\newglossaryentry{ekg:predicate}{
    type=\glsdefaulttype,
    name={predicate},
    text={predicate},
    description={%
        asserting an "\gls{rdf} triple" says that some relationship\,---\,indicated by the predicate\,---\,holds
        between the resources denoted by the subject and object.
        This statement corresponding to an RDF triple is known as an RDF statement.
        The predicate itself is an \gls{iri} and denotes a property, that is,
        a resource that can be thought of as a binary relation.
    }
}

\newglossaryentry{ekg:iri}{
    name={EKG/IRI},%
    text={EKG/IRI\glsadd{iri}},
    type=\glsdefaulttype,%
    first="Enterprise Knowledge Graph IRI (EKG/IRI)",
    firstplural="Enterprise Knowledge Graph IRIs (EKG/IRI)",
    description={
        An \gls{iri} that forms the identity of an object in the \gls{ekg}.
        Any given object in an \gls{ekg} has an EKG/IRI for which special rules are defined by the \gls{ekgf}.
        Not to be confused by \glspl{canonical-identifier}.
    }
}

\newglossaryentry{ekg:iri:predicate}{
    type=\glsdefaulttype,
    name={predicate-IRI},
    text={predicate-IRI},
    description={%
        see \gls{ekg:predicate}
    }
}

\newglossaryentry{canonical-identifier}{ % TODO: Since this is an EKG specific explanation we should prefix it with ekg:
    name={canonical identifier},%
    type=\glsdefaulttype,%
    text={Canonical Identifier},%
    description={A permanent identifier for an object, distinguished within an \gls{ekg}.}
}

