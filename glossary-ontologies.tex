%%
%% Glossary of Ontology Names. Keep this seperate from the normal glossary.tex file,
%% it will be printed as a separate glossary.
%%
\newglossaryentry{ont:legal-entity}{
    type=ont,
    name={Legal Entity Ontology},
    description={An ontology, or set of ontologies (like is the case with the FIBO Business Entities module), that describes the domain of Legal Entities}
}
\newglossaryentry{ont:prov}{
    type=ont,
    name={W3C PROV Ontology},
    description={The PROV Ontology provides a set of classes, properties, and restrictions that can be used to represent and interchange provenance information generated in different systems and under different contexts. It can also be specialized to create new classes and properties to model provenance information for different applications and domains.}
}
\newglossaryentry{ont:captured-schema}{
    type=ont,
    name={Captured Schema Ontology},
    description={An ontology that describes the physical schema of a given data source.}
}
\newglossaryentry{ont:data-profile}{
    type=ont,
    name={Data Profile Ontology},
    description={An ontology that describes all results gathered during a data profiling operation.}
}
\newglossaryentry{ont:workflow}{
    type=ont,
    name={Workflow Ontology},
    description={An ontology that describes any kind of workflow at an abstract level.}
}
\newglossaryentry{ont:entitlement}{
    type=ont,
    name={Entitlement Ontology},
    description={An ontology that describes any kind of "entitlement" (in the sense of access to information) or "entitlement policy" at an abstract level, comprising anything related to "access control" (which is further modelled in more specialised ontologies such as the Access Control Ontology) and any other aspect of Entitlement Management.}
}
\newglossaryentry{ont:access-control}{
    type=ont,
    name={Access Control Ontology},
    description={An ontology that models and describes Role Based Access Control (RBAC) in all its details.}
}
\newglossaryentry{ont:dublin-core}{
    type=ont,
    name={Dublin Core},
    description={The Dublin Core is a Schema (not really an Ontology in the OWL2 sense) with a set of vocabulary terms that can be
    used to describe digital and physical resources}
}
\newglossaryentry{ont:darwin-core}{
    type=ont,
    name={Darwin Core},
    description={The Darwin Core is a schema (not really an Ontology in the OWL2 sense) with a set of vocabulary terms that describe
    the biodiversity and is developed and maintained by Biodiversity Information Standards Group}
}
\newglossaryentry{ont:changeset}{
    type=ont,
    name={Changeset Ontology},
    description={This vocabulary defines a set of terms for describing changes to resource descriptions}
}

