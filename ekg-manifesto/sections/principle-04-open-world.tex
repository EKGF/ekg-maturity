\section{Open World}\label{sec:ekg-principle-open-world}

Information can vary over time, come from many internal and external sources, 
and be based on different identifiers and models. 
These \textit{multiple versions of the truth} need to be reconciled on access by context.

Different sources might not always be consistent about the same \gls{data-point}. 
And that may be legitimate for various business, geographical, privacy, legal or timing reasons. 
Rather than try to force everything into a single overruling set of facts, 
an \gls{ekg} allows them to coexist, with the access context used to make 
coherent selections which are consistent within themselves.
In order to make those selections multiple people and systems must have transparent access 
to all facts (including source, identity, meaning and value-at-time information) about all objects.
Machine-executable business rules may be used at query time to join instances of data and e
stablish quality rankings.

\paragraph{Rationale}

This approach allows the \gls{ekg} to represent the sometimes messy reality, which encompasses 
not only a variety of different orgarnizations and systems within the enterprise, but also external sources.

\paragraph{Implications}

Attention needs to be paid to maintaining sufficient context with each data set, 
and considering what is needed for each data usage use case or context, including quality.
