\section{Meaning}\label{sec:ekg-principle-meaning}
\index{principles!Meaning}

The meaning of every \gls{data-point} must be directly resolvable
to a machine-readable definition in verifiable formal logic.

A \gls{data-point} combines an object\,---\,using its \gls{ekg:iri} as its identifier\,---\,with the value
of a \textit{property} in some context.
Hence data is expressed at its most granular level for both data at rest and data in motion.

The property itself is always an \gls{iri}, often called "predicate-IRI", that refers to an object\footnote{%
    the official term in the \gls{rdf} standard for this object is "resource"%
} that represents "the meaning" of the given \gls{data-point}.
This object has its own identity and is defined through further properties
based on logic that allow information to be rigorously combined, queried and inferred.
These properties that define properties\,---\,also called "axioms"\,---\,are standardized by means of the \glsfirst{owl2}
by the \gls{w3c} and are grouped into \textit{"OWL ontologies"} for management purposes.

% JAG>PER I don't think the sentence below is useful or makes sense...
%However the machine-readable logic does not replace natural language definitions that
%link the properties and statements to the real world.

\paragraph{Rationale} Expressing data at a granular level allows ultimate flexibility
for it to be sliced, diced, combined and aggregated.
This capability to combine and infer information is further enhanced by the use of \gls{property} definitions
built on logic.
Having the properties themselves be objects that can be looked up means that all data
is self-defining and carries its meaning with it.
Since the information is self-defining there is no fixed schema for the \gls{ekg}
as a whole and it can non-disruptively incorporate additional knowledge.

\paragraph{Implications} Some further discovery, with subject matter experts and
creators of source systems, is often needed to truly understand what a given
set of data really means and what can be inferred from it.
In other words you cannot rely on the name of a column in a spreadsheet.
A deceptively simple column name such as "number of European customers" leaves
open the meaning of "European" and "customer" and timing (when does one start
and stop being a customer?).
And different sources could have different interpretations of that same name.
The benefit is consistency, accuracy and the ability to make sound business decisions.

\paragraph{Advanced} at higher levels of \gls{ekg:platform} maturity the term \gls{data-point} may in fact become a more
complex data structure that is used "on the wire" that represents the \gls{data-point} at a more "holistic" level,
supporting \enquote{\glsfirst{mvot}}.
Since an \gls{ekg} supports many datasets that have overlapping information coming from multiple sources, there
could be:
\begin{enumerate}
    \item multiple \gls{ekg} identifiers (\glspl{ekg:iri}) for the same object
    \begin{itemize}
        \item One object can have multiple identifiers that can be linked together
              \footnote{%
                  for instance via owl:sameAs i.e.
                  "\href{https://www.w3.org/TR/owl2-syntax/\#Individual_Equality}{Individual Equality}"
              } and therefore be rightfully addressable with any of these identifiers.
    \end{itemize}
    \item multiple definitions of meaning
    \begin{itemize}
        \item One property of an object can have multiple definitions of meaning,
              for instance "legal name" can be defined in multiple ontologies and be semantically equivalent
              \footnote{%
                  See
                  \href{https://www.w3.org/TR/owl2-syntax/\#Equivalent_Object_Properties}{Equivalent Object Properties}
                  or
                  \href{https://www.w3.org/TR/owl2-syntax/\#Equivalent_Data_Properties}{Equivalent Data Properties}
              } or one property can be defined as a subproperty of another but broader semantic definition
              \footnote{%
                  See \href{https://www.w3.org/TR/owl2-syntax/\#Data_Subproperties}{Data Subproperties}
              }.
    \end{itemize}
    \item multiple equal or different values coming from multiple sources

    \item multiple versions over time of those values (temporality)
\end{enumerate}

For each of these four "axes"\,---\,identity, meaning, source, temporality\,---\,you could have multiple
options to choose from even while logically, from a user perspective, it's the same data point.
Advanced client applications, services or AIs can use these \glspl{data-point} to perform last-minute "at the edge"
computations around finding the right value from the right timeline and source with the right quality for the given
context.



