%
% C.2.1 Ontologies -- Contribution to the Enterprise
%
Ontologies are needed to truly understand what a given set of data really means and what can be inferred from it.
For example you cannot rely on the name of a column in a spreadsheet.
A deceptively simple column name such as "number of European customers" leaves open the meaning of
"European" and "customer" and timing (when does one start and stop being a customer?).
And different sources could have different interpretations of that same name.
The benefit is consistency, accuracy and the ability to make sound business decisions.
Having the models themselves be resources that can be looked up means that all data is self-defining and
carries its meaning with it.
In an \gls{ekg} there is no fixed set of ontologies so it can non-disruptively incorporate additional knowledge.
Ontologies allow data to be understood independent of the format/technology and the vocabulary used in
different communities, saving misunderstandings and battles over which word to use.

