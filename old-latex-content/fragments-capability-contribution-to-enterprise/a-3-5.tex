%
% A.3.6 Capability Map -- Contribution to Enterprise
%
As a business/IT transformation enabler, capabilities may provide certain benefits to enterprises:

\begin{itemize}
    \item \textbf{Capabilities provide business with a common language.}
          Capabilities offer business professionals and C-level executives a common understanding of which areas of an
          enterprise they need to address.
          The larger the enterprise, the more useful this idea is since business issues and strategy may revolve
          around many business units, lines of business, business processes, or even enterprise boundaries.
    \item \textbf{Capabilities may enable more precise investments.}
          Measuring the \gls{roi} of investment projects can be very difficult as the results might be spread across a
          myriad of different \glspl{lob} or business units.
          Instead of reconciling results from all these different places,
          one could merely look at the resources being allocated at a certain capability and this may allow
          business executives to have better estimates of the impact of their investments on the
          enterprise as a \textit{whole}.
    \item \textbf{Capabilities serve as a baseline for strategic planning, change management and impact analysis.}
          By shifting the focus from business units, lines of business, IT systems and business processes
          to business capabilities, one is able to increase the traceability of strategic decisions into the
          daily operations and the business performance of the enterprise as a whole.
          So, capabilities serve as a starting point for tracking the impact horizontal and vertical
          impacts of strategic and tactical decisions from the executive team.
    \item \textbf{Capabilities lead to better business service specification and design.}
          Capabilities provide business-focused abstraction of the functionalities and information that
          information systems must provide.
          For instance, such capabilities may help in the selection or construction of services in a
          \gls{soa} as they directly embody business requirements.
\end{itemize}

In a few words, capabilities provide business executives the possibility to cut through the complexity inherent
to most enterprises in order to make sounder decisions for strategic planning, impact analysis and investment planning.
Using capabilities enables business to make decisions based on \textit{what} needs to be resolved without initially
dealing with the \textit{how}.
At the same time, it provides a way to tie the \textit{how} to the results of the \textit{what}
for further validation that business efforts are delivery the expected results and for better alignment.
On the other hand, the alternative to this approach involves repeating the traditional silo-focused approach of
looking at hundreds or even thousands of IT systems,
before being able to look at actual business actions and business results.
