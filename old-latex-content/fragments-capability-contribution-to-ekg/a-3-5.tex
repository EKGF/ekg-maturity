%
% A.3.5 Capability Map -- Contribution to the EKG
%
The \textit{Business Capability Map} enables an enterprise to structure their Enterprise Knowledge Graph.
Each capability is a candidate to become a \textit{use case} for the \gls{ekg}.

A well-designed mature \gls{ekg} is a facade in front of all technical and organizational silos which means that\,---\,
without any further structure\,---\,users do not experience these silos anymore.\index{silo}
In many ways that may be a good thing but having silos\,---\,or different perspectives\,---\,can also be necessary.
The \gls{ekg} allows an organization to rethink their silos without being held back by their current data or technology
landscape (or "technical debt")\index{technical debt}.

The \textit{Business Capability Map} is the ideal initial structure for the new \gls{ekg} "silos".
Business Capability Maps are usually rather coarse-grained and visualized in a three-level hierarchy whereas the
structure of the \gls{ekg} goes much further than that.
Translating each capability to a "use case" is a good start but each of these use cases can be further broken down
into smaller use cases where each use case becomes a highly reusable component of the \gls{ekg}.
This leads to a hierarchical structure, a taxonomy so you will, of all use cases, also called "the \gls{uct}".
This use case tree, at the higher levels, corresponds with the Business Capability Map and allows a business and
its executives to "own" and control all the various parts of their \gls{ekg}.

