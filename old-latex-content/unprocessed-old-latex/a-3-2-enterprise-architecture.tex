%
% A.3.2 Enterprise Architecture -- Summary
%
After having defined an operating model, a company should then draw its enterprise architecture which is
defined as:

\somequote{
    the organizing logic for business processes and IT infrastructure reflecting the integration and
    standardization requirements of the company’s operating model.%
}{Ross, Weil and Robertson (p. 47, 2006)}{}

A core diagram is a high-level view of the processes, data, technologies,
and customers used in the company’s daily operations.
The utility of such a core diagram is to facilitate communication about the company’s needs between IT
and business managers.
This could help increase the success rate of IT projects in terms of business approval,
customer satisfaction and return on investment.

A core diagram reflects the following point:

\begin{itemize}
    \item \textbf{Core business processes:} the main business processes that the company needs to execute on a
          daily-basis and previously defined in the operating model.
    \item \textbf{Shared data driving core processes:} this is data shared across different end-to-end
          business processes by all the parties participating in these processes.
          These parties may be business units, departments, or even external parties such as suppliers.
    \item \textbf{Key linking and automation technologies:} The so-called middleware technologies are meant to
          integrate business processes and provide access to shared data cross business units, business functions, etc.
          \gls{erp} packages are an example of such a technology.
    \item \textbf{Key customers:} The primary customer groups of the company are also depicted in the
          core diagram of a company.
\end{itemize}

