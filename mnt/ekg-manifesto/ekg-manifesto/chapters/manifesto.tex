\chapter{Manifesto}\label{ch:ekg-manifesto}

\section{---title?---}

TODO: A section of "reasons why" in non-data / non-tech terms. This section is very early days and needs much more work.

\begin{enumerate}
    \item Much higher levels of transparency, sustainability, fairness and accountability are going to be required at all levels in any organization
    \item Human Capital needs to be known, valued, leveraged and optimized
    \item Data Capital needs to be known, valued, leveraged and optimized
    \item Increasing competiveness depends more and more on having the highest quality and depth of data, information and knowledge
    \item One "censored", biased, single version of the truth is no longer good enough for many\,---\,if not most\,---\,use cases in most domains
    \item The world becomes more and more polarised due to "information bubbles" that many people are not escaping from, more depth and connectedness is needed
    \item The world becomes more and more complex and harder to understand, a holistic view around every given topic, showing all viewpoints, would help people to understand and make better decisions
    \item ...
    \item your input is welcome!
\end{enumerate}

\section{All data connected}

\begin{enumerate}
    \item All data will be connected.
          \begin{itemize}
              \item Information, Knowledge, Meaning: it's all data.
              \item Knowledge \& Meaning will be captured as machine-readable executable models.
          \end{itemize}
    \item All data will be made available anywhere\,---\,within entitlement limits\,---\,%
          at any time to any device, node or edge.
    \item We embrace the \textit{Open World} and deal with the realities of \gls{mvot}.
    \item We combine the digital footprint of activities along with a digital representation of
          information and knowledge, from which an \gls{ekg} emerges.
    \item All your connected data is an \gls{ekg}.
    \item An \glstext{ekg:ekg} is connections of Knowledge Graphs across an Enterprise and beyond.
\end{enumerate}

\section{Standards}

\begin{enumerate}
    \item \Glspl{ekg} are based on standards and therefore interoperable across boundaries.
    \item There are many different types of standards that an \gls{ekg} needs to be able to deal with.
    \item Standards are described as machine-readable models\,---\,i.e. ontologies\,---\,that \glspl{ekg:platforms}
          can execute, interpret or enforce.
\end{enumerate}

\section{Data Markets}

\begin{enumerate}
    \item Any data source will be turned into a data publisher\,---\,or supplier\,---\,of one or more \glspl{sdd}.
          \begin{itemize}
              \item Any data sink will be turned into a data consumer\,---\,of one or more datasets.
          \end{itemize}
    \item Data suppliers and consumers will find each other via a data market using a standard "lingua franca" for the
          data itself, its meaning, all its associated policies and metadata and especially also its use cases as
          executable models.
    \item The data market manages the information supply chains between all the various suppliers and consumers.
    \item The global data market will consist of many other more specific data markets
          e.g. per industry or per enterprise.
    \item An \gls{ekg} is the combination of one or more data markets and the deployment of its use cases.
\end{enumerate}

\section{Open World}

\begin{enumerate}
    \item For any given "Thing"\,---\,i.e. an object\,---\,there may be many representations in many datasets.
    \item An object's representations may be different in shape, meaning, timeliness, relevance and quality\,---\,i.e.
          any given representation of information about a given object may represent a different version of the truth.
    \item All representations of any given object shall be linked via shared identifiers.
          \begin{itemize}
              \item Identifiers shall be meaningless, opaque, web-resolvable and universally unique.
                    See principle~\ref{sec:ekg-principles-identity}.
              \item An object can have multiple identifiers.
          \end{itemize}
    \item Any given object consists of 1 or more datapoints.
    \item A datapoint represents a logical property of a given object.
          \begin{itemize}
              \item The identifier for a Datapoint is the identifier of the object it belongs to plus at least one
                    identifier of the axiom that describes its meaning (which is also an object).
          \end{itemize}
    \item Datapoints for the same object can exist in many different datasets.
          \begin{itemize}
              \item with potentially multiple versions of the truth in terms of meaning, timeliness,
                    relevance and quality.
          \end{itemize}
    \item Any representation for any given \gls{data-point} of any data source shall be made available to any device,
          node or edge in the network within legal, policy and entitlement limits in real-time.
    \item Every "object" that is represented as data in whichever dataset anywhere, shall have an identifier
          that is universally unique, permanent, meaningless or opaque and therefore shareable,
          resolvable through the HTTP protocol.
          See \ref{sec:ekg-principles-identity}~\nameref{sec:ekg-principles-identity}.
\end{enumerate}

Work in progress notes: that add more explanation to the above:

\begin{itemize}
    \item Data will be considered explained when its usage has no misconceptions nor ambiguities.
    \item The word “data” has a lot of different notions associated with it.
    We have these statements above, like “all data is connected”, “…data will be made available anywhere…”,
    which could reinforce a particular notion that data is something that “exists” all around us,
    like a digital footprint of activities.
    On top of these statements on “data”, the \gls{ekg} is introduced as a combination of one or more data markets,
    which could further reinforce that "\gls{ekg} is connected data".
    If that’s the notion that the manifesto wants to declare, it may appeal strongly to some sections more than others,
    like those dealing with consolidation, analysis, reporting, verification etc.
    If the manifesto’s intent is broader (here, I am very conscious of Carl’s note that it should stay away
    from any hubris!) it would help if there is a way to declare that information and knowledge is also data.
    And it is when we combine the digital footprint of activities along with a digital representation of
    information and knowledge, that a Knowledge Graph emerges.
    And on top of that notion, an Enterprise Knowledge Graph is connections of Knowledge Graphs across an Enterprise.
    The intent of the above is to appeal broadly to different sections of an Enterprise,
    accommodating different norms and notions (again referring to Carl’s insightful comment on Norms and Notions)

\end{itemize}
