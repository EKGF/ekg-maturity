\chapter{Mapping to FAIR data principles}

\section{Findable}

\begin{itemize}
    \item[F1] (Meta)data are assigned a globally unique and persistent identifier
    \item[F2] Data are described with rich metadata (defined by R1 below)
    \item[F3] Metadata clearly and explicitly include the identifier of the data they describe
    \item[F4] (Meta)data are registered or indexed in a searchable resource
\end{itemize}

\paragraph{Differences}

\gls{ekg} Principles are slightly more specific or prescriptive:

\begin{itemize}
    \item Metadata (objects) and data (objects) can potentially have multiple identifiers (but at least one).
    \item Those identifiers do not necessarily be "persistent" as long as they are (always) resolvable (through HTTP).
    \item The \gls{ekg} identifiers (\glspl{ekg:iri}) of data objects (but not necessarily metadata objects) should be
          "opaque" as in "meaningless", (relatively) safe to be emailed around, stored in other platforms,
          maximising "proliferation".
    \item FAIR principle F3 would be phrased the other way around: the data described by metadata refers to it via
          the metadata identifier (i.e. the predicate-IRI).
    \item FAIR principle F4 slightly differs as well, the \gls{ekg} Principles require metadata to be directly
          resolvable (via HTTP) machine-readable definitions of the semantics in verifiable formal logic
          (preferably OWL 2).
\end{itemize}

\section{Accessible}

Once the user finds the required data, they need to know how they can be accessed,
possibly including authentication and authorisation.

\begin{itemize}
    \item[A1] (Meta)data are retrievable by their identifier using a standardized communications protocol
    \item[A1.1] The protocol is open, free, and universally implementable
    \item[A1.2] The protocol allows for an authentication and authorisation procedure, where necessary
    \item[A2] Metadata are accessible, even when the data are no longer available
\end{itemize}

\paragraph{Differences}

\gls{ekg} Principles are slightly more specific or prescriptive:

\begin{itemize}
    \item \textit{"using a standardized communications protocol"} would be explicitly the HTTP protocol
          (or actually HTTPS/TLS) as a minimum requirement and in addition to that any other protocol,
          standardized or not.
    \item \textit{"metadata are accessible"} would be more explicit for the \gls{ekg}:
          all metadata has to be accessible through IRIs that are always "resolvable" via the HTTP protocol.
          In other words, make sure that all your OWL 2 ontologies or RDFS schema vocabularies are placed on
          highly available durable infrastructure that can always be accessed via HTTP.
\end{itemize}

\section{Interoperable}

The data usually need to be integrated with other data.
In addition, the data need to interoperate with applications or workflows for analysis, storage, and processing.

\begin{itemize}
    \item[I1] (Meta)data use a formal, accessible, shared, and broadly applicable language for knowledge representation.
    \item[I2] (Meta)data use vocabularies that follow FAIR principles
    \item[I3] (Meta)data include qualified references to other (meta)data
\end{itemize}

\paragraph{Differences}

\gls{ekg} Principles are slightly more specific or prescriptive:

\begin{itemize}
    \item In the \gls{ekg} the metadata that describes meaning i.e. the semantics
          (there are also many other types of metadata) that
          \textit{formal and broadly applicable language for knowledge representation}
          has to be preferably OWL 2 or at least \gls{ekg:rdf-schema} or SHACL.
\end{itemize}


\section{Reusable}

The ultimate goal of FAIR is to optimise the reuse of data.
To achieve this, metadata and data should be well-described so that they can be replicated and/or
combined in different settings.

\begin{itemize}
    \item[R1] Meta(data) are richly described with a plurality of accurate and relevant attributes
    \item[R1.1] (Meta)data are released with a clear and accessible data usage license
    \item[R1.2] (Meta)data are associated with detailed provenance
    \item[R1.3] (Meta)data meet domain-relevant community standards
\end{itemize}

The principles refer to three types of entities: data (or any digital object),
metadata (information about that digital object), and infrastructure.
For instance, principle F4 defines that both metadata and data are registered or indexed in a
searchable resource (the infrastructure component).

\paragraph{Differences}

\gls{ekg} Principles are slightly more specific or prescriptive:

\begin{itemize}
    \item \textit{optimization of the reuse of data is the ultimate goal}.
          It's an important goal for \gls{ekg} as well but reuse of knowledge and whole use cases\,---\,%
          with everything that comes with it\,---\,is an even higher level goal.
          Furthermore, overall connectedness of all data and knowledge is an equally important goal.
    \item \textit{metadata and data should be well-described} but should also be as "unbiased" as possible, not designed
          for one particular (set of) use case(s) but designed to represent the version of the truth that a given data
          source provides with the highest level of integrity.
          See principle \ref{sec:ekg-principle-self-describing} \nameref{sec:ekg-principle-self-describing}.
    \item \textit{Meta(data) are richly described with a plurality of accurate and relevant attributes}.
          \enquote{\textit{Richly described}} would not be specific enough for \gls{ekg}.
          It would have to be \textit{directly resolvable to a machine-readable definition in verifiable formal logic}.
    \item For all types of recognized metadata, which is metadata that services of the various \gls{ekg} platforms can
          recognize, the EKGF will specify accepted standards or define standards itself or in
          collaboration with partners like the \gls{omg}.
          Since an \gls{ekg} is a collection of \glspl{sdd}, each dataset will have its various types of metadata
          organized in a structured way.
\end{itemize}

