\section{Business Orientation}\label{sec:ekg-principle-business-orientation}

All information in the \gls{ekg}, and associated artifacts, 
are linked to defined and prioritized \textit{use cases}.
Nothing in the \gls{ekg} exists without a known business justification and purpose.

An \gls{ekg:use-case} encompasses a business narrative and outcome expressed in
business terms and links to relevant ontologies and \glspl{dataset}.

Importantly, the use cases for an \gls{ekg} can make use of lower level use cases, 
thus forming a \textit{use case tree}, though it is not a strict hierarchy 
since common use cases can be reusable. 
At an implementation level a use case may be associated with user stories that 
form the points of interctaion with end users or client systems.
The vision is that a fully fleshed out and implemented use case can be deployed as a reusable component.

\paragraph{Rationale}

Use cases anchor the \gls{ekg} to real business needs, and allow it to 
evolve incrementally while delivering business value. 
Without this, the tendency is often to focus on information modeling for its own sake, 
without a focus or rationale for what to include or exclude.
The use cases provide the basis for associating users with \gls{ekg} functionality 
and provide the context for information selection.

\paragraph{Implications}

The use cases themselves need to be developed and managed as part of the \gls{ekg} development method, 
and ideally as part of the \gls{ekg} itself.
