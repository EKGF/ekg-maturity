\section{Self-describing}\label{sec:ekg-principle-self-describing}
\index{principles!Self-describing}

An \gls{ekg} is composed of a set of \textit{self-describing datasets} that provide information about 
lineage, provenance, pedigree, maturity, quality, licensing and governance.

The properties in each data point are linked to their definition so the meaning is not in doubt. 
A \gls{dataset} definition supplements this with management information such as its pedigree (how/when 
was it derived/sourced?) and its provenance (where/who did it come from?). 
This applies whether the information is maintained in the \gls{ekg} itself or accessed/loaded 
from existing enterprise systems (data at rest); or received as data streams/messages (data in motion).

\paragraph{Rationale}

This information is essential for data selection, enforcing policy and management of the ecosystem as a whole.
As well as being essential for management, the definitions taken together comprise a 
knowledge \textit{catalog} for the content of the \gls{ekg}.

\paragraph{Implications}

The information needs to be maintained and made available on an ongoing basis.
It also needs to be sought out for external sourced data, whether accessed in place or loaded into an \gls{ekg} platform.
