\section{Identity}\label{sec:ekg-principles-identity}

Any object in the \gls{ekg} is identified with at least one universally unique, opaque, 
permanent and web-resolvable identifier called \gls{ekgiri}.

The identifier is permanent and used for the expression of facts about the object including relationships between objects. 
Additional identifiers may also be assigned, and they \textit{may} be more transient and reassigned.
\textit{Resolving} an identifier consists of using it in a query submitted or routed via an internet protocol.

\paragraph{Rationale}

While the semantic web generally allows for many and varied URIs, and this is still encouraged when integrating systems, 
there is benefit in being able to rely on one canonical and unchanging one, 
which can for example make the mapping of identities a many-to-one rather than a many-to-many task.

\paragraph{Implications}

There should be a mapping or service to resolve other names, keys or URIs to the canonical one. 
The set of canonical identifiers will generally be unique to each \gls{ekg}.
Since it is immutable, the canonical identifier will generally be \textit{opaque} 
and not be a human-readable name since even human names, company names, customer numbers, Social Security Numbers 
can change over time. 
In some specific cases, well-maintaind identifiers amy be suitable as the basis of a canonical identifier: 
for example \glspl{lei}, Financial Industry Global Identifiers (FIGIs). 
The \gls{ekgf} will maintain a list of these for convenience.
The use of multiple identities generally means that an \gls{ekg} should \textit{not} 
use the Unique Names Assumption (UNA) (where use of a different identifier would imply a different object). 
However the UNA \textit{would} apply specifically to the canonical identifier, and this should be ensured by any service.
  
