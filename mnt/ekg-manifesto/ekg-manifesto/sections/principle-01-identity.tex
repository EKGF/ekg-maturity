\section{Identity}\label{sec:ekg-principles-identity}

Any object in the \glsfmtshort{ekg} is identified with at least one \iindex{universally unique}, \iindex{opaque},
permanent, non-reassignable and \iindex{web-resolvable} identifier in the form of an \glsreverse{iri}
for the \glsfmtshort{ekg}\,---\,i.e. an \gls{ekg:iri}.

The \gls{ekg:iri} identifier is permanent, will be proliferated across the company's universe (including ecosystem),
and will be used for the expression of facts about the object including relationships between objects.

Additional non-\glsfmtshort{ekg} identifiers may also be assigned, and they may be human-readable,
"external" to the organization's \glsfmtshort{ekg} and be \iindex{transient} and reassignable.

\textit{Resolving} an identifier\index{identifier resolution} can be done in three ways:

\begin{enumerate}
    \item using it in a transaction\,---\,i.e. a query or update statement\,---\,submitted or routed via an
          internet protocol to a "lookup service" that translates one or more given "features" of an
          object to an \gls{ekg:iri}.
    \item constructing it via a standardised policy from key components and applying a hash and optionally
          signing it\,---\,where the object represented by the \gls{ekg:iri} may or may not already exist.
    \item constructing it by giving the object an \gls{ekg:iri} based on a random number in case the \gls{ekg} is
          the authoritative source for the given object.
\end{enumerate}

\paragraph{Rationale}

While the semantic web technologies\,---\,like \gls{rdf}\,---\,generally allow for many and varied \glspl{iri},
and this is still encouraged when integrating systems, there is benefit in being able to rely on one canonical
and unchanging one, which can for example make the mapping of identities a many-to-one rather than a many-to-many task.

In addition to that, to enhance the likelihood that various \glspl{ekg}\,---\,across departments, organizations or
ecosystems\,---\,can interoperate easily with each other, the use of standardized \glspl{ekg:iri} needs to be
encouraged since various \glspl{ekg} can come to the same identifiers independently, drastically increasing the
number of links across \glspl{ekg}.

\paragraph{Implications}

\begin{itemize}
    \item There should be a mapping or service to resolve other names, keys or \glspl{iri} to the \gls{ekg:iri}.
    \item Since it is immutable, the \gls{ekg:iri} will have to be \textit{opaque} i.e. not be a human-readable
          since even human names, company names, customer numbers, Social Security Numbers can change over time.
    \item Objects that already have a well-established \gls{rdf}-compliant and \gls{linked-data} compliant identifier
          may not necessarily need an additional \gls{ekg:iri}.
          In fact, they may already have one that is external to the company's \gls{ekg}.
          It is in many cases recommended to even give those well-established objects from well-established external
          datasets a company \gls{ekg:iri}. Examples of such objects are \glspl{lei} and \glspl{figi}.
          The \gls{ekgf} will maintain a list of these for convenience.
    \item The use of multiple identities generally means that an \gls{ekg} should \textit{not}
          use the \gls{una} (where use of a different identifier would imply a different object).
          However the \gls{una} \textit{would} apply specifically to the \gls{ekg:iri},
          and this should be ensured by any service.
\end{itemize}
  
