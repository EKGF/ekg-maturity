\section{Ecosystem}\label{sec:ekg-principle-ecosystem}

An enterprise will use a heterogenous set of technologies and data sources which
will be incorporated into the \gls{ekg} over time. 
All data entering the ecosystem are subject to service level agreements.

A true \gls{ekg} is a federation of multiple datasources and systems 
both within and external to the enterprise; seamlessly knitted together with standard protocols and APIs.
An \gls{ekg} needs to be managed and evolved as a fairly complex system of systems. 
To provide the flexibility and agility needed, its management needs to be automated, 
and linked to development processes, through the discipline of \textit{data operations}.

\paragraph{Rationale}

As an ecosystem, an \gls{ekg} can non-disruptively evolve over time in technology, 
scalability and information and business needs addressed.

\paragraph{Implications}

The management of the \gls{ekg} needs to be planned for and resourced.
It's important to coordinate with the owners of existing systems being used 
to provide capability under the \gls{ekg} umbrella.
